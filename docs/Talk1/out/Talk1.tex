\documentclass[xcolor=table]{beamer}
% generated by Madoko, version 1.1.6
%mdk-data-line={1}


\usepackage[heading-base={2},section-num={False},bib-label={hide},fontspec={True}]{madoko2}
%mdk-data-line={1;out\presentation.mdk:79}

    \ifbeamer\relax\else
      \providecommand{\usetheme}[2][]{}
      \providecommand{\usecolortheme}[2][]{}
      \providecommand{\usefonttheme}[2][]{}
      \providecommand{\pause}[1][]{}
      \providecommand{\AtBeginSection}[2][]{}
      \providecommand{\AtBeginSubsection}[2][]{}
      \providecommand{\AtBeginSubsubsection}[2][]{}
      \providecommand{\AtBeginPart}[2][]{}
      \providecommand{\AtBeginLecture}[2][]{}
      \providecommand{\theoremstyle}[2][]{}
      \makeatletter
      \def\newtheorem{\@ifstar\newtheoremx\newtheoremx}
      \makeatother
      \providecommand{\newtheoremx}[3][]{}{}
    \fi
%mdk-data-line={1;out\presentation.mdk:96}

    \ifbeamer\usetheme[]{singapore}\fi
\begin{document}



%mdk-data-line={9}
%mdk-data-line={9}
%mdk-data-line={10}
\mdxtitleblockstart{}
%mdk-data-line={10}
\mdxtitle{{\mdfontsize{3em}\mdline{10}Demo presentation in Madoko}}%mdk

%mdk-data-line={13}
\mdxsubtitle{{\mdfontsize{1.8em}\mdline{13}In both HTML and PDF}}%mdk
\mdxauthorstart{\Large{}
%mdk-data-line={18}
\mdxauthorname{{\mdfontsize{1.4em}\mdline{18}Daan Leijen}}%mdk

%mdk-data-line={21}
\mdxauthoraddress{\mdline{21}Microsoft Research}%mdk

%mdk-data-line={24}
\mdxauthoremail{\mdline{24}daan@microsoft.com}%mdk
}\mdxauthorend\mdtitleauthorrunning{}{}\mdxtitleblockend%mdk

%mdk-data-line={12}
\begin{mdframe}%mdk

\frametitle{Content}\label{heading-sec-content}%mdk%mdk
\mdline{14}
\begin{mdtoc}%mdk

\begin{mdtocblock}%mdk

\mdtocitemx{sec-content}{\mdref{sec-content}{1.\hspace*{0.5em}Content}}%mdk

\mdtocitemx{sec-madoko-presentation}{\mdref{sec-madoko-presentation}{2.\hspace*{0.5em}Madoko Presentation}}%mdk

\begin{mdtocblock}%mdk

\mdtocitemx{sec-revealjs}{\mdref{sec-revealjs}{2.1.\hspace*{0.5em}Reveal.js}}%mdk

\mdtocitemx{sec-code}{\mdref{sec-code}{2.2.\hspace*{0.5em}Code}}%mdk

\mdtocitemx{vertical}{\mdref{vertical}{2.3.\hspace*{0.5em}Vertical Slides}}%mdk

\mdtocitemx{sec-basement-level-1}{\mdref{sec-basement-level-1}{2.4.\hspace*{0.5em}Basement Level 1}}%mdk

\mdtocitemx{sec-basement-level-2}{\mdref{sec-basement-level-2}{2.5.\hspace*{0.5em}Basement Level 2}}%mdk

\mdtocitemx{sec-basement-level-3}{\mdref{sec-basement-level-3}{2.6.\hspace*{0.5em}Basement Level 3}}%mdk

\mdtocitemx{themes}{\mdref{themes}{2.7.\hspace*{0.5em}Themes}}%mdk

\mdtocitemx{transitions}{\mdref{transitions}{2.8.\hspace*{0.5em}Transitions}}%mdk

\mdtocitemx{sec-pauses}{\mdref{sec-pauses}{2.9.\hspace*{0.5em}Pauses?}}%mdk

\mdtocitemx{sec-point-of-view}{\mdref{sec-point-of-view}{2.10.\hspace*{0.5em}Point of View}}%mdk

\mdtocitemx{sec-works-in-mobile-safari}{\mdref{sec-works-in-mobile-safari}{2.11.\hspace*{0.5em}Works in Mobile Safari}}%mdk

\mdtocitemx{sec-printing}{\mdref{sec-printing}{2.12.\hspace*{0.5em}Printing}}%mdk

\mdtocitemx{sec-revealjs-library}{\mdref{sec-revealjs-library}{2.13.\hspace*{0.5em}RevealJS library}}%mdk

\mdtocitemx{sec-thanks-for-looking---}{\mdref{sec-thanks-for-looking---}{2.14.\hspace*{0.5em}Thanks for looking :-)}}%mdk
%mdk
\end{mdtocblock}%mdk
%mdk
\end{mdtocblock}%mdk
%mdk
\end{mdtoc}%mdk
%mdk
\end{mdframe}\label{sec-content}%mdk%mdk

%mdk-data-line={16}
\begin{mdframe}%mdk

\frametitle{Madoko Presentation}\label{heading-sec-madoko-presentation}%mdk%mdk

%mdk-data-line={18}
\noindent\mdline{18}Using\mdline{18}~\href{http://madoko.codeplex.com}{Madoko}\mdline{18} it is easy to create beautiful presentations.%mdk

%mdk-data-line={20}
\begin{itemize}[noitemsep,topsep=\mdcompacttopsep]%mdk

%mdk-data-line={20}
\item\mdline{20}\href{http://research.microsoft.com/en-us/um/people/daan/madoko/samples/slidedemo/out/slidedemo.html}{Html}\mdline{20}:
Uses the \mdline{21}\mdcode{Reveal.js}\mdline{21} library by\mdline{21}~\href{http://hakim.se}{Hakim El Hattab}\mdline{21}.\mdline{21}\mdbr
\mdline{22}This slide demo in Madoko is an adaption of his online demo.%mdk

%mdk-data-line={23}
\item\mdline{23}\href{http://research.microsoft.com/en-us/um/people/daan/madoko/samples/slidedemo/out/slidedemo.pdf}{Pdf}\mdline{23}:
Uses the \mdline{24}\mdcode{Beamer}\mdline{24} package for LaTeX%mdk

%mdk-data-line={25}
\item\mdline{25}\href{https://www.madoko.net/editor.html?\%23url=http://research.microsoft.com/en-us/um/people/daan/madoko/samples/slidedemo/slidedemo.mdk\%26options=\%7B"delayedUpdate":"true"\%7D}{Source}\mdline{25}:
Click to see the source of this presentation.%mdk
%mdk
\end{itemize}%mdk
%mdk
\end{mdframe}\label{sec-madoko-presentation}%mdk%mdk

%mdk-data-line={31}
\begin{mdframe}%mdk

\frametitle{Reveal.js}\label{heading-sec-revealjs}%mdk%mdk

%mdk-data-line={33}
\noindent\mdline{33}\mdcode{reveal.js}\mdline{33} is a framework for easily creating beautiful presentations using
HTML. You\mdline{34}'\mdline{34}ll need a browser with support for CSS 3D transforms to see it in
its full glory.%mdk

%mdk-data-line={37}
\mdline{37}And any Madoko features just work. Here is some math:%mdk

%mdk-data-line={39}
\begin{mdbmargintb}{1ex}{}%mdk
\begin{mdblock}{padding=1ex,border-width=\dimpx{1},border-color=black,border-style=solid}%mdk
%mdk-data-line={40}
\noindent\mdline{40}A famous equation is \mdline{40}$e = mc^2$\mdline{40}, but this one is 
famous too:%mdk
\label{eq-gaussian}%mdk
\noindent\mdline{43}\mdmathtag{(1)}\mdline{43}
\noindent\[%mdk-data-line={44}
\int_{-\infty}^\infty e^{-a x^2} d x = \sqrt{\frac{\pi}{a}} 
\]%mdk%mdk
\end{mdblock}%mdk
\end{mdbmargintb}%mdk
%mdk
\end{mdframe}\label{sec-revealjs}%mdk%mdk

%mdk-data-line={53}
\begin{mdframe}%mdk

\frametitle{Code}\label{heading-sec-code}%mdk%mdk

%mdk-data-line={55}
\noindent\mdline{55}Here is code, highlighted by Madoko%mdk
\begin{mdpre}%mdk
\noindent{\mdcolor{navy}function}~sqr(~x~)~\{\\
~~{\mdcolor{navy}var}~\ensuremath{\pi}~=~{\mdcolor{purple}3.141593};\\
~~{\mdcolor{navy}return}~x*x;~~{\mdcolor{darkgreen}/*}{\mdcolor{darkgreen}~the~square~}{\mdcolor{darkgreen}*/}\\
\}%mdk
\end{mdpre}\noindent\mdline{63}We used \mdline{63}\mdcode{\textbackslash{}(}\mdline{63} and \mdline{63}\mdcode{\textbackslash{})}\mdline{63} to escape into markdown to write \mdline{63}\ensuremath{\pi}\mdline{63}.

%mdk
\end{mdframe}\label{sec-code}%mdk%mdk

%mdk-data-line={68}
\begin{mdframe}%mdk

\frametitle{Vertical Slides}\label{heading-vertical}%mdk%mdk

%mdk-data-line={69}
\noindent\mdline{69}Slides can be nested inside of other slides,
try pressing\mdline{70}~\mdref{}{down}\mdline{70}.%mdk

%mdk-data-line={72}
\mdline{72}\mdref{}{\includegraphics[keepaspectratio=true,width=\dimpx{178},height=\dimpx{238}]{images/arrow}{}}\mdline{72}%mdk
%mdk
\end{mdframe}\label{vertical}%mdk%mdk

%mdk-data-line={78}
\begin{mdframe}%mdk

\frametitle{Basement Level 1}\label{heading-sec-basement-level-1}%mdk%mdk

%mdk-data-line={79}
\noindent\mdline{79}Press down or up to navigate.%mdk
%mdk
\end{mdframe}\label{sec-basement-level-1}%mdk%mdk

%mdk-data-line={81}
\begin{mdframe}%mdk

\frametitle{Basement Level 2}\label{heading-sec-basement-level-2}%mdk%mdk

%mdk-data-line={83}
\noindent\mdline{83}Use \mdline{83}\mdcode{columns}\mdline{83} to put blocks next to each other:%mdk
\begin{mdtabular}{2}{\dimeval{(\linewidth-\dimwidth{0.50})/1}}{0pt}%mdk
\begin{tabular}{ll}

%mdk-data-line={86}
\begin{mdcolumn}%mdk
\begin{mdblock}{width=\dimwidth{0.50}}%mdk
%mdk-data-line={87}
\noindent\mdline{87}A monarch butterfly (shown to the right)
spends the winter in Mexico.%mdk%mdk
\end{mdblock}%mdk
\end{mdcolumn}%mdk
&
%mdk-data-line={90}
\begin{mdcolumn}%mdk
\begin{mdblock}{width=\dimavailable}%mdk
%mdk-data-line={91}
\noindent\mdline{91}\mdinline{vertical-align=middle}{\includegraphics[keepaspectratio=true,width=\dimpx{280}]{images/butterfly}{}}%mdk%mdk
\end{mdblock}%mdk
\end{mdcolumn}%mdk
\\
\end{tabular}\end{mdtabular}


%mdk-data-line={105}
\noindent\mdline{105}\mdline{105}%mdk
%mdk
\end{mdframe}\label{sec-basement-level-2}%mdk%mdk

%mdk-data-line={108}
\begin{mdframe}%mdk

\frametitle{Basement Level 3}\label{heading-sec-basement-level-3}%mdk%mdk

%mdk-data-line={109}
\noindent\mdline{109}That\mdline{109}'\mdline{109}s it, time to go back up.%mdk

%mdk-data-line={111}
\mdline{111}\mdref{vertical}{\includegraphics[keepaspectratio=true,width=\dimpx{178},height=\dimpx{238}]{images/arrow}{}}\mdline{111}%mdk
%mdk
\end{mdframe}\label{sec-basement-level-3}%mdk%mdk

%mdk-data-line={119}
\begin{mdframe}%mdk

\frametitle{Themes}\label{heading-themes}%mdk%mdk

%mdk-data-line={121}
\noindent\mdline{121}Reveal.js comes with a few themes built in:%mdk

%mdk-data-line={123}
\begin{itemize}[noitemsep,topsep=\mdcompacttopsep]%mdk

%mdk-data-line={123}
\item\mdline{123}\href{?\%23/themes}{Default}\mdline{123}%mdk

%mdk-data-line={124}
\item\mdline{124}\href{?theme=sky\%23/themes}{Sky}\mdline{124}%mdk

%mdk-data-line={125}
\item\mdline{125}\href{?theme=beige\%23/themes}{Beige}\mdline{125}%mdk

%mdk-data-line={126}
\item\mdline{126}\href{?theme=serif\%23/themes}{Serif}\mdline{126}%mdk

%mdk-data-line={127}
\item\mdline{127}\href{?theme=simple\%23/themes}{Simple}\mdline{127}%mdk

%mdk-data-line={128}
\item\mdline{128}\href{?theme=night\%23/themes}{Night}\mdline{128}%mdk

%mdk-data-line={129}
\item\mdline{129}\href{?theme=moon\%23/themes}{Moon}\mdline{129}%mdk

%mdk-data-line={130}
\item\mdline{130}\href{?theme=solarized\%23/themes}{Solarized}\mdline{130}%mdk
%mdk
\end{itemize}%mdk

%mdk-data-line={132}
\noindent\mdline{132}Theme demos are loaded after the presentation which leads to flicker. In
production you should load your theme in the \mdline{133}\mdcode{\textless{}head\textgreater{}}\mdline{133} using a
\mdline{134}\mdcode{\textless{}link\textgreater{}}\mdline{134}.%mdk
%mdk
\end{mdframe}\label{themes}%mdk%mdk

%mdk-data-line={136}
\begin{mdframe}%mdk

\frametitle{Transitions}\label{heading-transitions}%mdk%mdk

%mdk-data-line={138}
\noindent\mdline{138}You can select from different transitions, like:\mdline{138}\mdbr
\mdline{139}\href{?transition=cube\%23/transitions}{Cube}\mdline{139} \mdline{139}-
\mdline{140}\href{?transition=page\%23/transitions}{Page}\mdline{140} \mdline{140}-
\mdline{141}\href{?transition=concave\%23/transitions}{Concave}\mdline{141} \mdline{141}-
\mdline{142}\href{?transition=zoom\%23/transitions}{Zoom}\mdline{142} \mdline{142}-
\mdline{143}\href{?transition=linear\%23/transitions}{Linear}\mdline{143} \mdline{143}-
\mdline{144}\href{?transition=fade\%23/transitions}{Fade}\mdline{144} \mdline{144}-
\mdline{145}\href{?transition=none\%23/transitions}{None}\mdline{145} \mdline{145}-
\mdline{146}\href{?\%23/transitions}{Default}\mdline{146}%mdk
%mdk
\end{mdframe}\label{transitions}%mdk%mdk

%mdk-data-line={150}
\begin{mdframe}%mdk

\frametitle{Pauses?}\label{heading-sec-pauses}%mdk%mdk

%mdk-data-line={152}
\noindent\mdline{152}Some pauses.%mdk

%mdk-data-line={154}
\begin{itemize}[noitemsep,topsep=\mdcompacttopsep]%mdk

%mdk-data-line={154}
\begin{onlyenv}<+->%mdk
\item\mdline{154}One
%mdk
\end{onlyenv}%mdk

%mdk-data-line={155}
\begin{onlyenv}<+->%mdk
\item\mdline{155}Two
%mdk
\end{onlyenv}%mdk

%mdk-data-line={156}
\begin{onlyenv}<+->%mdk
\item\mdline{156}Three
%mdk
\end{onlyenv}%mdk
%mdk
\end{itemize}%mdk

%mdk-data-line={158}
\noindent\mdline{158}And more:%mdk

%mdk-data-line={160}
\begin{itemize}[noitemsep,topsep=\mdcompacttopsep]%mdk
\begin{mdfragmented}%mdk

%mdk-data-line={160}
\item\mdline{160}Test 1%mdk

%mdk-data-line={161}
\item\mdline{161}Test 2%mdk

%mdk-data-line={162}
\item\mdline{162}Test 3%mdk
%mdk
\end{mdfragmented}%mdk
\end{itemize}%mdk

%mdk-data-line={165}
\noindent\mdline{165}Cool!.%mdk
%mdk
\end{mdframe}\label{sec-pauses}%mdk%mdk

%mdk-data-line={167}
\begin{mdframe}%mdk

%mdk-data-line={168}
\noindent\mdline{168}A slide with no header%mdk
%mdk
\end{mdframe}%mdk

%mdk-data-line={172}
\begin{mdframe}%mdk

\frametitle{Point of View}\label{heading-sec-point-of-view}%mdk%mdk

%mdk-data-line={174}
\noindent\mdline{174}In Reveal.js Press \mdline{174}\textbf{ESC}\mdline{174} to enter the slide overview.%mdk

%mdk-data-line={176}
\mdline{176}Hold down alt and click on any element to zoom in on it using 
\mdline{177}~\href{http://lab.hakim.se/zoom-js}{zoom.js}\mdline{177}. Alt\mdline{177} \mdline{177}+ click anywhere to zoom back out.%mdk
%mdk
\end{mdframe}\label{sec-point-of-view}%mdk%mdk

%mdk-data-line={183}
\begin{mdframe}%mdk

\frametitle{Works in Mobile Safari}\label{heading-sec-works-in-mobile-safari}%mdk%mdk

%mdk-data-line={185}
\noindent\mdline{185}Try it out! You can swipe through the slides and pinch your way to the
overview.%mdk
%mdk
\end{mdframe}\label{sec-works-in-mobile-safari}%mdk%mdk

%mdk-data-line={189}
\begin{mdframe}%mdk

\frametitle{Printing}\label{heading-sec-printing}%mdk%mdk

%mdk-data-line={191}
\noindent\mdline{191}You can print a \mdline{191}\mdcode{revealjs}\mdline{191} presentation nicely from the browser.%mdk

%mdk-data-line={193}
\mdline{193}First give the \mdline{193}\mdcode{?print-pdf}\mdline{193} or \mdline{193}\mdcode{?print-paper}\mdline{193} query on your final
presentation (in the browser address bar) and then print from the Chrome
browser selecting \mdline{195}\mdcode{Save~to~PDF}\mdline{195}.%mdk

%mdk-data-line={197}
\mdline{197}Read more about it at the\mdline{197}~\href{https://github.com/hakimel/reveal.js\%23pdf-export}{revealjs documentation}\mdline{197}%mdk
%mdk
\end{mdframe}\label{sec-printing}%mdk%mdk

%mdk-data-line={199}
\begin{mdframe}%mdk

\frametitle{RevealJS library}\label{heading-sec-revealjs-library}%mdk%mdk

%mdk-data-line={201}
\noindent\mdline{201}Normally, the \mdline{201}\mdcode{revealjs}\mdline{201} library is loaded from a CDN but you can specify your
own url using metadata:%mdk
\begin{mdpre}%mdk
\noindent Reveal~Url:~\textless{}my~url\textgreater{}%mdk
\end{mdpre}\noindent\mdline{206}This can be useful when using a\mdline{206}~\href{https://github.com/hakimel/reveal.js\%23full-setup}{server to run revealjs}\mdline{206}
where you may use something like:
\begin{mdpre}%mdk
\noindent@nopreview~Reveal~Url:~http://localhost:8000/reveal.js%mdk
\end{mdpre}%mdk
\end{mdframe}\label{sec-revealjs-library}%mdk%mdk

%mdk-data-line={212}
\begin{mdframe}%mdk

\frametitle{Thanks for looking :-)}\label{heading-sec-thanks-for-looking---}%mdk%mdk
%mdk
\end{mdframe}\label{sec-thanks-for-looking---}%mdk%mdk%mdk%mdk%mdk


\end{document}
