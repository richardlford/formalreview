\documentclass[12pt,twoside]{article}
% generated by Madoko, version 1.1.6
%mdk-data-line={1}


\usepackage[heading-base={2},section-num={False},bib-label={hide},fontspec={True}]{madoko2}
\usepackage[left=1.0in, right=0.75in]{geometry}


\begin{document}



%mdk-data-line={81}
\newcommand{\F}{\mathcal{F}}
\newcommand{\Equal}{\;\;\;=\;\;\;}
\newcommand{\Equiv}{\;\;\equiv\;\;}
\newcommand{\ES}{\;\;}
\newcommand{\ite}[3]{\textrm{if}\ES #1 \ES\textrm{then}\ES #2 \ES\textrm{else}\ES #3}
\newcommand{\Less}{\ll}
\newcommand{\Imp}{\;\Longrightarrow\;}
\renewcommand{\And}{\;\wedge\;}
\newcommand{\Or}{\;\vee\;}
\newcommand{\FBelow}{\ES\dot{\Rightarrow}\ES}
\newcommand{\least}{^{\downarrow}}
\newcommand{\greatest}{^{\uparrow}}
\newcommand{\Dfrac}[2]{%
  \ooalign{%
    $\genfrac{}{}{1.8pt}0{#1}{#2}$\cr%
    $\color{white}\genfrac{}{}{1pt}0{\phantom{#1}}{\phantom{#2}}$}%
}
\newcommand{\DfracA}[2]{%
  \ooalign{%
    $\genfrac{}{}{1.2pt}0{#1}{#2}$\cr%
    $\color{white}\genfrac{}{}{0.6pt}0{\phantom{#1}}{\phantom{#2}}$}%
}
\newcommand{\false}{\mathit{false}}
\newcommand{\true}{\mathit{true}}
\newcommand{\fib}{\mathit{fib}}
\newcommand{\PDownward}{\mathit{PDownward}}
\newcommand{\iterX}[1]{\hat{#1}}
\newcommand{\IterX}[1]{\check{#1}}
\renewcommand\vec[1]{\overrightarrow{#1}}
\newcommand\cev[1]{\overleftarrow{#1}}
\newcommand{\iterY}[1]{\vec{#1}}
\newcommand{\IterY}[1]{\cev{#1}}
\newcommand{\iter}[1]{{{}^{\flat}\!#1}}
\newcommand{\Iter}[1]{{{}^{\sharp}\!#1}}

% daan's fractions
\newcommand\xstrut{\vrule height 9.4pt depth 4.6pt width 0pt\relax}
\newcommand\xupstrut{\vrule height 9.4pt depth 0pt width 0pt\relax}
         
\renewcommand{\Dfrac}[2]{%
  \ooalign{%
    % first a thick fraction line
    $\genfrac{}{}{1.4pt}1{\displaystyle #1\strut}{\displaystyle #2\strut}$\cr%
    % and then a thinner white fraction line on top of it 
    $\color{white}\genfrac{}{}{0.6pt}1{\phantom{\displaystyle #1\strut}}{\phantom{\displaystyle #2\strut}}$}%
}
\renewcommand{\dfrac}[2]{%
   \displaystyle\genfrac{}{}{0.4pt}1{\displaystyle #1}{\displaystyle #2}%
}
%mdk-data-line={134}
\mdxtitleblockstart{}
%mdk-data-line={134}
\mdxtitle{\mdline{134}Formal Methods Review}%mdk

%mdk-data-line={137}
\mdxtitlenote{\mdline{137}2020-01-02 23:54}%mdk
\mdxauthorstart{}
%mdk-data-line={142}
\mdxauthorname{\mdline{142}Richard L. Ford}%mdk

%mdk-data-line={145}
\mdxauthoremail{\mdline{145}richardlford@gmail.com}%mdk
\mdxauthorend\mdtitleauthorrunning{}{}\mdxtitleblockend%mdk

%mdk-data-line={136}
\begin{abstract}%mdk

%mdk-data-line={137}
\noindent\mdline{137}This is a review of current formal methods theory, tools, projects and people.%mdk
%mdk
\end{abstract}%mdk
\mdline{140}
\begin{mdtoc}%mdk

\section*{Contents}\label{sec-contents}%mdk%mdk

\begin{mdtocblock}%mdk

\mdtocitemx{sec-introduction}{\mdref{sec-introduction}{1.\hspace*{0.5em}Introduction}}%mdk

\mdtocitemx{sec-formal-methods-research-groups}{\mdref{sec-formal-methods-research-groups}{2.\hspace*{0.5em}Formal Methods Research Groups}}%mdk

\begin{mdtocblock}%mdk

\mdtocitemx{sec-deepspec-project}{\mdref{sec-deepspec-project}{2.1.\hspace*{0.5em}DeepSpec Project}}%mdk

\mdtocitemx{sec-galois-inc}{\mdref{sec-galois-inc}{2.2.\hspace*{0.5em}Galois Inc}}%mdk

\mdtocitemx{sec-yale-flint-group}{\mdref{sec-yale-flint-group}{2.3.\hspace*{0.5em}Yale FLINT Group}}%mdk

\mdtocitemx{sec-everest-expedition}{\mdref{sec-everest-expedition}{2.4.\hspace*{0.5em}Everest Expedition}}%mdk

\mdtocitemx{sec-university-of-washington-programming-languages-and-software-engineering}{\mdref{sec-university-of-washington-programming-languages-and-software-engineering}{2.5.\hspace*{0.5em}University of Washington Programming Languages and Software Engineering}}%mdk
%mdk
\end{mdtocblock}%mdk

\mdtocitemx{sec-formal-methods-proof-systems}{\mdref{sec-formal-methods-proof-systems}{3.\hspace*{0.5em}Formal Methods Proof Systems}}%mdk

\begin{mdtocblock}%mdk

\mdtocitemx{sec-coq}{\mdref{sec-coq}{3.1.\hspace*{0.5em}Coq}}%mdk

\mdtocitemx{sec-keymaerax}{\mdref{sec-keymaerax}{3.2.\hspace*{0.5em}KeYmaeraX}}%mdk

\mdtocitemx{sec-isabelle}{\mdref{sec-isabelle}{3.3.\hspace*{0.5em}Isabelle}}%mdk

\mdtocitemx{sec-hol4}{\mdref{sec-hol4}{3.4.\hspace*{0.5em}HOL4}}%mdk

\mdtocitemx{sec-hol-light}{\mdref{sec-hol-light}{3.5.\hspace*{0.5em}HOL-Light}}%mdk

\mdtocitemx{sec-dafny}{\mdref{sec-dafny}{3.6.\hspace*{0.5em}Dafny}}%mdk

\mdtocitemx{sec-boogie}{\mdref{sec-boogie}{3.7.\hspace*{0.5em}Boogie}}%mdk

\mdtocitemx{sec-why3}{\mdref{sec-why3}{3.8.\hspace*{0.5em}Why3}}%mdk

\mdtocitemx{sec-frama-c}{\mdref{sec-frama-c}{3.9.\hspace*{0.5em}Frama-C}}%mdk

\mdtocitemx{sec-f-}{\mdref{sec-f-}{3.10.\hspace*{0.5em}F*}}%mdk

\mdtocitemx{sec-lean}{\mdref{sec-lean}{3.11.\hspace*{0.5em}LEAN}}%mdk
%mdk
\end{mdtocblock}%mdk

\mdtocitemx{sec-formal-methods-projects}{\mdref{sec-formal-methods-projects}{4.\hspace*{0.5em}Formal Methods Projects}}%mdk

\begin{mdtocblock}%mdk

\mdtocitemx{sec-deepspec-projects}{\mdref{sec-deepspec-projects}{4.1.\hspace*{0.5em}DeepSpec Projects}}%mdk

\begin{mdtocblock}%mdk

\mdtocitemx{sec-compcert}{\mdref{sec-compcert}{4.1.1.\hspace*{0.5em}CompCert}}%mdk

\mdtocitemx{sec-verified-software-toolchain}{\mdref{sec-verified-software-toolchain}{4.1.2.\hspace*{0.5em}Verified Software Toolchain}}%mdk

\mdtocitemx{sec-certikos---certified-kit-operating-system}{\mdref{sec-certikos---certified-kit-operating-system}{4.1.3.\hspace*{0.5em}CertiKOS - Certified Kit Operating System}}%mdk

\mdtocitemx{sec-veriml}{\mdref{sec-veriml}{4.1.4.\hspace*{0.5em}VeriML}}%mdk

\mdtocitemx{sec-certifying-low-level-programs-with-hardware-interrupts-and-preemptive-threads}{\mdref{sec-certifying-low-level-programs-with-hardware-interrupts-and-preemptive-threads}{4.1.5.\hspace*{0.5em}Certifying Low-Level Programs with Hardware Interrupts and Preemptive Threads}}%mdk

\mdtocitemx{sec-kami}{\mdref{sec-kami}{4.1.6.\hspace*{0.5em}Kami}}%mdk

\mdtocitemx{sec-haskell-core-spec}{\mdref{sec-haskell-core-spec}{4.1.7.\hspace*{0.5em}Haskell Core Spec}}%mdk

\mdtocitemx{sec-deep-spec-server}{\mdref{sec-deep-spec-server}{4.1.8.\hspace*{0.5em}Deep Spec Server}}%mdk

\begin{mdtocblock}%mdk

\mdtocitemx{sec-gnu-libmicrohttpd}{\mdref{sec-gnu-libmicrohttpd}{4.1.8.1.\hspace*{0.5em}GNU Libmicrohttpd}}%mdk
%mdk
\end{mdtocblock}%mdk

\mdtocitemx{sec-verdi}{\mdref{sec-verdi}{4.1.9.\hspace*{0.5em}Verdi}}%mdk

\mdtocitemx{sec-vellvm}{\mdref{sec-vellvm}{4.1.10.\hspace*{0.5em}Vellvm}}%mdk

\begin{mdtocblock}%mdk

\mdtocitemx{sec-latest-results}{\mdref{sec-latest-results}{4.1.10.1.\hspace*{0.5em}Latest Results}}%mdk
%mdk
\end{mdtocblock}%mdk

\mdtocitemx{sec-deep-spec-crypto}{\mdref{sec-deep-spec-crypto}{4.1.11.\hspace*{0.5em}Deep Spec Crypto}}%mdk

\mdtocitemx{sec-deepspecdb}{\mdref{sec-deepspecdb}{4.1.12.\hspace*{0.5em}DeepSpecDb}}%mdk

\mdtocitemx{sec-fiat}{\mdref{sec-fiat}{4.1.13.\hspace*{0.5em}Fiat}}%mdk

\mdtocitemx{sec-narcissus}{\mdref{sec-narcissus}{4.1.14.\hspace*{0.5em}Narcissus}}%mdk

\mdtocitemx{sec-bedrock2}{\mdref{sec-bedrock2}{4.1.15.\hspace*{0.5em}Bedrock2}}%mdk

\mdtocitemx{sec-certicoq}{\mdref{sec-certicoq}{4.1.16.\hspace*{0.5em}CertiCoq}}%mdk

\mdtocitemx{sec-template-coq}{\mdref{sec-template-coq}{4.1.17.\hspace*{0.5em}Template-Coq}}%mdk

\mdtocitemx{sec-quickchick}{\mdref{sec-quickchick}{4.1.18.\hspace*{0.5em}QuickChick}}%mdk

\mdtocitemx{sec-galois-voting-system}{\mdref{sec-galois-voting-system}{4.1.19.\hspace*{0.5em}Galois Voting System}}%mdk
%mdk
\end{mdtocblock}%mdk

\mdtocitemx{sec-everest-projects}{\mdref{sec-everest-projects}{4.2.\hspace*{0.5em}Everest Projects}}%mdk

\begin{mdtocblock}%mdk

\mdtocitemx{sec-everest}{\mdref{sec-everest}{4.2.1.\hspace*{0.5em}Everest}}%mdk

\mdtocitemx{sec-quackyducky}{\mdref{sec-quackyducky}{4.2.2.\hspace*{0.5em}Quackyducky}}%mdk
%mdk
\end{mdtocblock}%mdk

\mdtocitemx{sec-other-projects}{\mdref{sec-other-projects}{4.3.\hspace*{0.5em}Other Projects}}%mdk

\begin{mdtocblock}%mdk

\mdtocitemx{sec-cakeml}{\mdref{sec-cakeml}{4.3.1.\hspace*{0.5em}CakeML}}%mdk

\mdtocitemx{sec-vcc---a-verifier-for-concurrent-c}{\mdref{sec-vcc---a-verifier-for-concurrent-c}{4.3.2.\hspace*{0.5em}VCC - A verifier for Concurrent C}}%mdk

\mdtocitemx{sec-compositional-compcert}{\mdref{sec-compositional-compcert}{4.3.3.\hspace*{0.5em}Compositional CompCert}}%mdk

\mdtocitemx{sec-galoisinc-projects}{\mdref{sec-galoisinc-projects}{4.3.4.\hspace*{0.5em}GaloisInc Projects}}%mdk

\mdtocitemx{sec-bedrock}{\mdref{sec-bedrock}{4.3.5.\hspace*{0.5em}Bedrock}}%mdk

\mdtocitemx{sec-fscq}{\mdref{sec-fscq}{4.3.6.\hspace*{0.5em}FSCQ}}%mdk

\mdtocitemx{sec-urweb}{\mdref{sec-urweb}{4.3.7.\hspace*{0.5em}Ur/Web}}%mdk
%mdk
\end{mdtocblock}%mdk
%mdk
\end{mdtocblock}%mdk

\mdtocitemx{sec-formal-methods-researchers}{\mdref{sec-formal-methods-researchers}{5.\hspace*{0.5em}Formal Methods Researchers}}%mdk

\begin{mdtocblock}%mdk

\mdtocitemx{sec-andrew-w-appel}{\mdref{sec-andrew-w-appel}{5.1.\hspace*{0.5em}Andrew W Appel}}%mdk

\mdtocitemx{sec-adam-chlipala}{\mdref{sec-adam-chlipala}{5.2.\hspace*{0.5em}Adam Chlipala}}%mdk

\mdtocitemx{sec-robert-harper}{\mdref{sec-robert-harper}{5.3.\hspace*{0.5em}Robert Harper}}%mdk

\mdtocitemx{sec-benjamin-pierce}{\mdref{sec-benjamin-pierce}{5.4.\hspace*{0.5em}Benjamin Pierce}}%mdk

\mdtocitemx{sec-zhong-shao}{\mdref{sec-zhong-shao}{5.5.\hspace*{0.5em}Zhong Shao}}%mdk
%mdk
\end{mdtocblock}%mdk

\mdtocitemx{sec-statistics}{\mdref{sec-statistics}{6.\hspace*{0.5em}Statistics}}%mdk

\mdtocitemx{sec-bibliography}{\mdref{sec-bibliography}{References}}%mdk
%mdk
\end{mdtocblock}%mdk
%mdk
\end{mdtoc}%mdk

%mdk-data-line={143}
\section{\mdline{143}1.\hspace*{0.5em}\mdline{143}Introduction}\label{sec-introduction}%mdk%mdk

%mdk-data-line={145}
\noindent\mdline{145}This document briefly summarizes current formal method groups, proof
systems, projects and researchers. For each it gives pointers for
further information. It is a work-in-progress and is necessarily
incomplete.%mdk

%mdk-data-line={150}
\section{\mdline{150}2.\hspace*{0.5em}\mdline{150}Formal Methods Research Groups}\label{sec-formal-methods-research-groups}%mdk%mdk

%mdk-data-line={151}
\subsection{\mdline{151}2.1.\hspace*{0.5em}\mdline{151}DeepSpec Project}\label{sec-deepspec-project}%mdk%mdk

%mdk-data-line={152}
\begin{mddefinitions}%mdk

\mddefterm{\noindent{\bfseries Home Page}}%mdk

%mdk-data-line={152}
\begin{mdbmarginx}{}{}{}{1.5em}%mdk
\begin{mddefdata}%mdk
\mdline{152}\href{https://deepspec.org}{{\ttfamily https://\hspace{0pt}deepspec.\hspace{0pt}org}}%mdk
\end{mddefdata}%mdk
\end{mdbmarginx}%mdk
%mdk
\end{mddefinitions}%mdk

%mdk-data-line={155}
\subsection{\mdline{155}2.2.\hspace*{0.5em}\mdline{155}Galois Inc}\label{sec-galois-inc}%mdk%mdk

%mdk-data-line={156}
\begin{mddefinitions}%mdk

\mddefterm{\noindent{\bfseries Home Page }}%mdk

%mdk-data-line={156}
\begin{mdbmarginx}{}{}{}{1.5em}%mdk
\begin{mddefdata}%mdk
\mdline{156}\href{https://galois.com/}{{\ttfamily https://\hspace{0pt}galois.\hspace{0pt}com/\hspace{0pt}}}\mdline{156}
%mdk
\end{mddefdata}%mdk
\end{mdbmarginx}%mdk

\mddefterm{\noindent{\bfseries Sources }}%mdk

%mdk-data-line={158}
\begin{mdbmarginx}{}{}{}{1.5em}%mdk
\begin{mddefdata}%mdk
\mdline{158}\href{https://github.com/GaloisInc}{{\ttfamily https://\hspace{0pt}github.\hspace{0pt}com/\hspace{0pt}GaloisInc}}%mdk
\end{mddefdata}%mdk
\end{mdbmarginx}%mdk
%mdk
\end{mddefinitions}%mdk

%mdk-data-line={161}
\subsection{\mdline{161}2.3.\hspace*{0.5em}\mdline{161}Yale FLINT Group}\label{sec-yale-flint-group}%mdk%mdk

%mdk-data-line={162}
\begin{itemize}[noitemsep,topsep=\mdcompacttopsep]%mdk

%mdk-data-line={162}
\item\mdline{162}Home Page :: \mdline{162}\href{http://flint.cs.yale.edu/flint/}{{\ttfamily http://\hspace{0pt}flint.\hspace{0pt}cs.\hspace{0pt}yale.\hspace{0pt}edu/\hspace{0pt}flint/\hspace{0pt}}}\mdline{162}%mdk

%mdk-data-line={163}
\item\mdline{163}Publications:

%mdk-data-line={164}
\begin{itemize}[noitemsep,topsep=\mdcompacttopsep]%mdk

%mdk-data-line={164}
\item\mdline{164}\href{http://flint.cs.yale.edu/flint}{{\ttfamily http://\hspace{0pt}flint.\hspace{0pt}cs.\hspace{0pt}yale.\hspace{0pt}edu/\hspace{0pt}flint}}\mdline{164}%mdk
%mdk
\end{itemize}%mdk%mdk
%mdk
\end{itemize}%mdk

%mdk-data-line={166}
\noindent\mdline{166}"\mdline{166}The FLINT group at Yale aims to develop a novel and practical
programming infrastructure for constructing large-scale certified
systems software. By combining recent new advances in programming
languages, formal semantics, certified operating systems, program
verification, proof assistants and automation, language-based
secrurity, and certifying compilers, we hope to attack the following
research questions:%mdk

%mdk-data-line={174}
\begin{itemize}[noitemsep,topsep=\mdcompacttopsep]%mdk

%mdk-data-line={174}
\item\mdline{174}what system software structures can offer the best support for extensibility, security, and resilience?%mdk

%mdk-data-line={175}
\item\mdline{175}what program logics and semantic models can best capture these abstractions?%mdk

%mdk-data-line={176}
\item\mdline{176}what are the right programming languages and environments for developing such certified system software?%mdk

%mdk-data-line={177}
\item\mdline{177}how to build new automation facilities to make certified software really scale?\mdline{177}"\mdline{177}%mdk
%mdk
\end{itemize}%mdk

%mdk-data-line={179}
\subsection{\mdline{179}2.4.\hspace*{0.5em}\mdline{179}Everest Expedition}\label{sec-everest-expedition}%mdk%mdk

%mdk-data-line={180}
\begin{itemize}[noitemsep,topsep=\mdcompacttopsep]%mdk

%mdk-data-line={180}
\item\mdline{180}Home Page :: \mdline{180}\href{https://project-everest.github.io/}{{\ttfamily https://\hspace{0pt}project-\hspace{0pt}everest.\hspace{0pt}github.\hspace{0pt}io/\hspace{0pt}}}\mdline{180}%mdk

%mdk-data-line={181}
\item\mdline{181}Source :: \mdline{181}\href{https://github.com/project-everest}{{\ttfamily https://\hspace{0pt}github.\hspace{0pt}com/\hspace{0pt}project-\hspace{0pt}everest}}\mdline{181}%mdk
%mdk
\end{itemize}%mdk

%mdk-data-line={183}
\noindent\mdline{183}"\mdline{183}We are a team of researchers and engineers from several
organizations, including Microsoft Research, Carnegie Mellon
University, INRIA, and the MSR-INRIA joint center.%mdk

%mdk-data-line={187}
\mdline{187}Everest is a recursive acronym: It stands for the “Everest VERified
End-to-end Secure Transport”.\mdline{188}"\mdline{188}%mdk

%mdk-data-line={190}
\subsection{\mdline{190}2.5.\hspace*{0.5em}\mdline{190}University of Washington Programming Languages and Software Engineering}\label{sec-university-of-washington-programming-languages-and-software-engineering}%mdk%mdk

%mdk-data-line={191}
\begin{itemize}[noitemsep,topsep=\mdcompacttopsep]%mdk

%mdk-data-line={191}
\item\mdline{191}Home Page :: \mdline{191}\href{http://uwplse.org/}{{\ttfamily http://\hspace{0pt}uwplse.\hspace{0pt}org/\hspace{0pt}}}\mdline{191}%mdk

%mdk-data-line={192}
\item\mdline{192}Source :: \mdline{192}\href{https://github.com/uwplse}{{\ttfamily https://\hspace{0pt}github.\hspace{0pt}com/\hspace{0pt}uwplse}}\mdline{192}%mdk
%mdk
\end{itemize}%mdk

%mdk-data-line={194}
\section{\mdline{194}3.\hspace*{0.5em}\mdline{194}Formal Methods Proof Systems}\label{sec-formal-methods-proof-systems}%mdk%mdk

%mdk-data-line={195}
\subsection{\mdline{195}3.1.\hspace*{0.5em}\mdline{195}Coq}\label{sec-coq}%mdk%mdk

%mdk-data-line={196}
\begin{itemize}[noitemsep,topsep=\mdcompacttopsep]%mdk

%mdk-data-line={196}
\item\mdline{196}Home Page :: \mdline{196}\href{https://coq.inria.fr/}{{\ttfamily https://\hspace{0pt}coq.\hspace{0pt}inria.\hspace{0pt}fr/\hspace{0pt}}}\mdline{196}%mdk

%mdk-data-line={197}
\item\mdline{197}Source :: \mdline{197}\href{https://github.com/coq/coq.git}{{\ttfamily https://\hspace{0pt}github.\hspace{0pt}com/\hspace{0pt}coq/\hspace{0pt}coq.\hspace{0pt}git}}\mdline{197}%mdk
%mdk
\end{itemize}%mdk

%mdk-data-line={199}
\noindent\mdline{199}Coq is a formal proof management system. It provides a formal language
to write mathematical definitions, executable algorithms and theorems
together with an environment for semi-interactive development of
machine-checked proofs. Typical applications include the certification
of properties of programming languages (e.g. the CompCert compiler
certification project, or the Bedrock verified low-level programming
library), the formalization of mathematics (e.g. the full
formalization of the Feit-Thompson theorem or homotopy type theory)
and teaching.%mdk

%mdk-data-line={209}
\subsection{\mdline{209}3.2.\hspace*{0.5em}\mdline{209}KeYmaeraX}\label{sec-keymaerax}%mdk%mdk

%mdk-data-line={210}
\begin{itemize}[noitemsep,topsep=\mdcompacttopsep]%mdk

%mdk-data-line={210}
\item\mdline{210}Home Page :: \mdline{210}\href{http://www.ls.cs.cmu.edu/KeYmaeraX/}{{\ttfamily http://\hspace{0pt}www.\hspace{0pt}ls.\hspace{0pt}cs.\hspace{0pt}cmu.\hspace{0pt}edu/\hspace{0pt}KeYmaeraX/\hspace{0pt}}}\mdline{210}%mdk

%mdk-data-line={211}
\item\mdline{211}Source :: \mdline{211}\href{https://github.com/LS-Lab/KeYmaeraX-release}{{\ttfamily https://\hspace{0pt}github.\hspace{0pt}com/\hspace{0pt}LS-\hspace{0pt}Lab/\hspace{0pt}KeYmaeraX-\hspace{0pt}release}}\mdline{211}%mdk
%mdk
\end{itemize}%mdk

%mdk-data-line={213}
\noindent\mdline{213}KeYmaeraX is a system for specifying and verifying Hybrid
Cyber-Physical Systems.%mdk

%mdk-data-line={216}
\subsection{\mdline{216}3.3.\hspace*{0.5em}\mdline{216}Isabelle}\label{sec-isabelle}%mdk%mdk

%mdk-data-line={217}
\begin{itemize}[noitemsep,topsep=\mdcompacttopsep]%mdk

%mdk-data-line={217}
\item\mdline{217}Home Page :: \mdline{217}\href{https://isabelle.in.tum.de/}{{\ttfamily https://\hspace{0pt}isabelle.\hspace{0pt}in.\hspace{0pt}tum.\hspace{0pt}de/\hspace{0pt}}}\mdline{217}%mdk

%mdk-data-line={218}
\item\mdline{218}Source :: hg clone \mdline{218}\href{https://isabelle.in.tum.de/repos/isabelle}{{\ttfamily https://\hspace{0pt}isabelle.\hspace{0pt}in.\hspace{0pt}tum.\hspace{0pt}de/\hspace{0pt}repos/\hspace{0pt}isabelle}}\mdline{218}%mdk
%mdk
\end{itemize}%mdk

%mdk-data-line={220}
\noindent\mdline{220}\textquotedblleft{}Isabelle is a generic proof assistant. It allows mathematical
formulas to be expressed in a formal language and provides tools for
proving those formulas in a logical calculus. The main application is
the formalization of mathematical proofs and in particular formal
verification, which includes proving the correctness of computer
hardware or software and proving properties of computer languages and
protocols.\textquotedblright{}\mdline{226}%mdk

%mdk-data-line={228}
\subsection{\mdline{228}3.4.\hspace*{0.5em}\mdline{228}HOL4}\label{sec-hol4}%mdk%mdk

%mdk-data-line={229}
\begin{itemize}[noitemsep,topsep=\mdcompacttopsep]%mdk

%mdk-data-line={229}
\item\mdline{229}Home Page :: \mdline{229}\href{https://hol-theorem-prover.org/}{{\ttfamily https://\hspace{0pt}hol-\hspace{0pt}theorem-\hspace{0pt}prover.\hspace{0pt}org/\hspace{0pt}}}\mdline{229}%mdk

%mdk-data-line={230}
\item\mdline{230}Source :: \mdline{230}\href{https://github.com/HOL-Theorem-Prover}{{\ttfamily https://\hspace{0pt}github.\hspace{0pt}com/\hspace{0pt}HOL-\hspace{0pt}Theorem-\hspace{0pt}Prover}}\mdline{230}%mdk
%mdk
\end{itemize}%mdk

%mdk-data-line={232}
\noindent\mdline{232}\textquotedblleft{}The HOL interactive theorem prover is a proof assistant for
higher-order logic: a programming environment in which theorems can be
proved and proof tools implemented. Built-in decision procedures and
theorem provers can automatically establish many simple theorems
(users may have to prove the hard theorems themselves!) An oracle
mechanism gives access to external programs such as SMT and BDD
engines. HOL is particularly suitable as a platform for implementing
combinations of deduction, execution and property checking.\textquotedblright{}\mdline{239}%mdk

%mdk-data-line={241}
\subsection{\mdline{241}3.5.\hspace*{0.5em}\mdline{241}HOL-Light}\label{sec-hol-light}%mdk%mdk

%mdk-data-line={242}
\begin{itemize}[noitemsep,topsep=\mdcompacttopsep]%mdk

%mdk-data-line={242}
\item\mdline{242}Home Page :: \mdline{242}\href{https://www.cl.cam.ac.uk/~jrh13/hol-light/}{{\ttfamily https://\hspace{0pt}www.\hspace{0pt}cl.\hspace{0pt}cam.\hspace{0pt}ac.\hspace{0pt}uk/\hspace{0pt}\textasciitilde{}jrh13/\hspace{0pt}hol-\hspace{0pt}light/\hspace{0pt}}}\mdline{242}%mdk

%mdk-data-line={243}
\item\mdline{243}Source :: \mdline{243}\href{https://github.com/jrh13/hol-light/}{{\ttfamily https://\hspace{0pt}github.\hspace{0pt}com/\hspace{0pt}jrh13/\hspace{0pt}hol-\hspace{0pt}light/\hspace{0pt}}}\mdline{243}%mdk
%mdk
\end{itemize}%mdk

%mdk-data-line={245}
\noindent\mdline{245}\textquotedblleft{}HOL Light is a computer program to help users prove interesting
mathematical theorems completely formally in higher order logic. It
sets a very exacting standard of correctness, but provides a number of
automated tools and pre-proved mathematical theorems (e.g. about
arithmetic, basic set theory and real analysis) to save the user
work. It is also fully programmable, so users can extend it with new
theorems and inference rules without compromising its soundness.\textquotedblright{}\mdline{251}%mdk

%mdk-data-line={253}
\subsection{\mdline{253}3.6.\hspace*{0.5em}\mdline{253}Dafny}\label{sec-dafny}%mdk%mdk

%mdk-data-line={254}
\begin{itemize}[noitemsep,topsep=\mdcompacttopsep]%mdk

%mdk-data-line={254}
\item\mdline{254}Home Page ::\mdline{254}~\href{https://www.microsoft.com/en-us/research/project/dafny-a-language-and-program-verifier-for-functional-correctness}{Dafny}\mdline{254}%mdk

%mdk-data-line={255}
\item\mdline{255}Source :: \mdline{255}\href{https://github.com/Microsoft/dafny}{{\ttfamily https://\hspace{0pt}github.\hspace{0pt}com/\hspace{0pt}Microsoft/\hspace{0pt}dafny}}\mdline{255}%mdk

%mdk-data-line={256}
\item\mdline{256}Tutorial :: \mdline{256}\href{https://rise4fun.com/dafny}{{\ttfamily https://\hspace{0pt}rise4fun.\hspace{0pt}com/\hspace{0pt}dafny}}\mdline{256}%mdk
%mdk
\end{itemize}%mdk

%mdk-data-line={258}
\subsection{\mdline{258}3.7.\hspace*{0.5em}\mdline{258}Boogie}\label{sec-boogie}%mdk%mdk

%mdk-data-line={259}
\begin{itemize}[noitemsep,topsep=\mdcompacttopsep]%mdk

%mdk-data-line={259}
\item\mdline{259}Home Page ::\mdline{259}~\href{https://www.microsoft.com/en-us/research/project/boogie-an-intermediate-verification-language}{Boogie}\mdline{259}%mdk

%mdk-data-line={260}
\item\mdline{260}Source :: \mdline{260}\href{https://github.com/boogie-org/boogie}{{\ttfamily https://\hspace{0pt}github.\hspace{0pt}com/\hspace{0pt}boogie-\hspace{0pt}org/\hspace{0pt}boogie}}\mdline{260}%mdk

%mdk-data-line={261}
\item\mdline{261}Documentation :: \mdline{261}\href{https://boogie-docs.readthedocs.io/en/latest/}{{\ttfamily https://\hspace{0pt}boogie-\hspace{0pt}docs.\hspace{0pt}readthedocs.\hspace{0pt}io/\hspace{0pt}en/\hspace{0pt}latest/\hspace{0pt}}}\mdline{261}%mdk

%mdk-data-line={262}
\item\mdline{262}Online trial :: \mdline{262}\href{https://rise4fun.com/Boogie}{{\ttfamily https://\hspace{0pt}rise4fun.\hspace{0pt}com/\hspace{0pt}Boogie}}\mdline{262}%mdk
%mdk
\end{itemize}%mdk

%mdk-data-line={264}
\subsection{\mdline{264}3.8.\hspace*{0.5em}\mdline{264}Why3}\label{sec-why3}%mdk%mdk

%mdk-data-line={265}
\begin{itemize}[noitemsep,topsep=\mdcompacttopsep]%mdk

%mdk-data-line={265}
\item\mdline{265}Home Page :: \mdline{265}\href{http://why3.lri.fr/}{{\ttfamily http://\hspace{0pt}why3.\hspace{0pt}lri.\hspace{0pt}fr/\hspace{0pt}}}\mdline{265}%mdk
%mdk
\end{itemize}%mdk

%mdk-data-line={267}
\subsection{\mdline{267}3.9.\hspace*{0.5em}\mdline{267}Frama-C}\label{sec-frama-c}%mdk%mdk

%mdk-data-line={268}
\begin{itemize}[noitemsep,topsep=\mdcompacttopsep]%mdk

%mdk-data-line={268}
\item\mdline{268}Home Page :: \mdline{268}\href{https://frama-c.com/}{{\ttfamily https://\hspace{0pt}frama-\hspace{0pt}c.\hspace{0pt}com/\hspace{0pt}}}\mdline{268}%mdk
%mdk
\end{itemize}%mdk

%mdk-data-line={270}
\subsection{\mdline{270}3.10.\hspace*{0.5em}\mdline{270}F\mdline{270}*}\label{sec-f-}%mdk%mdk

%mdk-data-line={271}
\begin{itemize}%mdk

%mdk-data-line={271}
\item{}
%mdk-data-line={271}
\mdline{271}Home Page :: \mdline{271}\href{https://www.fstar-lang.org/}{{\ttfamily https://\hspace{0pt}www.\hspace{0pt}fstar-\hspace{0pt}lang.\hspace{0pt}org/\hspace{0pt}}}\mdline{271}%mdk%mdk

%mdk-data-line={272}
\item{}
%mdk-data-line={272}
\mdline{272}Source :: \mdline{272}\href{http://github.com/FStarLang/FStar}{{\ttfamily http://\hspace{0pt}github.\hspace{0pt}com/\hspace{0pt}FStarLang/\hspace{0pt}FStar}}\mdline{272}%mdk%mdk

%mdk-data-line={274}
\item{}
%mdk-data-line={274}
\mdline{274}Papers:%mdk

%mdk-data-line={275}
\begin{itemize}[noitemsep,topsep=\mdcompacttopsep]%mdk

%mdk-data-line={275}
\item\mdline{275}\href{https://www.fstar-lang.org/tutorial/}{F* Tutorial}\mdline{275}%mdk
%mdk
\end{itemize}%mdk%mdk
%mdk
\end{itemize}%mdk

%mdk-data-line={277}
\subsection{\mdline{277}3.11.\hspace*{0.5em}\mdline{277}LEAN}\label{sec-lean}%mdk%mdk

%mdk-data-line={280}
\begin{itemize}[noitemsep,topsep=\mdcompacttopsep]%mdk

%mdk-data-line={280}
\item\mdline{280}Papers:

%mdk-data-line={281}
\begin{itemize}[noitemsep,topsep=\mdcompacttopsep]%mdk

%mdk-data-line={281}
\item\mdline{281}\href{https://pp.ipd.kit.edu/uploads/publikationen/ebner17meta.pdf}{A Metaprogramming Framework for Formal Verification}\mdline{281}%mdk
%mdk
\end{itemize}%mdk%mdk
%mdk
\end{itemize}%mdk

%mdk-data-line={283}
\section{\mdline{283}4.\hspace*{0.5em}\mdline{283}Formal Methods Projects}\label{sec-formal-methods-projects}%mdk%mdk

%mdk-data-line={285}
\subsection{\mdline{285}4.1.\hspace*{0.5em}\mdline{285}DeepSpec Projects}\label{sec-deepspec-projects}%mdk%mdk

%mdk-data-line={287}
\subsubsection{\mdline{287}4.1.1.\hspace*{0.5em}\mdline{287}CompCert}\label{sec-compcert}%mdk%mdk

%mdk-data-line={288}
\begin{itemize}[noitemsep,topsep=\mdcompacttopsep]%mdk

%mdk-data-line={288}
\item\mdline{288}Home Page :: \mdline{288}\href{http://compcert.inria.fr}{{\ttfamily http://\hspace{0pt}compcert.\hspace{0pt}inria.\hspace{0pt}fr}}\mdline{288}%mdk

%mdk-data-line={289}
\item\mdline{289}Source :: \mdline{289}\href{https://github.com/AbsInt/CompCert.git}{{\ttfamily https://\hspace{0pt}github.\hspace{0pt}com/\hspace{0pt}AbsInt/\hspace{0pt}CompCert.\hspace{0pt}git}}\mdline{289}%mdk
%mdk
\end{itemize}%mdk

%mdk-data-line={291}
\noindent\mdline{291}The CompCert C verified compiler is a compiler for a large subset of
the C programming language that generates code for the PowerPC, ARM,
x86 and RISC-V processors.%mdk

%mdk-data-line={295}
\mdline{295}The distinguishing feature of CompCert is that it has been formally
verified using the Coq proof assistant: the generated assembly code is
formally guaranteed to behave as prescribed by the semantics of the
source C code.%mdk

%mdk-data-line={300}
\subsubsection{\mdline{300}4.1.2.\hspace*{0.5em}\mdline{300}Verified Software Toolchain}\label{sec-verified-software-toolchain}%mdk%mdk

%mdk-data-line={301}
\begin{itemize}[noitemsep,topsep=\mdcompacttopsep]%mdk

%mdk-data-line={301}
\item\mdline{301}Home Page :: \mdline{301}\href{http://vst.cs.princeton.edu/}{{\ttfamily http://\hspace{0pt}vst.\hspace{0pt}cs.\hspace{0pt}princeton.\hspace{0pt}edu/\hspace{0pt}}}\mdline{301}%mdk

%mdk-data-line={302}
\item\mdline{302}Source :: \mdline{302}\href{https://github.com/PrincetonUniversity/VST.git}{{\ttfamily https://\hspace{0pt}github.\hspace{0pt}com/\hspace{0pt}PrincetonUniversity/\hspace{0pt}VST.\hspace{0pt}git}}\mdline{302}%mdk
%mdk
\end{itemize}%mdk

%mdk-data-line={304}
\noindent\mdline{304}The software toolchain includes static analyzers to check assertions
about your program; optimizing compilers to translate your program to
machine language; operating systems and libraries to supply context
for your program. The Verified Software Toolchain project assures with
machine-checked proofs that the assertions claimed at the top of the
toolchain really hold in the machine-language program, running in the
operating-system context.%mdk

%mdk-data-line={312}
\mdline{312}In some application domains it is not enough to build reliable
software systems, one wants proved-correct software. This is the case
for safety-critical systems (where software bugs can cause injury or
death) and for security-critical applications (where an attacker is
deliberately searching for, and exploiting, software bugs). Since
proofs are large and complex, the proof-checking must be
mechanized. Machine-checked proofs of real software systems are
difficult, but now should be possible, given the recent advances in
the theory and engineering of mechanized proof systems applied to
software verification. But there are several challenges:%mdk

%mdk-data-line={323}
\begin{itemize}%mdk

%mdk-data-line={323}
\item{}
%mdk-data-line={323}
\mdline{323}Real software systems are usually built from components in different
programming languages.%mdk%mdk

%mdk-data-line={326}
\item{}
%mdk-data-line={326}
\mdline{326}Some parts of the program need full correctness proofs, which must
be constructed with great effort; other parts need only safety
proofs, which can be constructed automatically.%mdk%mdk

%mdk-data-line={330}
\item{}
%mdk-data-line={330}
\mdline{330}One reasons about correctness at the source-code level, but one runs
a machine-code program translated by a compiler; the compiler must
be proved correct.%mdk%mdk

%mdk-data-line={334}
\item{}
%mdk-data-line={334}
\mdline{334}These proofs about different properties, with respect to different
programming languages, must be integrated together end-to-end in a
way that is also proved correct and machine-checked.%mdk%mdk
%mdk
\end{itemize}%mdk

%mdk-data-line={338}
\noindent\mdline{338}We address these challenges by defining Verifiable C, a program logic
for the C programming language. Verifiable C is proved sound with
respect to the operational semantics of CompCert C; in turn, the
CompCert verified optimizing C compiler is proved correct with respect
to the assembly-language semantics of the PowerPC, ARM, and x86
processors.%mdk

%mdk-data-line={345}
\subsubsection{\mdline{345}4.1.3.\hspace*{0.5em}\mdline{345}CertiKOS\mdline{345} \mdline{345}- Certified Kit Operating System}\label{sec-certikos---certified-kit-operating-system}%mdk%mdk

%mdk-data-line={346}
\begin{itemize}[noitemsep,topsep=\mdcompacttopsep]%mdk

%mdk-data-line={346}
\item\mdline{346}Home Page :: \mdline{346}\href{http://flint.cs.yale.edu/certikos/}{{\ttfamily http://\hspace{0pt}flint.\hspace{0pt}cs.\hspace{0pt}yale.\hspace{0pt}edu/\hspace{0pt}certikos/\hspace{0pt}}}\mdline{346}%mdk

%mdk-data-line={347}
\item\mdline{347}Source :: \mdline{347}\href{https://github.com/npe9/certikos.git}{{\ttfamily https://\hspace{0pt}github.\hspace{0pt}com/\hspace{0pt}npe9/\hspace{0pt}certikos.\hspace{0pt}git}}\mdline{347}%mdk
%mdk
\end{itemize}%mdk

%mdk-data-line={349}
\noindent\mdline{349}Developed by the FLINT group.%mdk

%mdk-data-line={351}
\subsubsection{\mdline{351}4.1.4.\hspace*{0.5em}\mdline{351}VeriML}\label{sec-veriml}%mdk%mdk

%mdk-data-line={352}
\begin{itemize}[noitemsep,topsep=\mdcompacttopsep]%mdk

%mdk-data-line={352}
\item\mdline{352}Home Page :: \mdline{352}\href{http://flint.cs.yale.edu/shao/papers/veriml.html}{{\ttfamily http://\hspace{0pt}flint.\hspace{0pt}cs.\hspace{0pt}yale.\hspace{0pt}edu/\hspace{0pt}shao/\hspace{0pt}papers/\hspace{0pt}veriml.\hspace{0pt}html}}\mdline{352}%mdk

%mdk-data-line={353}
\item\mdline{353}Source :: \mdline{353}\href{http://flint.cs.yale.edu/flint/publications/veriml-0.1.tar.gz}{{\ttfamily http://\hspace{0pt}flint.\hspace{0pt}cs.\hspace{0pt}yale.\hspace{0pt}edu/\hspace{0pt}flint/\hspace{0pt}publications/\hspace{0pt}veriml-\hspace{0pt}0.\hspace{0pt}1.\hspace{0pt}tar.\hspace{0pt}gz}}\mdline{353}%mdk

%mdk-data-line={354}
\item\mdline{354}Paper :: \mdline{354}\href{http://flint.cs.yale.edu/flint/publications/verimltr.pdf}{{\ttfamily http://\hspace{0pt}flint.\hspace{0pt}cs.\hspace{0pt}yale.\hspace{0pt}edu/\hspace{0pt}flint/\hspace{0pt}publications/\hspace{0pt}verimltr.\hspace{0pt}pdf}}\mdline{354}%mdk
%mdk
\end{itemize}%mdk

%mdk-data-line={356}
\noindent\mdline{356}Developed by the FLINT group.%mdk

%mdk-data-line={358}
\mdline{358}Modern proof assistants such as Coq and Isabelle provide high degrees
of expressiveness and assurance because they support formal reasoning
in higher-order logic and supply explicit machine-checkable proof
objects. Unfortunately, large scale proof development in these proof
assistants is still an extremely difficult and time-consuming
task. One major weakness of these proof assistants is the lack of a
single language where users can develop complex tactics and decision
procedures using a rich programming model and in a typeful
manner. This limits the scalability of the proof development process,
as users avoid developing domain-specific tactics and decision
procedures.
¯˘
In this paper, we present VeriML\mdline{370}\textemdash{}\mdline{370}a novel language design that
couples a type-safe effectful computational language with first-class
support for manipulating logical terms such as propositions and
proofs. The main idea behind our design is to integrate a rich logical
framework\mdline{374}\textemdash{}\mdline{374}similar to the one supported by Coq\mdline{374}\textemdash{}\mdline{374}inside a
computational language inspired by ML. The language design is such
that the added features are orthogonal to the rest of the
computational language, and also do not require significant additions
to the logic language, so soundness is guaranteed. We have built a
prototype implementation of VeriML including both its type-checker and
an interpreter. We demonstrate the effectiveness of our design by
showing a number of type-safe tactics and decision procedures written
in VeriML.%mdk

%mdk-data-line={384}
\subsubsection{\mdline{384}4.1.5.\hspace*{0.5em}\mdline{384}Certifying Low-Level Programs with Hardware Interrupts and Preemptive Threads}\label{sec-certifying-low-level-programs-with-hardware-interrupts-and-preemptive-threads}%mdk%mdk

%mdk-data-line={385}
\begin{itemize}[noitemsep,topsep=\mdcompacttopsep]%mdk

%mdk-data-line={385}
\item\mdline{385}Home Page :: \mdline{385}\href{http://flint.cs.yale.edu/shao/papers/aimjar.html}{{\ttfamily http://\hspace{0pt}flint.\hspace{0pt}cs.\hspace{0pt}yale.\hspace{0pt}edu/\hspace{0pt}shao/\hspace{0pt}papers/\hspace{0pt}aimjar.\hspace{0pt}html}}\mdline{385}%mdk

%mdk-data-line={386}
\item\mdline{386}Source :: \mdline{386}\href{http://flint.cs.yale.edu/flint/publications/aim.coq.tar.gz}{{\ttfamily http://\hspace{0pt}flint.\hspace{0pt}cs.\hspace{0pt}yale.\hspace{0pt}edu/\hspace{0pt}flint/\hspace{0pt}publications/\hspace{0pt}aim.\hspace{0pt}coq.\hspace{0pt}tar.\hspace{0pt}gz}}\mdline{386}%mdk

%mdk-data-line={387}
\item\mdline{387}Local repo :: e/certhwint%mdk
%mdk
\end{itemize}%mdk

%mdk-data-line={389}
\noindent\mdline{389}Developed by the FLINT group.%mdk

%mdk-data-line={391}
\subsubsection{\mdline{391}4.1.6.\hspace*{0.5em}\mdline{391}Kami}\label{sec-kami}%mdk%mdk

%mdk-data-line={392}
\begin{itemize}[noitemsep,topsep=\mdcompacttopsep]%mdk

%mdk-data-line={392}
\item\mdline{392}Home Page :: \mdline{392}\href{http://plv.csail.mit.edu/kami/}{{\ttfamily http://\hspace{0pt}plv.\hspace{0pt}csail.\hspace{0pt}mit.\hspace{0pt}edu/\hspace{0pt}kami/\hspace{0pt}}}\mdline{392}%mdk

%mdk-data-line={393}
\item\mdline{393}Source :: \mdline{393}\href{https://github.com/mit-plv/kami}{{\ttfamily https://\hspace{0pt}github.\hspace{0pt}com/\hspace{0pt}mit-\hspace{0pt}plv/\hspace{0pt}kami}}\mdline{393}%mdk
%mdk
\end{itemize}%mdk

%mdk-data-line={395}
\noindent\mdline{395}Kami is a library that turns Coq into an IDE for digital hardware
development, based on a clean-slate reimplementation of a core of the
\mdline{397}\href{http://www.bluespec.com}{Bluespec}\mdline{397} language. We span the gap from mathematical specifications to
hardware circuit descriptions (RTL netlists). We support specifying,
implementing, verifying, and compiling hardware, reasoning at a high
level about particular hardware components but in the end deriving
first-principles Coq theorems about circuits. No part of Kami need be
trusted beside the formalization of low-level (Verilog-style) circuit
descriptions; all other aspects have end-to-end correctness proofs
checked by Coq. Hardware designs are broken into separately verified
modules, reasoned about with a novel take on labeled transition
systems. Furthermore, Coq provides a natural and expressive platform
for metaprogramming, or building verified circuit generators, as for a
memory caching system autogenerated for a particular shape of cache
hierarchy, or a CPU generated given a number of concurrent cores as
input.%mdk

%mdk-data-line={412}
\subsubsection{\mdline{412}4.1.7.\hspace*{0.5em}\mdline{412}Haskell Core Spec}\label{sec-haskell-core-spec}%mdk%mdk

%mdk-data-line={413}
\begin{itemize}[noitemsep,topsep=\mdcompacttopsep]%mdk

%mdk-data-line={413}
\item\mdline{413}Home Page :: \mdline{413}\href{https://deepspec.org/entry/Project/Haskell+CoreSpec}{{\ttfamily https://\hspace{0pt}deepspec.\hspace{0pt}org/\hspace{0pt}entry/\hspace{0pt}Project/\hspace{0pt}Haskell+\hspace{0pt}CoreSpec}}\mdline{413}%mdk

%mdk-data-line={414}
\item\mdline{414}Source :: \mdline{414}\href{https://github.com/sweirich/corespec.git}{{\ttfamily https://\hspace{0pt}github.\hspace{0pt}com/\hspace{0pt}sweirich/\hspace{0pt}corespec.\hspace{0pt}git}}\mdline{414}%mdk
%mdk
\end{itemize}%mdk

%mdk-data-line={416}
\noindent\mdline{416}The Haskell CoreSpec Project aims to develop formal specifications for
a high-level, industrially-relevant functional programming
language. In particular, this project targets the core language of the
Glasgow Haskell Compiler, the primary compiler for the Haskell
programming language. GHC has long been used as both an industrial
strength compiler and a platform for language research. The compiler
itself is open source, and has primarily been developed and is
currently maintained by researchers at Microsoft Research,
Cambridge. The CoreSpec project will develop a formal Coq
specification of the GHC Core language, including the syntax, type
system, and semantics, and connect that specification to other
components of the DeepSpec project.%mdk

%mdk-data-line={429}
\subsubsection{\mdline{429}4.1.8.\hspace*{0.5em}\mdline{429}Deep Spec Server}\label{sec-deep-spec-server}%mdk%mdk

%mdk-data-line={430}
\begin{itemize}[noitemsep,topsep=\mdcompacttopsep]%mdk

%mdk-data-line={430}
\item\mdline{430}Home Page :: \mdline{430}\href{https://deepspec.org/entry/Project/DeepSpec+Web+Server}{{\ttfamily https://\hspace{0pt}deepspec.\hspace{0pt}org/\hspace{0pt}entry/\hspace{0pt}Project/\hspace{0pt}DeepSpec+\hspace{0pt}Web+\hspace{0pt}Server}}\mdline{430}%mdk

%mdk-data-line={431}
\item\mdline{431}Source :: Not available but see Libmicrohttpd below.%mdk
%mdk
\end{itemize}%mdk

%mdk-data-line={434}
\noindent\mdline{434}\textquotedblleft{}For a final demo, unifying many of the Expedition threads, we aim to
build a verified web server.\textquotedblright{}\mdline{435}%mdk

%mdk-data-line={437}
\mdline{437}Status: \mdline{437}\textquotedblleft{}A team at Penn has begun exploring the design space and
building a first-draft prototype (for now, running on Linux) of a web
server library loosely based on the popular libmicrohttpd.  The goal
of this short-term effort is to understand the integration issues that
will be involved in putting together a fully functional server from
components under development within DeepSpec.  In particular, we want
to understand what demands it will place on CertiKOS in terms of OS
features (IPC, network support, shared-memory processes,
interoperation between native clients and Linux VMs), what
verification challenges it raises for VST, what integration challenges
it poses for using VST and CertiKOS together.\textquotedblright{}\mdline{447}%mdk

%mdk-data-line={449}
\paragraph{\mdline{449}4.1.8.1.\hspace*{0.5em}\mdline{449}GNU Libmicrohttpd}\label{sec-gnu-libmicrohttpd}%mdk%mdk

%mdk-data-line={450}
\begin{itemize}[noitemsep,topsep=\mdcompacttopsep]%mdk

%mdk-data-line={450}
\item\mdline{450}Home Page :: \mdline{450}\href{https://www.gnu.org/software/libmicrohttpd/}{{\ttfamily https://\hspace{0pt}www.\hspace{0pt}gnu.\hspace{0pt}org/\hspace{0pt}software/\hspace{0pt}libmicrohttpd/\hspace{0pt}}}\mdline{450}%mdk

%mdk-data-line={451}
\item\mdline{451}Source :: \mdline{451}\href{https://gnunet.org/git/libmicrohttpd.git}{{\ttfamily https://\hspace{0pt}gnunet.\hspace{0pt}org/\hspace{0pt}git/\hspace{0pt}libmicrohttpd.\hspace{0pt}git}}\mdline{451}%mdk
%mdk
\end{itemize}%mdk

%mdk-data-line={453}
\subsubsection{\mdline{453}4.1.9.\hspace*{0.5em}\mdline{453}Verdi}\label{sec-verdi}%mdk%mdk

%mdk-data-line={454}
\begin{itemize}[noitemsep,topsep=\mdcompacttopsep]%mdk

%mdk-data-line={454}
\item\mdline{454}Home Page :: \mdline{454}\href{http://verdi.uwplse.org/}{{\ttfamily http://\hspace{0pt}verdi.\hspace{0pt}uwplse.\hspace{0pt}org/\hspace{0pt}}}\mdline{454}%mdk

%mdk-data-line={455}
\item\mdline{455}Source :: \mdline{455}\href{https://github.com/uwplse/verdi}{{\ttfamily https://\hspace{0pt}github.\hspace{0pt}com/\hspace{0pt}uwplse/\hspace{0pt}verdi}}\mdline{455}%mdk

%mdk-data-line={456}
\item\mdline{456}Example :: \mdline{456}\href{https://github.com/uwplse/verdi-raft}{{\ttfamily https://\hspace{0pt}github.\hspace{0pt}com/\hspace{0pt}uwplse/\hspace{0pt}verdi-\hspace{0pt}raft}}\mdline{456}%mdk
%mdk
\end{itemize}%mdk

%mdk-data-line={458}
\noindent\mdline{458}Verification of distributed systems.%mdk

%mdk-data-line={460}
\subsubsection{\mdline{460}4.1.10.\hspace*{0.5em}\mdline{460}Vellvm}\label{sec-vellvm}%mdk%mdk

%mdk-data-line={461}
\begin{itemize}[noitemsep,topsep=\mdcompacttopsep]%mdk

%mdk-data-line={461}
\item\mdline{461}Home Page :: \mdline{461}\href{http://www.cis.upenn.edu/~stevez/vellvm/}{{\ttfamily http://\hspace{0pt}www.\hspace{0pt}cis.\hspace{0pt}upenn.\hspace{0pt}edu/\hspace{0pt}\textasciitilde{}stevez/\hspace{0pt}vellvm/\hspace{0pt}}}\mdline{461}%mdk

%mdk-data-line={462}
\item\mdline{462}Source :: \mdline{462}\href{https://github.com/vellvm/vellvm}{{\ttfamily https://\hspace{0pt}github.\hspace{0pt}com/\hspace{0pt}vellvm/\hspace{0pt}vellvm}}\mdline{462}%mdk

%mdk-data-line={463}
\item\mdline{463}Old Source :: \mdline{463}\href{https://github.com/vellvm/vellvm-legacy}{{\ttfamily https://\hspace{0pt}github.\hspace{0pt}com/\hspace{0pt}vellvm/\hspace{0pt}vellvm-\hspace{0pt}legacy}}\mdline{463}%mdk
%mdk
\end{itemize}%mdk

%mdk-data-line={465}
\noindent\mdline{465}"\mdline{465}The Vellvm project is building a (verified LLVM), a framework for
reasoning about programs expressed in LLVM\mdline{466}'\mdline{466}s intermediate
representation and transformations that operate on it. Vellvm provides
a mechanized formal semantics of LLVM\mdline{468}'\mdline{468}s intermediate representation,
its type system, and properties of its SSA form. The framework is
built using the Coq interactive theorem prover. It includes multiple
operational semantics and proves relations among them to facilitate
different reasoning styles and proof techniques.%mdk

%mdk-data-line={474}
\paragraph{\mdline{474}4.1.10.1.\hspace*{0.5em}\mdline{474}Latest Results}\label{sec-latest-results}%mdk%mdk

%mdk-data-line={476}
\noindent\mdline{476}During the first year of DeepSpec we%mdk

%mdk-data-line={478}
\begin{enumerate}%mdk

%mdk-data-line={478}
\item{}
%mdk-data-line={478}
\mdline{478}worked on developing a new modular semantics for Vellvm, factoring out the memory model%mdk%mdk

%mdk-data-line={480}
\item{}
%mdk-data-line={480}
\mdline{480}made progress on connecting LLVM-IR like SSA semantics with higher-level structural operational semantics%mdk%mdk

%mdk-data-line={482}
\item{}
%mdk-data-line={482}
\mdline{482}applied low-level language verification techniques to the problem of race detectection instrumentation\mdline{482}"\mdline{482}%mdk%mdk
%mdk
\end{enumerate}%mdk

%mdk-data-line={484}
\subsubsection{\mdline{484}4.1.11.\hspace*{0.5em}\mdline{484}Deep Spec Crypto}\label{sec-deep-spec-crypto}%mdk%mdk

%mdk-data-line={485}
\begin{itemize}[noitemsep,topsep=\mdcompacttopsep]%mdk

%mdk-data-line={485}
\item\mdline{485}Home Page :: \mdline{485}\href{https://deepspec.org/entry/Project/Cryptography}{{\ttfamily https://\hspace{0pt}deepspec.\hspace{0pt}org/\hspace{0pt}entry/\hspace{0pt}Project/\hspace{0pt}Cryptography}}\mdline{485}%mdk

%mdk-data-line={486}
\item\mdline{486}Source :: \mdline{486}\href{https://github.com/mit-plv/fiat-crypto}{{\ttfamily https://\hspace{0pt}github.\hspace{0pt}com/\hspace{0pt}mit-\hspace{0pt}plv/\hspace{0pt}fiat-\hspace{0pt}crypto}}\mdline{486}%mdk

%mdk-data-line={487}
\item\mdline{487}Papers :: \mdline{487}\href{http://adam.chlipala.net/papers/FiatCryptoSP19/FiatCryptoSP19.pdf}{{\ttfamily http://\hspace{0pt}adam.\hspace{0pt}chlipala.\hspace{0pt}net/\hspace{0pt}papers/\hspace{0pt}FiatCryptoSP19/\hspace{0pt}FiatCryptoSP19.\hspace{0pt}pdf}}\mdline{487}
\begin{mdpre}%mdk
\begin{mdbmarginx}{}{}{}{1em}%mdk
\noindent{\small\preindent{6}\textless{}http://www.cs.princeton.edu/\textasciitilde{}appel/papers/verified-hmac-drbg.pdf\textgreater{}}%mdk
\end{mdbmarginx}%mdk
\end{mdpre}%mdk
%mdk
\end{itemize}%mdk

%mdk-data-line={490}
\noindent\mdline{490}\textquotedblleft{}We are pursuing end-to-end proofs of cryptographic functionality, via
verification of C code at Princeton and synthesis of assembly code at
MIT.  We are considering both cryptographic primitives
(e.g. pseudorandom number generation with VST at Princeton and
elliptic curve operations with fiat-crypto at MIT) and protocols
(verified with the Foundational Cryptography Framework and connected
to results about C and assembly programs).\textquotedblright{}\mdline{496}%mdk

%mdk-data-line={498}
\paragraph{\mdline{498}Latest Results:}\label{sec-latest-results-}%mdk%mdk

%mdk-data-line={500}
\noindent\mdline{500}Fiat Cryptography is now used in Google\mdline{500}'\mdline{500}s BoringSSL library for
elliptic-curve arithmetic.  As a result, Chrome HTTPS connections now
run our Coq-generated code.  Our  upcoming S\mdline{502}\&\mdline{502}P 2019 paper goes into
more detail.%mdk

%mdk-data-line={505}
\mdline{505}VST verification has recently focused on the primitives HMAC-DGBG and
HKDF\mdline{506} \mdline{506}\textendash{}\mdline{506} both clients of HMAC/SHA256, AES, and parts of the TweetNaCl
library. In addition to verifying (families of) primitives, we hope to
soon turn to integration in larger contexts like verified TLS
libraries.%mdk

%mdk-data-line={511}
\subsubsection{\mdline{511}4.1.12.\hspace*{0.5em}\mdline{511}DeepSpecDb}\label{sec-deepspecdb}%mdk%mdk

%mdk-data-line={512}
\begin{itemize}[noitemsep,topsep=\mdcompacttopsep]%mdk

%mdk-data-line={512}
\item\mdline{512}Home Page :: ?%mdk

%mdk-data-line={513}
\item\mdline{513}Source :: \mdline{513}\href{https://github.com/PrincetonUniversity/DeepSpecDB}{{\ttfamily https://\hspace{0pt}github.\hspace{0pt}com/\hspace{0pt}PrincetonUniversity/\hspace{0pt}DeepSpecDB}}\mdline{513}%mdk

%mdk-data-line={514}
\item\mdline{514}Papers:

%mdk-data-line={515}
\begin{itemize}[noitemsep,topsep=\mdcompacttopsep]%mdk

%mdk-data-line={515}
\item\mdline{515}\href{http://perso.eleves.ens-rennes.fr/people/Aurele.Barriere/papers/vstbtrees.pdf}{VST Verification of B+Trees with Cursors}\mdline{515}%mdk

%mdk-data-line={516}
\item\mdline{516}\href{https://github.com/PrincetonUniversity/DeepSpecDB/blob/master/papers/adewale/Masters_Thesis.pdf}{Implementing a high-performance key-value store using a trie of B+-Trees with cursors}\mdline{516}%mdk

%mdk-data-line={517}
\item\mdline{517}\href{https://github.com/PrincetonUniversity/DeepSpecDB/blob/master/papers/luke/report.pdf}{Project Report on DeepSpecDB}\mdline{517}%mdk

%mdk-data-line={518}
\item\mdline{518}\href{https://github.com/PrincetonUniversity/DeepSpecDB/blob/master/papers/mcswiggen/McSwiggen-Thesis.pdf}{The Theory and Verification of B+Tree Cursor Relations}\mdline{518}%mdk
%mdk
\end{itemize}%mdk%mdk
%mdk
\end{itemize}%mdk

%mdk-data-line={520}
\subsubsection{\mdline{520}4.1.13.\hspace*{0.5em}\mdline{520}Fiat}\label{sec-fiat}%mdk%mdk

%mdk-data-line={521}
\begin{itemize}[noitemsep,topsep=\mdcompacttopsep]%mdk

%mdk-data-line={521}
\item\mdline{521}Home Page :: \mdline{521}\href{http://plv.csail.mit.edu/fiat/}{{\ttfamily http://\hspace{0pt}plv.\hspace{0pt}csail.\hspace{0pt}mit.\hspace{0pt}edu/\hspace{0pt}fiat/\hspace{0pt}}}\mdline{521}%mdk

%mdk-data-line={522}
\item\mdline{522}Source :: \mdline{522}\href{https://github.com/mit-plv/fiat.git}{{\ttfamily https://\hspace{0pt}github.\hspace{0pt}com/\hspace{0pt}mit-\hspace{0pt}plv/\hspace{0pt}fiat.\hspace{0pt}git}}\mdline{522}%mdk

%mdk-data-line={523}
\item\mdline{523}Papers ::

%mdk-data-line={524}
\begin{itemize}[noitemsep,topsep=\mdcompacttopsep]%mdk

%mdk-data-line={524}
\item\mdline{524}The End of History? Using a Proof Assistant to
Replace Language Design with Library Design
\mdline{526}\href{https://www.cs.purdue.edu/homes/bendy/Fiat/fiat-snapl.pdf}{{\ttfamily https://\hspace{0pt}www.\hspace{0pt}cs.\hspace{0pt}purdue.\hspace{0pt}edu/\hspace{0pt}homes/\hspace{0pt}bendy/\hspace{0pt}Fiat/\hspace{0pt}fiat-\hspace{0pt}snapl.\hspace{0pt}pdf}}\mdline{526}%mdk
%mdk
\end{itemize}%mdk%mdk
%mdk
\end{itemize}%mdk

%mdk-data-line={528}
\noindent\mdline{528}\textquotedblleft{}Fiat is a library for the Coq proof assistant for synthesizing
efficient correct-by-construction programs from declarative
specifications. Programming by Fiat starts with a high-level
description of a program, which can be written using libraries of
specification languages for describing common programming tasks like
querying a relational database. These specifications are then
iteratively refined into efficient implementations via automated
tactics. Each derivation in Fiat produces a formal proof trail
certifying that the synthesized program meets the original
specification. Code synthesized by Fiat can be extracted to an
equivalent OCaml program that can be compiled and run as normal.\textquotedblright{}\mdline{538}%mdk

%mdk-data-line={540}
\subsubsection{\mdline{540}4.1.14.\hspace*{0.5em}\mdline{540}Narcissus}\label{sec-narcissus}%mdk%mdk

%mdk-data-line={541}
\begin{itemize}[noitemsep,topsep=\mdcompacttopsep]%mdk

%mdk-data-line={541}
\item\mdline{541}Home Page :: \mdline{541}\href{https://www.cs.purdue.edu/homes/bendy/Narcissus/}{{\ttfamily https://\hspace{0pt}www.\hspace{0pt}cs.\hspace{0pt}purdue.\hspace{0pt}edu/\hspace{0pt}homes/\hspace{0pt}bendy/\hspace{0pt}Narcissus/\hspace{0pt}}}\mdline{541}%mdk

%mdk-data-line={542}
\item\mdline{542}Source :: \mdline{542}\href{https://github.com/mit-plv/fiat/tree/master/src/Narcissus}{{\ttfamily https://\hspace{0pt}github.\hspace{0pt}com/\hspace{0pt}mit-\hspace{0pt}plv/\hspace{0pt}fiat/\hspace{0pt}tree/\hspace{0pt}master/\hspace{0pt}src/\hspace{0pt}Narcissus}}\mdline{542}
\begin{mdpre}%mdk
\begin{mdbmarginx}{}{}{}{1em}%mdk
\noindent{\small\preindent{6}\textless{}https://github.com/bendy/fiat-asn.1\textgreater{}}%mdk
\end{mdbmarginx}%mdk
\end{mdpre}%mdk

%mdk-data-line={544}
\item\mdline{544}Papers :: NARCISSUS: Deriving Correct-By-Construction Decoders and Encoders from Binary Formats
\begin{mdpre}%mdk
\begin{mdbmarginx}{}{}{}{1em}%mdk
\noindent{\small\preindent{6}\textless{}https://www.cs.purdue.edu/homes/bendy/Narcissus/narcissus.pdf\textgreater{}}%mdk
\end{mdbmarginx}%mdk
\end{mdpre}%mdk

%mdk-data-line={546}
\item\mdline{546}Video :: \mdline{546}\mdline{546}%mdk
%mdk
\end{itemize}%mdk

%mdk-data-line={548}
\noindent\mdline{548}Narcissus is part of the fiat project to derive
Correct-By-Construction Decoders and Encoders from Binary Formats.%mdk

%mdk-data-line={551}
\subsubsection{\mdline{551}4.1.15.\hspace*{0.5em}\mdline{551}Bedrock2}\label{sec-bedrock2}%mdk%mdk

%mdk-data-line={552}
\begin{itemize}[noitemsep,topsep=\mdcompacttopsep]%mdk

%mdk-data-line={552}
\item\mdline{552}Source ::%mdk
%mdk
\end{itemize}%mdk

%mdk-data-line={554}
\subsubsection{\mdline{554}4.1.16.\hspace*{0.5em}\mdline{554}CertiCoq}\label{sec-certicoq}%mdk%mdk

%mdk-data-line={555}
\begin{itemize}[noitemsep,topsep=\mdcompacttopsep]%mdk

%mdk-data-line={555}
\item\mdline{555}Home Page :: \mdline{555}\href{https://www.cs.princeton.edu/~appel/certicoq/}{{\ttfamily https://\hspace{0pt}www.\hspace{0pt}cs.\hspace{0pt}princeton.\hspace{0pt}edu/\hspace{0pt}\textasciitilde{}appel/\hspace{0pt}certicoq/\hspace{0pt}}}\mdline{555}%mdk

%mdk-data-line={556}
\item\mdline{556}Source :: \mdline{556}\href{https://github.com/PrincetonUniversity/certicoq}{{\ttfamily https://\hspace{0pt}github.\hspace{0pt}com/\hspace{0pt}PrincetonUniversity/\hspace{0pt}certicoq}}\mdline{556}%mdk

%mdk-data-line={557}
\item\mdline{557}Paper :: \mdline{557}\href{http://www.cs.princeton.edu/~appel/papers/certicoq-coqpl.pdf}{{\ttfamily http://\hspace{0pt}www.\hspace{0pt}cs.\hspace{0pt}princeton.\hspace{0pt}edu/\hspace{0pt}\textasciitilde{}appel/\hspace{0pt}papers/\hspace{0pt}certicoq-\hspace{0pt}coqpl.\hspace{0pt}pdf}}\mdline{557}%mdk
%mdk
\end{itemize}%mdk

%mdk-data-line={559}
\noindent\mdline{559}\textquotedblleft{}The CertiCoq project aims to build a proven-correct compiler for
dependently-typed, functional languages, such as Gallinathe core
language of the Coq proof assistant. A proved-correct compiler
consists of a high-level functional specification, machine-verified
proofs of important properties, such as safety and correctness, and a
mechanism to transport those proofs to the generated machine code. The
project exposes both engineering challenges and foundational questions
about compilers for dependently-typed languages.\textquotedblright{}\mdline{566}%mdk

%mdk-data-line={568}
\subsubsection{\mdline{568}4.1.17.\hspace*{0.5em}\mdline{568}Template-Coq}\label{sec-template-coq}%mdk%mdk

%mdk-data-line={569}
\begin{itemize}[noitemsep,topsep=\mdcompacttopsep]%mdk

%mdk-data-line={569}
\item\mdline{569}Home Page :: \mdline{569}\href{https://template-coq.github.io/template-coq/}{{\ttfamily https://\hspace{0pt}template-\hspace{0pt}coq.\hspace{0pt}github.\hspace{0pt}io/\hspace{0pt}template-\hspace{0pt}coq/\hspace{0pt}}}\mdline{569}%mdk

%mdk-data-line={570}
\item\mdline{570}Source :: \mdline{570}\href{https://github.com/Template-Coq/template-coq}{{\ttfamily https://\hspace{0pt}github.\hspace{0pt}com/\hspace{0pt}Template-\hspace{0pt}Coq/\hspace{0pt}template-\hspace{0pt}coq}}\mdline{570}%mdk

%mdk-data-line={571}
\item\mdline{571}Papeers :: \mdline{571}\href{https://popl18.sigplan.org/event/coqpl-2018-typed-template-coq}{{\ttfamily https://\hspace{0pt}popl18.\hspace{0pt}sigplan.\hspace{0pt}org/\hspace{0pt}event/\hspace{0pt}coqpl-\hspace{0pt}2018-\hspace{0pt}typed-\hspace{0pt}template-\hspace{0pt}coq}}\mdline{571}%mdk
%mdk
\end{itemize}%mdk

%mdk-data-line={573}
\noindent\mdline{573}Template Coq is a quoting library for\mdline{573}~\href{http://coq.inria.fr}{Coq}\mdline{573}. It
takes \mdline{574}\mdcode{\small Coq}\mdline{574} terms and constructs a representation of their syntax tree as
a \mdline{575}\mdcode{\small Coq}\mdline{575} inductive data type. The representation is based on the kernel\mdline{575}'\mdline{575}s
term representation.%mdk

%mdk-data-line={578}
\mdline{578}This is used as the first stage of CertiCoq.%mdk

%mdk-data-line={580}
\subsubsection{\mdline{580}4.1.18.\hspace*{0.5em}\mdline{580}QuickChick}\label{sec-quickchick}%mdk%mdk

%mdk-data-line={581}
\begin{itemize}%mdk

%mdk-data-line={581}
\item{}
%mdk-data-line={581}
\mdline{581}Home Page :: \mdline{581}\href{https://deepspec.org/entry/Project/QuickChick}{{\ttfamily https://\hspace{0pt}deepspec.\hspace{0pt}org/\hspace{0pt}entry/\hspace{0pt}Project/\hspace{0pt}QuickChick}}\mdline{581}%mdk%mdk

%mdk-data-line={582}
\item{}
%mdk-data-line={582}
\mdline{582}Source :: \mdline{582}\href{https://github.com/QuickChick}{{\ttfamily https://\hspace{0pt}github.\hspace{0pt}com/\hspace{0pt}QuickChick}}\mdline{582}%mdk%mdk

%mdk-data-line={583}
\item{}
%mdk-data-line={583}
\mdline{583}Book ::\mdline{583}~\href{https://softwarefoundations.cis.upenn.edu/qc-current}{QuickChick: Property-Based Testing in Coq}\mdline{583}%mdk%mdk

%mdk-data-line={585}
\item{}
%mdk-data-line={585}
\mdline{585}Randomized property-based testing plugin for Coq; a clone of Haskell QuickCheck%mdk%mdk

%mdk-data-line={586}
\item{}
%mdk-data-line={586}
\mdline{586}Includes a foundational verification framework for testing code%mdk%mdk

%mdk-data-line={587}
\item{}
%mdk-data-line={587}
\mdline{587}Includes a mechanism for automatically deriving generators for inductive relations%mdk%mdk
%mdk
\end{itemize}%mdk

%mdk-data-line={589}
\subsubsection{\mdline{589}4.1.19.\hspace*{0.5em}\mdline{589}Galois Voting System}\label{sec-galois-voting-system}%mdk%mdk

%mdk-data-line={590}
\begin{itemize}[noitemsep,topsep=\mdcompacttopsep]%mdk

%mdk-data-line={590}
\item\mdline{590}Home Page :: \mdline{590}\href{https://galois.com/blog/2009/03/trustworthy-voting-systems/}{{\ttfamily https://\hspace{0pt}galois.\hspace{0pt}com/\hspace{0pt}blog/\hspace{0pt}2009/\hspace{0pt}03/\hspace{0pt}trustworthy-\hspace{0pt}voting-\hspace{0pt}systems/\hspace{0pt}}}\mdline{590}%mdk
%mdk
\end{itemize}%mdk

%mdk-data-line={592}
\subsection{\mdline{592}4.2.\hspace*{0.5em}\mdline{592}Everest Projects}\label{sec-everest-projects}%mdk%mdk

%mdk-data-line={593}
\subsubsection{\mdline{593}4.2.1.\hspace*{0.5em}\mdline{593}Everest}\label{sec-everest}%mdk%mdk

%mdk-data-line={594}
\begin{itemize}[noitemsep,topsep=\mdcompacttopsep]%mdk

%mdk-data-line={594}
\item\mdline{594}Home Page ::%mdk

%mdk-data-line={595}
\item\mdline{595}Source :: \mdline{595}\href{https://github.com/project-everest/everest.git}{{\ttfamily https://\hspace{0pt}github.\hspace{0pt}com/\hspace{0pt}project-\hspace{0pt}everest/\hspace{0pt}everest.\hspace{0pt}git}}\mdline{595}%mdk
%mdk
\end{itemize}%mdk

%mdk-data-line={597}
\subsubsection{\mdline{597}4.2.2.\hspace*{0.5em}\mdline{597}Quackyducky}\label{sec-quackyducky}%mdk%mdk

%mdk-data-line={598}
\begin{itemize}[noitemsep,topsep=\mdcompacttopsep]%mdk

%mdk-data-line={598}
\item\mdline{598}Source :: \mdline{598}\href{https://github.com/project-everest/quackyducky}{{\ttfamily https://\hspace{0pt}github.\hspace{0pt}com/\hspace{0pt}project-\hspace{0pt}everest/\hspace{0pt}quackyducky}}\mdline{598}%mdk
%mdk
\end{itemize}%mdk

%mdk-data-line={600}
\noindent\mdline{600}QuackyDucky is a small tool to translate informal specification of
message formats found in RFC (in particular for TLS 1.3) into formal
F\mdline{602}\#\mdline{602} specifications, which are in turn transformed into efficient and
verified parser implementations.%mdk

%mdk-data-line={605}
\subsection{\mdline{605}4.3.\hspace*{0.5em}\mdline{605}Other Projects}\label{sec-other-projects}%mdk%mdk

%mdk-data-line={606}
\subsubsection{\mdline{606}4.3.1.\hspace*{0.5em}\mdline{606}CakeML}\label{sec-cakeml}%mdk%mdk

%mdk-data-line={607}
\begin{itemize}[noitemsep,topsep=\mdcompacttopsep]%mdk

%mdk-data-line={607}
\item\mdline{607}Home page :: \mdline{607}\href{https://cakeml.org/}{{\ttfamily https://\hspace{0pt}cakeml.\hspace{0pt}org/\hspace{0pt}}}\mdline{607}%mdk

%mdk-data-line={608}
\item\mdline{608}Source :: \mdline{608}\href{https://github.com/CakeML/cakeml}{{\ttfamily https://\hspace{0pt}github.\hspace{0pt}com/\hspace{0pt}CakeML/\hspace{0pt}cakeml}}\mdline{608}%mdk
%mdk
\end{itemize}%mdk

%mdk-data-line={610}
\noindent\mdline{610}\textquotedblleft{}CakeML is a functional programming language and an ecosystem of proofs
and tools built around the language. The ecosystem includes a
proven-correct compiler that can bootstrap itself.\textquotedblright{}\mdline{612}%mdk

%mdk-data-line={614}
\subsubsection{\mdline{614}4.3.2.\hspace*{0.5em}\mdline{614}VCC\mdline{614} \mdline{614}- A verifier for Concurrent C}\label{sec-vcc---a-verifier-for-concurrent-c}%mdk%mdk

%mdk-data-line={615}
\begin{itemize}[noitemsep,topsep=\mdcompacttopsep]%mdk

%mdk-data-line={615}
\item\mdline{615}Home Page :: \mdline{615}\href{https://www.microsoft.com/en-us/research/project/vcc-a-verifier-for-concurrent-c/}{{\ttfamily https://\hspace{0pt}www.\hspace{0pt}microsoft.\hspace{0pt}com/\hspace{0pt}en-\hspace{0pt}us/\hspace{0pt}research/\hspace{0pt}project/\hspace{0pt}vcc-\hspace{0pt}a-\hspace{0pt}verifier-\hspace{0pt}for-\hspace{0pt}concurrent-\hspace{0pt}c/\hspace{0pt}}}\mdline{615}%mdk

%mdk-data-line={616}
\item\mdline{616}Source :: \mdline{616}\href{https://github.com/Microsoft/vcc.git}{{\ttfamily https://\hspace{0pt}github.\hspace{0pt}com/\hspace{0pt}Microsoft/\hspace{0pt}vcc.\hspace{0pt}git}}\mdline{616}%mdk
%mdk
\end{itemize}%mdk

%mdk-data-line={618}
\subsubsection{\mdline{618}4.3.3.\hspace*{0.5em}\mdline{618}Compositional CompCert}\label{sec-compositional-compcert}%mdk%mdk

%mdk-data-line={619}
\begin{itemize}[noitemsep,topsep=\mdcompacttopsep]%mdk

%mdk-data-line={619}
\item\mdline{619}Source :: \mdline{619}\href{https://github.com/PrincetonUniversity/compcomp}{{\ttfamily https://\hspace{0pt}github.\hspace{0pt}com/\hspace{0pt}PrincetonUniversity/\hspace{0pt}compcomp}}\mdline{619}%mdk
%mdk
\end{itemize}%mdk

%mdk-data-line={621}
\noindent\mdline{621}Compcert supporting separate compilation. Last modified in 2015.%mdk

%mdk-data-line={623}
\subsubsection{\mdline{623}4.3.4.\hspace*{0.5em}\mdline{623}GaloisInc Projects}\label{sec-galoisinc-projects}%mdk%mdk

%mdk-data-line={624}
\begin{itemize}%mdk

%mdk-data-line={624}
\item{}
%mdk-data-line={624}
\mdline{624}Home Page :: \mdline{624}\href{https://galois.com/}{{\ttfamily https://\hspace{0pt}galois.\hspace{0pt}com/\hspace{0pt}}}\mdline{624}%mdk%mdk

%mdk-data-line={625}
\item{}
%mdk-data-line={625}
\mdline{625}Source :: \mdline{625}\href{https://github.com/GaloisInc}{{\ttfamily https://\hspace{0pt}github.\hspace{0pt}com/\hspace{0pt}GaloisInc}}\mdline{625}
  \mdline{626}"\mdline{626}Galois develops technology to guarantee the trustworthiness of
  systems where failure is unacceptable.%mdk

%mdk-data-line={629}
\mdline{629}  We apply cutting edge computer science and mathematics to advance the
  state of the art in software and hardware trustworthiness.\mdline{630}"\mdline{630}%mdk%mdk
%mdk
\end{itemize}%mdk

%mdk-data-line={632}
\subsubsection{\mdline{632}4.3.5.\hspace*{0.5em}\mdline{632}Bedrock}\label{sec-bedrock}%mdk%mdk

%mdk-data-line={633}
\begin{itemize}[noitemsep,topsep=\mdcompacttopsep]%mdk

%mdk-data-line={633}
\item\mdline{633}Home Page :: \mdline{633}\href{http://plv.csail.mit.edu/bedrock/}{{\ttfamily http://\hspace{0pt}plv.\hspace{0pt}csail.\hspace{0pt}mit.\hspace{0pt}edu/\hspace{0pt}bedrock/\hspace{0pt}}}\mdline{633}%mdk
%mdk
\end{itemize}%mdk

%mdk-data-line={635}
\subsubsection{\mdline{635}4.3.6.\hspace*{0.5em}\mdline{635}FSCQ}\label{sec-fscq}%mdk%mdk

%mdk-data-line={636}
\begin{itemize}[noitemsep,topsep=\mdcompacttopsep]%mdk

%mdk-data-line={636}
\item\mdline{636}Home Page :: \mdline{636}\href{http://css.csail.mit.edu/fscq/}{{\ttfamily http://\hspace{0pt}css.\hspace{0pt}csail.\hspace{0pt}mit.\hspace{0pt}edu/\hspace{0pt}fscq/\hspace{0pt}}}\mdline{636}%mdk
%mdk
\end{itemize}%mdk

%mdk-data-line={638}
\noindent\mdline{638}A file system verified in Coq using a separation logic for reasoning about crash safety%mdk

%mdk-data-line={640}
\subsubsection{\mdline{640}4.3.7.\hspace*{0.5em}\mdline{640}Ur/Web}\label{sec-urweb}%mdk%mdk

%mdk-data-line={641}
\begin{itemize}[noitemsep,topsep=\mdcompacttopsep]%mdk

%mdk-data-line={641}
\item\mdline{641}Home Page :: \mdline{641}\href{http://plv.csail.mit.edu/ur/}{{\ttfamily http://\hspace{0pt}plv.\hspace{0pt}csail.\hspace{0pt}mit.\hspace{0pt}edu/\hspace{0pt}ur/\hspace{0pt}}}\mdline{641}%mdk
%mdk
\end{itemize}%mdk

%mdk-data-line={643}
\section{\mdline{643}5.\hspace*{0.5em}\mdline{643}Formal Methods Researchers}\label{sec-formal-methods-researchers}%mdk%mdk

%mdk-data-line={644}
\noindent\mdline{644}Alphabetically by last name, then first.%mdk

%mdk-data-line={646}
\subsection{\mdline{646}5.1.\hspace*{0.5em}\mdline{646}Andrew W Appel}\label{sec-andrew-w-appel}%mdk%mdk

%mdk-data-line={647}
\begin{itemize}[noitemsep,topsep=\mdcompacttopsep]%mdk

%mdk-data-line={647}
\item\mdline{647}Home Page :: \mdline{647}\href{http://www.cs.princeton.edu/~appel/index.html}{{\ttfamily http://\hspace{0pt}www.\hspace{0pt}cs.\hspace{0pt}princeton.\hspace{0pt}edu/\hspace{0pt}\textasciitilde{}appel/\hspace{0pt}index.\hspace{0pt}html}}\mdline{647}%mdk
%mdk
\end{itemize}%mdk

%mdk-data-line={649}
\subsection{\mdline{649}5.2.\hspace*{0.5em}\mdline{649}Adam Chlipala}\label{sec-adam-chlipala}%mdk%mdk

%mdk-data-line={650}
\begin{itemize}[noitemsep,topsep=\mdcompacttopsep]%mdk

%mdk-data-line={650}
\item\mdline{650}Home Page :: \mdline{650}\href{http://adam.chlipala.net/}{{\ttfamily http://\hspace{0pt}adam.\hspace{0pt}chlipala.\hspace{0pt}net/\hspace{0pt}}}\mdline{650}%mdk
%mdk
\end{itemize}%mdk

%mdk-data-line={652}
\subsection{\mdline{652}5.3.\hspace*{0.5em}\mdline{652}Robert Harper}\label{sec-robert-harper}%mdk%mdk

%mdk-data-line={653}
\begin{itemize}[noitemsep,topsep=\mdcompacttopsep]%mdk

%mdk-data-line={653}
\item\mdline{653}Home Page :: \mdline{653}\href{http://www.cs.cmu.edu/~rwh/}{{\ttfamily http://\hspace{0pt}www.\hspace{0pt}cs.\hspace{0pt}cmu.\hspace{0pt}edu/\hspace{0pt}\textasciitilde{}rwh/\hspace{0pt}}}\mdline{653}%mdk
%mdk
\end{itemize}%mdk

%mdk-data-line={655}
\subsection{\mdline{655}5.4.\hspace*{0.5em}\mdline{655}Benjamin Pierce}\label{sec-benjamin-pierce}%mdk%mdk

%mdk-data-line={656}
\begin{itemize}[noitemsep,topsep=\mdcompacttopsep]%mdk

%mdk-data-line={656}
\item\mdline{656}Home Page :: \mdline{656}\href{http://www.cis.upenn.edu/~bcpierce/}{{\ttfamily http://\hspace{0pt}www.\hspace{0pt}cis.\hspace{0pt}upenn.\hspace{0pt}edu/\hspace{0pt}\textasciitilde{}bcpierce/\hspace{0pt}}}\mdline{656}%mdk

%mdk-data-line={657}
\item\mdline{657}LinkedIn :: \mdline{657}\href{https://github.com/bcpierce00}{{\ttfamily https://\hspace{0pt}github.\hspace{0pt}com/\hspace{0pt}bcpierce00}}\mdline{657}
Professor Department of Computer and Information Science University of Pennsylvania.
Author of\mdline{659}~\href{http://www.cis.upenn.edu/~bcpierce/sf}{Software Foundations}\mdline{659}.%mdk
%mdk
\end{itemize}%mdk

%mdk-data-line={661}
\subsection{\mdline{661}5.5.\hspace*{0.5em}\mdline{661}Zhong Shao}\label{sec-zhong-shao}%mdk%mdk

%mdk-data-line={662}
\begin{itemize}[noitemsep,topsep=\mdcompacttopsep]%mdk

%mdk-data-line={662}
\item\mdline{662}Home Page :: \mdline{662}\href{http://cs-www.cs.yale.edu/homes/shao/}{{\ttfamily http://\hspace{0pt}cs-\hspace{0pt}www.\hspace{0pt}cs.\hspace{0pt}yale.\hspace{0pt}edu/\hspace{0pt}homes/\hspace{0pt}shao/\hspace{0pt}}}\mdline{662}%mdk
%mdk
\end{itemize}%mdk

%mdk-data-line={664}
\section{\mdline{664}6.\hspace*{0.5em}\mdline{664}Statistics}\label{sec-statistics}%mdk%mdk

%mdk-data-line={665}
\noindent\mdline{665}Here are some statistics for projects using Coq. The \mdline{665}\textquotedblleft{}Types\textquotedblright{}\mdline{665} column is
the number of inductive types defined. The \mdline{666}\textquotedblleft{}Defs\textquotedblright{}\mdline{666} column is the number
of \mdline{667}\textquotedblleft{}Definitions\textquotedblright{}\mdline{667}. Some projects are broken up by component.%mdk
\begin{mdtabular}{7}{\dimeval{(\linewidth-6cm)/6}}{1ex}%mdk
\begin{tabular}{lrrrrrl}\midrule[\dimpx{2}]
\multicolumn{1}{|l}{{\mdseries\mdline{670} Project}}&\multicolumn{1}{|r}{{\mdseries\mdline{670} \mdline{670}\#\mdline{670}Coq}}&\multicolumn{1}{|r}{{\mdseries\mdline{670} SLOC}}&\multicolumn{1}{|r}{{\mdseries\mdline{670} Proofs}}&\multicolumn{1}{|r}{{\mdseries\mdline{670} Types}}&\multicolumn{1}{|r}{{\mdseries\mdline{670} Defs}}&\multicolumn{1}{c|}{\mdinline{padding-top=3pt,padding-bottom=3pt,width=6cm}{{\mdseries\mdline{670} Notes}}}\\
\multicolumn{1}{|l}{{\mdseries\mdline{671}}}&\multicolumn{1}{|r}{{\mdseries\mdline{671} Files}}&\multicolumn{1}{|r}{{\mdseries\mdline{671}}}&\multicolumn{1}{|r}{{\mdseries\mdline{671}}}&\multicolumn{1}{|r}{{\mdseries\mdline{671}}}&\multicolumn{1}{|r}{{\mdseries\mdline{671}}}&\multicolumn{1}{c|}{\mdinline{padding-bottom=3pt,width=6cm}{{\mdseries\mdline{671}}}}\\

\midrule
\multicolumn{1}{|l}{\mdline{673} certikos}&\multicolumn{1}{|r}{\mdline{673}}&\multicolumn{1}{|r}{\mdline{673}}&\multicolumn{1}{|r}{\mdline{673}}&\multicolumn{1}{|r}{\mdline{673}}&\multicolumn{1}{|r}{\mdline{673}}&\multicolumn{1}{l|}{\mdinline{padding-top=3pt,width=6cm}{\mdline{673} Kit Operating System}}\\
\multicolumn{1}{|l}{\mdline{674} \mdline{674}- compcert}&\multicolumn{1}{|r}{\mdline{674} 205}&\multicolumn{1}{|r}{\mdline{674} 206270}&\multicolumn{1}{|r}{\mdline{674} 5359}&\multicolumn{1}{|r}{\mdline{674} 465}&\multicolumn{1}{|r}{\mdline{674} 2945}&\multicolumn{1}{l|}{\mdinline{width=6cm}{\mdline{674} Modified compcert}}\\
\multicolumn{1}{|l}{\mdline{675} \mdline{675}- compcertx}&\multicolumn{1}{|r}{\mdline{675} 50}&\multicolumn{1}{|r}{\mdline{675} 8645}&\multicolumn{1}{|r}{\mdline{675} 326}&\multicolumn{1}{|r}{\mdline{675} 25}&\multicolumn{1}{|r}{\mdline{675} 60}&\multicolumn{1}{l|}{\mdinline{width=6cm}{\mdline{675} Compcert for sep compilation}}\\
\multicolumn{1}{|l}{\mdline{676} \mdline{676}- liblayers}&\multicolumn{1}{|r}{\mdline{676} 55}&\multicolumn{1}{|r}{\mdline{676} 22122}&\multicolumn{1}{|r}{\mdline{676} 725}&\multicolumn{1}{|r}{\mdline{676} 43}&\multicolumn{1}{|r}{\mdline{676} 189}&\multicolumn{1}{l|}{\mdinline{width=6cm}{\mdline{676}}}\\
\multicolumn{1}{|l}{\mdline{677} \mdline{677}- mcertikos}&\multicolumn{1}{|r}{\mdline{677} 449}&\multicolumn{1}{|r}{\mdline{677} 207281}&\multicolumn{1}{|r}{\mdline{677} 5591}&\multicolumn{1}{|r}{\mdline{677} 324}&\multicolumn{1}{|r}{\mdline{677} 1757}&\multicolumn{1}{l|}{\mdinline{width=6cm}{\mdline{677}}}\\
\midrule
\multicolumn{1}{|l}{\mdline{679} cfml}&\multicolumn{1}{|r}{\mdline{679} 177}&\multicolumn{1}{|r}{\mdline{679} 65587}&\multicolumn{1}{|r}{\mdline{679} 2769}&\multicolumn{1}{|r}{\mdline{679} 131}&\multicolumn{1}{|r}{\mdline{679} 892}&\multicolumn{1}{l|}{\mdinline{width=6cm}{\mdline{679} Tool for proving OCaml}}\\
\multicolumn{1}{|l}{\mdline{680}}&\multicolumn{1}{|r}{\mdline{680}}&\multicolumn{1}{|r}{\mdline{680}}&\multicolumn{1}{|r}{\mdline{680}}&\multicolumn{1}{|r}{\mdline{680}}&\multicolumn{1}{|r}{\mdline{680}}&\multicolumn{1}{l|}{\mdinline{width=6cm}{\mdline{680} programs in Separation Logic}}\\
\multicolumn{1}{|l}{\mdline{681} ch2o}&\multicolumn{1}{|r}{\mdline{681} 116}&\multicolumn{1}{|r}{\mdline{681} 49351}&\multicolumn{1}{|r}{\mdline{681} 4472}&\multicolumn{1}{|r}{\mdline{681} 153}&\multicolumn{1}{|r}{\mdline{681} 424}&\multicolumn{1}{l|}{\mdinline{width=6cm}{\mdline{681} A formalization of the C11}}\\
\multicolumn{1}{|l}{\mdline{682}}&\multicolumn{1}{|r}{\mdline{682}}&\multicolumn{1}{|r}{\mdline{682}}&\multicolumn{1}{|r}{\mdline{682}}&\multicolumn{1}{|r}{\mdline{682}}&\multicolumn{1}{|r}{\mdline{682}}&\multicolumn{1}{l|}{\mdinline{width=6cm}{\mdline{682} standard in Coq}}\\
\multicolumn{1}{|l}{\mdline{683} compcert 3.3}&\multicolumn{1}{|r}{\mdline{683} 231}&\multicolumn{1}{|r}{\mdline{683} 215450}&\multicolumn{1}{|r}{\mdline{683} 6728}&\multicolumn{1}{|r}{\mdline{683} 593}&\multicolumn{1}{|r}{\mdline{683} 4601}&\multicolumn{1}{l|}{\mdinline{width=6cm}{\mdline{683} Formally Verified C Compiler}}\\
\multicolumn{1}{|l}{\mdline{684} compcert 3.4}&\multicolumn{1}{|r}{\mdline{684} 225}&\multicolumn{1}{|r}{\mdline{684} 177117}&\multicolumn{1}{|r}{\mdline{684} 6729}&\multicolumn{1}{|r}{\mdline{684} 525}&\multicolumn{1}{|r}{\mdline{684} 3031}&\multicolumn{1}{l|}{\mdinline{width=6cm}{\mdline{684} Formally Verified C Compiler}}\\
\multicolumn{1}{|l}{\mdline{685} coq}&\multicolumn{1}{|r}{\mdline{685} 1984}&\multicolumn{1}{|r}{\mdline{685} 247663}&\multicolumn{1}{|r}{\mdline{685} 12131}&\multicolumn{1}{|r}{\mdline{685} 1097}&\multicolumn{1}{|r}{\mdline{685} 5666}&\multicolumn{1}{l|}{\mdinline{width=6cm}{\mdline{685} Coq Proof Assistant Library}}\\
\multicolumn{1}{|l}{\mdline{686} Coq-dL}&\multicolumn{1}{|r}{\mdline{686} 84}&\multicolumn{1}{|r}{\mdline{686} 83871}&\multicolumn{1}{|r}{\mdline{686} 2849}&\multicolumn{1}{|r}{\mdline{686} 50}&\multicolumn{1}{|r}{\mdline{686} 894}&\multicolumn{1}{l|}{\mdinline{width=6cm}{\mdline{686} Formalization of KeYmaeraX}}\\
\multicolumn{1}{|l}{\mdline{687}}&\multicolumn{1}{|r}{\mdline{687}}&\multicolumn{1}{|r}{\mdline{687}}&\multicolumn{1}{|r}{\mdline{687}}&\multicolumn{1}{|r}{\mdline{687}}&\multicolumn{1}{|r}{\mdline{687}}&\multicolumn{1}{l|}{\mdinline{width=6cm}{\mdline{687} in Coq}}\\
\multicolumn{1}{|l}{\mdline{688} coquelicot}&\multicolumn{1}{|r}{\mdline{688} 28}&\multicolumn{1}{|r}{\mdline{688} 41615}&\multicolumn{1}{|r}{\mdline{688} 1751}&\multicolumn{1}{|r}{\mdline{688} 6}&\multicolumn{1}{|r}{\mdline{688} 324}&\multicolumn{1}{l|}{\mdinline{width=6cm}{\mdline{688} User friendly Calculus in Coq}}\\
\multicolumn{1}{|l}{\mdline{689} corespec}&\multicolumn{1}{|r}{\mdline{689} 41}&\multicolumn{1}{|r}{\mdline{689} 35694}&\multicolumn{1}{|r}{\mdline{689} 1351}&\multicolumn{1}{|r}{\mdline{689} 33}&\multicolumn{1}{|r}{\mdline{689} 215}&\multicolumn{1}{l|}{\mdinline{width=6cm}{\mdline{689} Formalization of Haskell Core}}\\
\multicolumn{1}{|l}{\mdline{690}}&\multicolumn{1}{|r}{\mdline{690}}&\multicolumn{1}{|r}{\mdline{690}}&\multicolumn{1}{|r}{\mdline{690}}&\multicolumn{1}{|r}{\mdline{690}}&\multicolumn{1}{|r}{\mdline{690}}&\multicolumn{1}{l|}{\mdinline{width=6cm}{\mdline{690} in Coq}}\\
\multicolumn{1}{|l}{\mdline{691} Corn}&\multicolumn{1}{|r}{\mdline{691} 348}&\multicolumn{1}{|r}{\mdline{691} 156363}&\multicolumn{1}{|r}{\mdline{691} 6895}&\multicolumn{1}{|r}{\mdline{691} 33}&\multicolumn{1}{|r}{\mdline{691} 2118}&\multicolumn{1}{l|}{\mdinline{width=6cm}{\mdline{691} Coq Constructive Repository at}}\\
\multicolumn{1}{|l}{\mdline{692}}&\multicolumn{1}{|r}{\mdline{692}}&\multicolumn{1}{|r}{\mdline{692}}&\multicolumn{1}{|r}{\mdline{692}}&\multicolumn{1}{|r}{\mdline{692}}&\multicolumn{1}{|r}{\mdline{692}}&\multicolumn{1}{l|}{\mdinline{width=6cm}{\mdline{692} Nijmegen (Reals)}}\\
\multicolumn{1}{|l}{\mdline{693} DeepSpecDB}&\multicolumn{1}{|r}{\mdline{693} 55}&\multicolumn{1}{|r}{\mdline{693} 32788}&\multicolumn{1}{|r}{\mdline{693} 531}&\multicolumn{1}{|r}{\mdline{693} 30}&\multicolumn{1}{|r}{\mdline{693} 1151}&\multicolumn{1}{l|}{\mdinline{padding-bottom=4pt,width=6cm}{\mdline{693} DeepSpec Data Base}}\\
\midrule[\dimpx{2}]
\end{tabular}\end{mdtabular}
\begin{mdtabular}{7}{\dimeval{(\linewidth-6cm)/6}}{1ex}%mdk
\begin{tabular}{lrrrrrl}\midrule[\dimpx{2}]
\multicolumn{1}{|l}{{\mdseries\mdline{698} Project}}&\multicolumn{1}{|r}{{\mdseries\mdline{698} \mdline{698}\#\mdline{698}Coq}}&\multicolumn{1}{|r}{{\mdseries\mdline{698} SLOC}}&\multicolumn{1}{|r}{{\mdseries\mdline{698} Proofs}}&\multicolumn{1}{|r}{{\mdseries\mdline{698} Types}}&\multicolumn{1}{|r}{{\mdseries\mdline{698} Defs}}&\multicolumn{1}{c|}{\mdinline{padding-top=3pt,padding-bottom=3pt,width=6cm}{{\mdseries\mdline{698} Notes}}}\\
\multicolumn{1}{|l}{{\mdseries\mdline{699}}}&\multicolumn{1}{|r}{{\mdseries\mdline{699} Files}}&\multicolumn{1}{|r}{{\mdseries\mdline{699}}}&\multicolumn{1}{|r}{{\mdseries\mdline{699}}}&\multicolumn{1}{|r}{{\mdseries\mdline{699}}}&\multicolumn{1}{|r}{{\mdseries\mdline{699}}}&\multicolumn{1}{c|}{\mdinline{padding-bottom=3pt,width=6cm}{{\mdseries\mdline{699}}}}\\

\midrule
\multicolumn{1}{|l}{\mdline{701} certikos}&\multicolumn{1}{|r}{\mdline{701}}&\multicolumn{1}{|r}{\mdline{701}}&\multicolumn{1}{|r}{\mdline{701}}&\multicolumn{1}{|r}{\mdline{701}}&\multicolumn{1}{|r}{\mdline{701}}&\multicolumn{1}{l|}{\mdinline{padding-top=3pt,width=6cm}{\mdline{701} Kit Operating System}}\\
\multicolumn{1}{|l}{\mdline{702} \mdline{702}- compcert}&\multicolumn{1}{|r}{\mdline{702} 205}&\multicolumn{1}{|r}{\mdline{702} 206270}&\multicolumn{1}{|r}{\mdline{702} 5359}&\multicolumn{1}{|r}{\mdline{702} 465}&\multicolumn{1}{|r}{\mdline{702} 2945}&\multicolumn{1}{l|}{\mdinline{width=6cm}{\mdline{702} Modified compcert}}\\
\multicolumn{1}{|l}{\mdline{703} \mdline{703}- compcertx}&\multicolumn{1}{|r}{\mdline{703} 50}&\multicolumn{1}{|r}{\mdline{703} 8645}&\multicolumn{1}{|r}{\mdline{703} 326}&\multicolumn{1}{|r}{\mdline{703} 25}&\multicolumn{1}{|r}{\mdline{703} 60}&\multicolumn{1}{l|}{\mdinline{width=6cm}{\mdline{703} Compcert for sep compilation}}\\
\multicolumn{1}{|l}{\mdline{704} \mdline{704}- liblayers}&\multicolumn{1}{|r}{\mdline{704} 55}&\multicolumn{1}{|r}{\mdline{704} 22122}&\multicolumn{1}{|r}{\mdline{704} 725}&\multicolumn{1}{|r}{\mdline{704} 43}&\multicolumn{1}{|r}{\mdline{704} 189}&\multicolumn{1}{l|}{\mdinline{width=6cm}{\mdline{704}}}\\
\multicolumn{1}{|l}{\mdline{705} \mdline{705}- mcertikos}&\multicolumn{1}{|r}{\mdline{705} 449}&\multicolumn{1}{|r}{\mdline{705} 207281}&\multicolumn{1}{|r}{\mdline{705} 5591}&\multicolumn{1}{|r}{\mdline{705} 324}&\multicolumn{1}{|r}{\mdline{705} 1757}&\multicolumn{1}{l|}{\mdinline{width=6cm}{\mdline{705}}}\\
\midrule
\multicolumn{1}{|l}{\mdline{707} dsss17\mdline{707} \mdline{707}-total}&\multicolumn{1}{|r}{\mdline{707} 490}&\multicolumn{1}{|r}{\mdline{707} 302318}&\multicolumn{1}{|r}{\mdline{707} 10580}&\multicolumn{1}{|r}{\mdline{707} 1061}&\multicolumn{1}{|r}{\mdline{707} 5162}&\multicolumn{1}{l|}{\mdinline{width=6cm}{\mdline{707} DeepSpec Summer School 2017}}\\
\multicolumn{1}{|l}{\mdline{708} \mdline{708}- auto}&\multicolumn{1}{|r}{\mdline{708} 6}&\multicolumn{1}{|r}{\mdline{708} 3495}&\multicolumn{1}{|r}{\mdline{708} 148}&\multicolumn{1}{|r}{\mdline{708} 16}&\multicolumn{1}{|r}{\mdline{708} 23}&\multicolumn{1}{l|}{\mdinline{width=6cm}{\mdline{708} Proof Automation\mdline{708} \mdline{708}- Chlipala}}\\
\multicolumn{1}{|l}{\mdline{709} \mdline{709}- CAL}&\multicolumn{1}{|r}{\mdline{709} 378}&\multicolumn{1}{|r}{\mdline{709} 245501}&\multicolumn{1}{|r}{\mdline{709} 8762}&\multicolumn{1}{|r}{\mdline{709} 834}&\multicolumn{1}{|r}{\mdline{709} 4321}&\multicolumn{1}{l|}{\mdinline{width=6cm}{\mdline{709} Certifying software with crashes}}\\
\multicolumn{1}{|l}{\mdline{710}}&\multicolumn{1}{|r}{\mdline{710}}&\multicolumn{1}{|r}{\mdline{710}}&\multicolumn{1}{|r}{\mdline{710}}&\multicolumn{1}{|r}{\mdline{710}}&\multicolumn{1}{|r}{\mdline{710}}&\multicolumn{1}{l|}{\mdinline{width=6cm}{\mdline{710} (Cert Abstr layers)}}\\
\multicolumn{1}{|l}{\mdline{711} \mdline{711}- compiler}&\multicolumn{1}{|r}{\mdline{711} 6}&\multicolumn{1}{|r}{\mdline{711} 3813}&\multicolumn{1}{|r}{\mdline{711} 116}&\multicolumn{1}{|r}{\mdline{711} 22}&\multicolumn{1}{|r}{\mdline{711} 48}&\multicolumn{1}{l|}{\mdinline{width=6cm}{\mdline{711} Compiler for Imp (Xaxier)}}\\
\multicolumn{1}{|l}{\mdline{712} \mdline{712}- Metalib}&\multicolumn{1}{|r}{\mdline{712} 18}&\multicolumn{1}{|r}{\mdline{712} 7015}&\multicolumn{1}{|r}{\mdline{712} 307}&\multicolumn{1}{|r}{\mdline{712} 7}&\multicolumn{1}{|r}{\mdline{712} 97}&\multicolumn{1}{l|}{\mdinline{width=6cm}{\mdline{712} Support for Stlc}}\\
\multicolumn{1}{|l}{\mdline{713} \mdline{713}- qc}&\multicolumn{1}{|r}{\mdline{713} 9}&\multicolumn{1}{|r}{\mdline{713} 5073}&\multicolumn{1}{|r}{\mdline{713} 26}&\multicolumn{1}{|r}{\mdline{713} 24}&\multicolumn{1}{|r}{\mdline{713} 84}&\multicolumn{1}{l|}{\mdinline{width=6cm}{\mdline{713} QuickChick}}\\
\multicolumn{1}{|l}{\mdline{714} \mdline{714}- SF}&\multicolumn{1}{|r}{\mdline{714} 34}&\multicolumn{1}{|r}{\mdline{714} 20875}&\multicolumn{1}{|r}{\mdline{714} 656}&\multicolumn{1}{|r}{\mdline{714} 107}&\multicolumn{1}{|r}{\mdline{714} 316}&\multicolumn{1}{l|}{\mdinline{width=6cm}{\mdline{714} Software Foundations}}\\
\multicolumn{1}{|l}{\mdline{715} \mdline{715}- Stlc}&\multicolumn{1}{|r}{\mdline{715} 12}&\multicolumn{1}{|r}{\mdline{715} 8942}&\multicolumn{1}{|r}{\mdline{715} 392}&\multicolumn{1}{|r}{\mdline{715} 27}&\multicolumn{1}{|r}{\mdline{715} 62}&\multicolumn{1}{l|}{\mdinline{width=6cm}{\mdline{715} Lang Spec and Variable binding}}\\
\multicolumn{1}{|l}{\mdline{716} \mdline{716}- vminus}&\multicolumn{1}{|r}{\mdline{716} 27}&\multicolumn{1}{|r}{\mdline{716} 7604}&\multicolumn{1}{|r}{\mdline{716} 173}&\multicolumn{1}{|r}{\mdline{716} 24}&\multicolumn{1}{|r}{\mdline{716} 211}&\multicolumn{1}{l|}{\mdinline{width=6cm}{\mdline{716} Vellvm: Verifying the LLVM}}\\
\midrule
\multicolumn{1}{|l}{\mdline{718} dsss18\mdline{718} \mdline{718}- total}&\multicolumn{1}{|r}{\mdline{718} 743}&\multicolumn{1}{|r}{\mdline{718} 258178}&\multicolumn{1}{|r}{\mdline{718} 8617}&\multicolumn{1}{|r}{\mdline{718} 614}&\multicolumn{1}{|r}{\mdline{718} 4399}&\multicolumn{1}{l|}{\mdinline{width=6cm}{\mdline{718} DeepSpec Summeer School 2018}}\\
\multicolumn{1}{|l}{\mdline{719} \mdline{719}- charIO}&\multicolumn{1}{|r}{\mdline{719} 18}&\multicolumn{1}{|r}{\mdline{719} 3704}&\multicolumn{1}{|r}{\mdline{719} 84}&\multicolumn{1}{|r}{\mdline{719} 23}&\multicolumn{1}{|r}{\mdline{719} 167}&\multicolumn{1}{l|}{\mdinline{width=6cm}{\mdline{719}}}\\
\multicolumn{1}{|l}{\mdline{720} \mdline{720}- dw}&\multicolumn{1}{|r}{\mdline{720} 88}&\multicolumn{1}{|r}{\mdline{720} 17300}&\multicolumn{1}{|r}{\mdline{720} 199}&\multicolumn{1}{|r}{\mdline{720} 45}&\multicolumn{1}{|r}{\mdline{720} 572}&\multicolumn{1}{l|}{\mdinline{width=6cm}{\mdline{720}}}\\
\multicolumn{1}{|l}{\mdline{721} \mdline{721}- kami}&\multicolumn{1}{|r}{\mdline{721} 75}&\multicolumn{1}{|r}{\mdline{721} 47358}&\multicolumn{1}{|r}{\mdline{721} 1996}&\multicolumn{1}{|r}{\mdline{721} 97}&\multicolumn{1}{|r}{\mdline{721} 624}&\multicolumn{1}{l|}{\mdinline{width=6cm}{\mdline{721}}}\\
\multicolumn{1}{|l}{\mdline{722} \mdline{722}- lf}&\multicolumn{1}{|r}{\mdline{722} 38}&\multicolumn{1}{|r}{\mdline{722} 17147}&\multicolumn{1}{|r}{\mdline{722} 480}&\multicolumn{1}{|r}{\mdline{722} 78}&\multicolumn{1}{|r}{\mdline{722} 194}&\multicolumn{1}{l|}{\mdinline{width=6cm}{\mdline{722} SF\mdline{722} \mdline{722}- Logical Foundations}}\\
\multicolumn{1}{|l}{\mdline{723} \mdline{723}- plf}&\multicolumn{1}{|r}{\mdline{723} 48}&\multicolumn{1}{|r}{\mdline{723} 33305}&\multicolumn{1}{|r}{\mdline{723} 589}&\multicolumn{1}{|r}{\mdline{723} 140}&\multicolumn{1}{|r}{\mdline{723} 249}&\multicolumn{1}{l|}{\mdinline{width=6cm}{\mdline{723} SF\mdline{723} \mdline{723}- Programming Languages}}\\
\multicolumn{1}{|l}{\mdline{724}}&\multicolumn{1}{|r}{\mdline{724}}&\multicolumn{1}{|r}{\mdline{724}}&\multicolumn{1}{|r}{\mdline{724}}&\multicolumn{1}{|r}{\mdline{724}}&\multicolumn{1}{|r}{\mdline{724}}&\multicolumn{1}{l|}{\mdinline{width=6cm}{\mdline{724} Foundations}}\\
\multicolumn{1}{|l}{\mdline{725} \mdline{725}- qc}&\multicolumn{1}{|r}{\mdline{725} 10}&\multicolumn{1}{|r}{\mdline{725} 6767}&\multicolumn{1}{|r}{\mdline{725} 21}&\multicolumn{1}{|r}{\mdline{725} 28}&\multicolumn{1}{|r}{\mdline{725} 80}&\multicolumn{1}{l|}{\mdinline{width=6cm}{\mdline{725} Quick Chick}}\\
\multicolumn{1}{|l}{\mdline{726} \mdline{726}- vc}&\multicolumn{1}{|r}{\mdline{726} 15}&\multicolumn{1}{|r}{\mdline{726} 9172}&\multicolumn{1}{|r}{\mdline{726} 4915}&\multicolumn{1}{|r}{\mdline{726} 174}&\multicolumn{1}{|r}{\mdline{726} 1685}&\multicolumn{1}{l|}{\mdinline{width=6cm}{\mdline{726} Verifiable C (Proofs using VST)}}\\
\multicolumn{1}{|l}{\mdline{727} \mdline{727}- vfa}&\multicolumn{1}{|r}{\mdline{727} 30}&\multicolumn{1}{|r}{\mdline{727} 8680}&\multicolumn{1}{|r}{\mdline{727} 205}&\multicolumn{1}{|r}{\mdline{727} 28}&\multicolumn{1}{|r}{\mdline{727} 168}&\multicolumn{1}{l|}{\mdinline{width=6cm}{\mdline{727} SF\mdline{727} \mdline{727}- Verified Functional}}\\
\multicolumn{1}{|l}{\mdline{728}}&\multicolumn{1}{|r}{\mdline{728}}&\multicolumn{1}{|r}{\mdline{728}}&\multicolumn{1}{|r}{\mdline{728}}&\multicolumn{1}{|r}{\mdline{728}}&\multicolumn{1}{|r}{\mdline{728}}&\multicolumn{1}{l|}{\mdinline{width=6cm}{\mdline{728} Algorithms}}\\
\midrule
\multicolumn{1}{|l}{\mdline{730} fiat}&\multicolumn{1}{|r}{\mdline{730} 647}&\multicolumn{1}{|r}{\mdline{730} 197824}&\multicolumn{1}{|r}{\mdline{730} 5623}&\multicolumn{1}{|r}{\mdline{730} 76}&\multicolumn{1}{|r}{\mdline{730} 4075}&\multicolumn{1}{l|}{\mdinline{width=6cm}{\mdline{730} Deductive Synthesis of Abstract}}\\
\multicolumn{1}{|l}{\mdline{731}}&\multicolumn{1}{|r}{\mdline{731}}&\multicolumn{1}{|r}{\mdline{731}}&\multicolumn{1}{|r}{\mdline{731}}&\multicolumn{1}{|r}{\mdline{731}}&\multicolumn{1}{|r}{\mdline{731}}&\multicolumn{1}{l|}{\mdinline{width=6cm}{\mdline{731} Data Types in a Proof Assistant}}\\
\multicolumn{1}{|l}{\mdline{732} \mdline{732}- Narcissus}&\multicolumn{1}{|r}{\mdline{732} 72}&\multicolumn{1}{|r}{\mdline{732} 32310}&\multicolumn{1}{|r}{\mdline{732} 847}&\multicolumn{1}{|r}{\mdline{732} 12}&\multicolumn{1}{|r}{\mdline{732} 605}&\multicolumn{1}{l|}{\mdinline{width=6cm}{\mdline{732} Subset of Fiat for interface}}\\
\multicolumn{1}{|l}{\mdline{733}}&\multicolumn{1}{|r}{\mdline{733}}&\multicolumn{1}{|r}{\mdline{733}}&\multicolumn{1}{|r}{\mdline{733}}&\multicolumn{1}{|r}{\mdline{733}}&\multicolumn{1}{|r}{\mdline{733}}&\multicolumn{1}{l|}{\mdinline{padding-bottom=4pt,width=6cm}{\mdline{733} generation}}\\
\midrule[\dimpx{2}]
\end{tabular}\end{mdtabular}
\begin{mdtabular}{7}{\dimeval{(\linewidth-6cm)/6}}{1ex}%mdk
\begin{tabular}{lrrrrrl}\midrule[\dimpx{2}]
\multicolumn{1}{|l}{{\mdseries\mdline{738} Project}}&\multicolumn{1}{|r}{{\mdseries\mdline{738} \mdline{738}\#\mdline{738}Coq}}&\multicolumn{1}{|r}{{\mdseries\mdline{738} SLOC}}&\multicolumn{1}{|r}{{\mdseries\mdline{738} Proofs}}&\multicolumn{1}{|r}{{\mdseries\mdline{738} Types}}&\multicolumn{1}{|r}{{\mdseries\mdline{738} Defs}}&\multicolumn{1}{c|}{\mdinline{padding-top=3pt,padding-bottom=3pt,width=6cm}{{\mdseries\mdline{738} Notes}}}\\
\multicolumn{1}{|l}{{\mdseries\mdline{739}}}&\multicolumn{1}{|r}{{\mdseries\mdline{739} Files}}&\multicolumn{1}{|r}{{\mdseries\mdline{739}}}&\multicolumn{1}{|r}{{\mdseries\mdline{739}}}&\multicolumn{1}{|r}{{\mdseries\mdline{739}}}&\multicolumn{1}{|r}{{\mdseries\mdline{739}}}&\multicolumn{1}{c|}{\mdinline{padding-bottom=3pt,width=6cm}{{\mdseries\mdline{739}}}}\\

\midrule
\multicolumn{1}{|l}{\mdline{741} certikos}&\multicolumn{1}{|r}{\mdline{741}}&\multicolumn{1}{|r}{\mdline{741}}&\multicolumn{1}{|r}{\mdline{741}}&\multicolumn{1}{|r}{\mdline{741}}&\multicolumn{1}{|r}{\mdline{741}}&\multicolumn{1}{l|}{\mdinline{padding-top=3pt,width=6cm}{\mdline{741} Kit Operating System}}\\
\multicolumn{1}{|l}{\mdline{742} \mdline{742}- compcert}&\multicolumn{1}{|r}{\mdline{742} 205}&\multicolumn{1}{|r}{\mdline{742} 206270}&\multicolumn{1}{|r}{\mdline{742} 5359}&\multicolumn{1}{|r}{\mdline{742} 465}&\multicolumn{1}{|r}{\mdline{742} 2945}&\multicolumn{1}{l|}{\mdinline{width=6cm}{\mdline{742} Modified compcert}}\\
\multicolumn{1}{|l}{\mdline{743} \mdline{743}- compcertx}&\multicolumn{1}{|r}{\mdline{743} 50}&\multicolumn{1}{|r}{\mdline{743} 8645}&\multicolumn{1}{|r}{\mdline{743} 326}&\multicolumn{1}{|r}{\mdline{743} 25}&\multicolumn{1}{|r}{\mdline{743} 60}&\multicolumn{1}{l|}{\mdinline{width=6cm}{\mdline{743} Compcert for sep compilation}}\\
\multicolumn{1}{|l}{\mdline{744} \mdline{744}- liblayers}&\multicolumn{1}{|r}{\mdline{744} 55}&\multicolumn{1}{|r}{\mdline{744} 22122}&\multicolumn{1}{|r}{\mdline{744} 725}&\multicolumn{1}{|r}{\mdline{744} 43}&\multicolumn{1}{|r}{\mdline{744} 189}&\multicolumn{1}{l|}{\mdinline{width=6cm}{\mdline{744}}}\\
\multicolumn{1}{|l}{\mdline{745} \mdline{745}- mcertikos}&\multicolumn{1}{|r}{\mdline{745} 449}&\multicolumn{1}{|r}{\mdline{745} 207281}&\multicolumn{1}{|r}{\mdline{745} 5591}&\multicolumn{1}{|r}{\mdline{745} 324}&\multicolumn{1}{|r}{\mdline{745} 1757}&\multicolumn{1}{l|}{\mdinline{width=6cm}{\mdline{745}}}\\
\midrule
\multicolumn{1}{|l}{\mdline{747} flocq}&\multicolumn{1}{|r}{\mdline{747} 40}&\multicolumn{1}{|r}{\mdline{747} 67543}&\multicolumn{1}{|r}{\mdline{747} 1225}&\multicolumn{1}{|r}{\mdline{747} 21}&\multicolumn{1}{|r}{\mdline{747} 317}&\multicolumn{1}{l|}{\mdinline{width=6cm}{\mdline{747} Formalization of floating point}}\\
\multicolumn{1}{|l}{\mdline{748} kami}&\multicolumn{1}{|r}{\mdline{748} 101}&\multicolumn{1}{|r}{\mdline{748} 53910}&\multicolumn{1}{|r}{\mdline{748} 1937}&\multicolumn{1}{|r}{\mdline{748} 87}&\multicolumn{1}{|r}{\mdline{748} 976}&\multicolumn{1}{l|}{\mdinline{width=6cm}{\mdline{748} Framework to Support}}\\
\multicolumn{1}{|l}{\mdline{749}}&\multicolumn{1}{|r}{\mdline{749}}&\multicolumn{1}{|r}{\mdline{749}}&\multicolumn{1}{|r}{\mdline{749}}&\multicolumn{1}{|r}{\mdline{749}}&\multicolumn{1}{|r}{\mdline{749}}&\multicolumn{1}{l|}{\mdinline{width=6cm}{\mdline{749} Implementing and Verifying}}\\
\multicolumn{1}{|l}{\mdline{750}}&\multicolumn{1}{|r}{\mdline{750}}&\multicolumn{1}{|r}{\mdline{750}}&\multicolumn{1}{|r}{\mdline{750}}&\multicolumn{1}{|r}{\mdline{750}}&\multicolumn{1}{|r}{\mdline{750}}&\multicolumn{1}{l|}{\mdinline{width=6cm}{\mdline{750} Bluespec-style Hardware}}\\
\multicolumn{1}{|l}{\mdline{751}}&\multicolumn{1}{|r}{\mdline{751}}&\multicolumn{1}{|r}{\mdline{751}}&\multicolumn{1}{|r}{\mdline{751}}&\multicolumn{1}{|r}{\mdline{751}}&\multicolumn{1}{|r}{\mdline{751}}&\multicolumn{1}{l|}{\mdinline{width=6cm}{\mdline{751} Components}}\\
\multicolumn{1}{|l}{\mdline{752} math-comp}&\multicolumn{1}{|r}{\mdline{752} 92}&\multicolumn{1}{|r}{\mdline{752} 111079}&\multicolumn{1}{|r}{\mdline{752} 11379}&\multicolumn{1}{|r}{\mdline{752} 38}&\multicolumn{1}{|r}{\mdline{752} 3509}&\multicolumn{1}{l|}{\mdinline{width=6cm}{\mdline{752} Mathematical Components}}\\
\multicolumn{1}{|l}{\mdline{753}}&\multicolumn{1}{|r}{\mdline{753}}&\multicolumn{1}{|r}{\mdline{753}}&\multicolumn{1}{|r}{\mdline{753}}&\multicolumn{1}{|r}{\mdline{753}}&\multicolumn{1}{|r}{\mdline{753}}&\multicolumn{1}{l|}{\mdinline{width=6cm}{\mdline{753} Library}}\\
\multicolumn{1}{|l}{\mdline{754} qc}&\multicolumn{1}{|r}{\mdline{754} 9}&\multicolumn{1}{|r}{\mdline{754} 6239}&\multicolumn{1}{|r}{\mdline{754} 21}&\multicolumn{1}{|r}{\mdline{754} 28}&\multicolumn{1}{|r}{\mdline{754} 79}&\multicolumn{1}{l|}{\mdinline{width=6cm}{\mdline{754} SF\mdline{754} \mdline{754}- Quick Chick 1.0}}\\
\multicolumn{1}{|l}{\mdline{755} template-coq}&\multicolumn{1}{|r}{\mdline{755} 72}&\multicolumn{1}{|r}{\mdline{755} 11541}&\multicolumn{1}{|r}{\mdline{755} 180}&\multicolumn{1}{|r}{\mdline{755} 64}&\multicolumn{1}{|r}{\mdline{755} 472}&\multicolumn{1}{l|}{\mdinline{width=6cm}{\mdline{755} Quoting library for Coq}}\\
\multicolumn{1}{|l}{\mdline{756}}&\multicolumn{1}{|r}{\mdline{756}}&\multicolumn{1}{|r}{\mdline{756}}&\multicolumn{1}{|r}{\mdline{756}}&\multicolumn{1}{|r}{\mdline{756}}&\multicolumn{1}{|r}{\mdline{756}}&\multicolumn{1}{l|}{\mdinline{width=6cm}{\mdline{756} (frontend for CertiCoq)}}\\
\multicolumn{1}{|l}{\mdline{757} tlc}&\multicolumn{1}{|r}{\mdline{757} 58}&\multicolumn{1}{|r}{\mdline{757} 40300}&\multicolumn{1}{|r}{\mdline{757} 2496}&\multicolumn{1}{|r}{\mdline{757} 89}&\multicolumn{1}{|r}{\mdline{757} 552}&\multicolumn{1}{l|}{\mdinline{width=6cm}{\mdline{757} General purpose alternate to}}\\
\multicolumn{1}{|l}{\mdline{758}}&\multicolumn{1}{|r}{\mdline{758}}&\multicolumn{1}{|r}{\mdline{758}}&\multicolumn{1}{|r}{\mdline{758}}&\multicolumn{1}{|r}{\mdline{758}}&\multicolumn{1}{|r}{\mdline{758}}&\multicolumn{1}{l|}{\mdinline{width=6cm}{\mdline{758} Coq\mdline{758}'\mdline{758}s Standard Library}}\\
\multicolumn{1}{|l}{\mdline{759} vellvm}&\multicolumn{1}{|r}{\mdline{759} 55}&\multicolumn{1}{|r}{\mdline{759} 24006}&\multicolumn{1}{|r}{\mdline{759} 1085}&\multicolumn{1}{|r}{\mdline{759} 122}&\multicolumn{1}{|r}{\mdline{759} 635}&\multicolumn{1}{l|}{\mdinline{width=6cm}{\mdline{759} Verifying LLVM}}\\
\multicolumn{1}{|l}{\mdline{760} verified-ifc}&\multicolumn{1}{|r}{\mdline{760} 58}&\multicolumn{1}{|r}{\mdline{760} 31527}&\multicolumn{1}{|r}{\mdline{760} 849}&\multicolumn{1}{|r}{\mdline{760} 123}&\multicolumn{1}{|r}{\mdline{760} 395}&\multicolumn{1}{l|}{\mdinline{width=6cm}{\mdline{760} A Verified Information-Flow}}\\
\multicolumn{1}{|l}{\mdline{761}}&\multicolumn{1}{|r}{\mdline{761}}&\multicolumn{1}{|r}{\mdline{761}}&\multicolumn{1}{|r}{\mdline{761}}&\multicolumn{1}{|r}{\mdline{761}}&\multicolumn{1}{|r}{\mdline{761}}&\multicolumn{1}{l|}{\mdinline{width=6cm}{\mdline{761} Architecture}}\\
\multicolumn{1}{|l}{\mdline{762} vst}&\multicolumn{1}{|r}{\mdline{762} 508}&\multicolumn{1}{|r}{\mdline{762} 314515}&\multicolumn{1}{|r}{\mdline{762} 11812}&\multicolumn{1}{|r}{\mdline{762} 481}&\multicolumn{1}{|r}{\mdline{762} 7882}&\multicolumn{1}{l|}{\mdinline{width=6cm}{\mdline{762} Verified Software Toolchain}}\\
\multicolumn{1}{|l}{\mdline{763} why2}&\multicolumn{1}{|r}{\mdline{763} 98}&\multicolumn{1}{|r}{\mdline{763} 40045}&\multicolumn{1}{|r}{\mdline{763} 260}&\multicolumn{1}{|r}{\mdline{763} 67}&\multicolumn{1}{|r}{\mdline{763} 1787}&\multicolumn{1}{l|}{\mdinline{width=6cm}{\mdline{763} Why2 verification tool}}\\
\multicolumn{1}{|l}{\mdline{764} why3 1.0}&\multicolumn{1}{|r}{\mdline{764} 189}&\multicolumn{1}{|r}{\mdline{764} 44304}&\multicolumn{1}{|r}{\mdline{764} 968}&\multicolumn{1}{|r}{\mdline{764} 365}&\multicolumn{1}{|r}{\mdline{764} 1030}&\multicolumn{1}{l|}{\mdinline{padding-bottom=4pt,width=6cm}{\mdline{764} Why3 verification tool}}\\
\midrule[\dimpx{2}]
\end{tabular}\end{mdtabular}

%mdk-data-line={768;out/FormalReview-bib.bbl.mdk:1}
%mdk-data-line={768;out/FormalReview-bib.bbl.mdk:2}
\mdsetrefname{References}%mdk
{\mdsupressbiblabel\mdbibindent{2}\begin{thebibliography}{222}%mdk
\label{sec-bibliography}%mdk

%mdk-data-line={FormalReview.bib:771}
\bibitem{absint_compcert_nodate}\mdbibitemlabel{}Absint. (n.d.). CompCert - Publications. Retrieved January 31, 2019, from \href{http://compcert.inria.fr/publi.html}{{\ttfamily http://\hspace{0pt}compcert.\hspace{0pt}inria.\hspace{0pt}fr/\hspace{0pt}publi.\hspace{0pt}html}}\label{absint_compcert_nodate}%mdk%mdk

%mdk-data-line={FormalReview.bib:2343}
\bibitem{adams_common_2015}\mdbibitemlabel{}Adams, M.~(2015). The Common HOL Platform. \emph{Electronic Proceedings in Theoretical Computer Science}, \emph{186}, 42–56. https://doi.org/\href{https://dx.doi.org/10.4204/EPTCS.186.6}{10.4204/EPTCS.186.6}\label{adams_common_2015}%mdk%mdk

%mdk-data-line={FormalReview.bib:1362}
\bibitem{adewale_implementing_nodate}\mdbibitemlabel{}Adewale, O.~(n.d.). Implementing a high-performance key-value store using a trie of B+-Trees with cursors \textbar{} Computer Science Department at Princeton University. Retrieved February 1, 2019, from \href{https://www.cs.princeton.edu/research/techreps/TR-004-18}{{\ttfamily https://\hspace{0pt}www.\hspace{0pt}cs.\hspace{0pt}princeton.\hspace{0pt}edu/\hspace{0pt}research/\hspace{0pt}techreps/\hspace{0pt}TR-\hspace{0pt}004-\hspace{0pt}18}}\label{adewale_implementing_nodate}%mdk%mdk

%mdk-data-line={FormalReview.bib:1998}
\bibitem{ahman_recalling_2017}\mdbibitemlabel{}Ahman, D., Fournet, C., Hriţcu, C., Maillard, K., Rastogi, A., \& Swamy, N.~(2017). Recalling a Witness: Foundations and Applications of Monotonic State. \emph{Proc. ACM Program. Lang.}, \emph{2}, 65:1–65:30. https://doi.org/\href{https://dx.doi.org/10.1145/3158153}{10.1145/3158153}\label{ahman_recalling_2017}%mdk%mdk

%mdk-data-line={FormalReview.bib:1971}
\bibitem{ahman_dijkstra_2017}\mdbibitemlabel{}Ahman, D., Hriţcu, C., Maillard, K., Martínez, G., Plotkin, G., Protzenko, J., … Swamy, N.~(2017). Dijkstra Monads for Free. In \emph{Proceedings of the 44th ACM SIGPLAN Symposium on Principles of Programming Languages} (pp. 515–529). New York, NY, USA: ACM.~https://doi.org/\href{https://dx.doi.org/10.1145/3009837.3009878}{10.1145/3009837.3009878}\label{ahman_dijkstra_2017}%mdk%mdk

%mdk-data-line={FormalReview.bib:1044}
\bibitem{ahrendt_deductive_nodate}\mdbibitemlabel{}Ahrendt, W.~(n.d.). Deductive Software Verification – The KeY BookFrom Theory to Practice – The KeY Project. Retrieved January 31, 2019, from \href{https://www.key-project.org/thebook2/}{{\ttfamily https://\hspace{0pt}www.\hspace{0pt}key-\hspace{0pt}project.\hspace{0pt}org/\hspace{0pt}thebook2/\hspace{0pt}}}\label{ahrendt_deductive_nodate}%mdk%mdk

%mdk-data-line={FormalReview.bib:915}
\bibitem{hutchison_verifying_2007}\mdbibitemlabel{}Ahrendt, W., Beckert, B., Hähnle, R., Rümmer, P., \& Schmitt, P.~H.~(2007). Verifying Object-Oriented Programs with KeY: A Tutorial. In F.~S.~de Boer, M.~M.~Bonsangue, S.~Graf, \& W.-P.~de Roever (Eds.), \emph{Formal Methods for Components and Objects} (Vol. 4709, pp. 70–101). Berlin, Heidelberg: Springer Berlin Heidelberg. https://doi.org/\href{https://dx.doi.org/10.1007/978-3-540-74792-5_4}{10.1007/978-3-540-74792-5\_4}\label{hutchison_verifying_2007}%mdk%mdk

%mdk-data-line={FormalReview.bib:207}
\bibitem{altenkirch_quotient_2018}\mdbibitemlabel{}Altenkirch, T., Capriotti, P., Dijkstra, G., Kraus, N., \& Forsberg, F.~N.~(2018). Quotient inductive-inductive types. \emph{arXiv:1612.02346 {}[cs]}, \emph{10803}, 293–310. https://doi.org/\href{https://dx.doi.org/10.1007/978-3-319-89366-2_16}{10.1007/978-3-319-89366-2\_16}\label{altenkirch_quotient_2018}%mdk%mdk

%mdk-data-line={FormalReview.bib:1190}
\bibitem{amin_computing_2016}\mdbibitemlabel{}Amin, N., Leino, R., \& Rompf, T.~(2016). Computing with an SMT Solver, \emph{8570}. Retrieved from \href{https://www.microsoft.com/en-us/research/publication/computing-smt-solver/}{{\ttfamily https://\hspace{0pt}www.\hspace{0pt}microsoft.\hspace{0pt}com/\hspace{0pt}en-\hspace{0pt}us/\hspace{0pt}research/\hspace{0pt}publication/\hspace{0pt}computing-\hspace{0pt}smt-\hspace{0pt}solver/\hspace{0pt}}}\label{amin_computing_2016}%mdk%mdk

%mdk-data-line={FormalReview.bib:3051}
\bibitem{amorim_verified_2013}\mdbibitemlabel{}Amorim, A.~A.~de, Collins, N., DeHon, A., Demange, D., Hritcu, C., Pichardie, D., … Tolmach, A.~(2013). \emph{A Verified Information-Flow Architecture (Long version)}.\label{amorim_verified_2013}%mdk%mdk

%mdk-data-line={FormalReview.bib:1691}
\bibitem{anand_towards_nodate}\mdbibitemlabel{}Anand, A., Boulier, S., Cohen, C., Sozeau, M., \& Tabareau, N.~(n.d.). Towards Certified Meta-Programming with Typed Template-Coq \textbar{} SpringerLink. Retrieved February 1, 2019, from \href{https://link.springer.com/chapter/10.1007\%252F978-3-319-94821-8_2}{{\ttfamily https://\hspace{0pt}link.\hspace{0pt}springer.\hspace{0pt}com/\hspace{0pt}chapter/\hspace{0pt}10.\hspace{0pt}1007\hspace{0pt}\%2F978-\hspace{0pt}3-\hspace{0pt}319-\hspace{0pt}94821-\hspace{0pt}8\_\hspace{0pt}2}}\label{anand_towards_nodate}%mdk%mdk

%mdk-data-line={FormalReview.bib:1337}
\bibitem{anand_typed_nodate}\mdbibitemlabel{}Anand, A., Tabareau, S.~B.~N., \& Sozeau, M.~(n.d.). Typed Template Coq, 2.\label{anand_typed_nodate}%mdk%mdk

%mdk-data-line={FormalReview.bib:3059}
\bibitem{andrew_oracle_2008}\mdbibitemlabel{}Andrew, A.~(2008). \emph{Oracle Semantics Aquinas Hobor}.\label{andrew_oracle_2008}%mdk%mdk

%mdk-data-line={FormalReview.bib:1612}
\bibitem{appel_andrew_w._position_2017}\mdbibitemlabel{}Appel Andrew W., Beringer Lennart, Chlipala Adam, Pierce Benjamin C., Shao Zhong, Weirich Stephanie, \& Zdancewic Steve. (2017). Position paper: the science of deep specification. \emph{Philosophical Transactions of the Royal Society A: Mathematical, Physical and Engineering Sciences}, \emph{375}(2104), 20160331. https://doi.org/\href{https://dx.doi.org/10.1098/rsta.2016.0331}{10.1098/rsta.2016.0331}\label{appel_andrew_w._position_2017}%mdk%mdk

%mdk-data-line={FormalReview.bib:331}
\bibitem{jouannaud_verismall:_2011}\mdbibitemlabel{}Appel, A.~W.~(2011). VeriSmall: Verified Smallfoot Shape Analysis. In J.-P.~Jouannaud \& Z.~Shao (Eds.), \emph{Certified Programs and Proofs} (Vol. 7086, pp. 231–246). Berlin, Heidelberg: Springer Berlin Heidelberg. https://doi.org/\href{https://dx.doi.org/10.1007/978-3-642-25379-9_18}{10.1007/978-3-642-25379-9\_18}\label{jouannaud_verismall:_2011}%mdk%mdk

%mdk-data-line={FormalReview.bib:367}
\bibitem{appel_verified_2012}\mdbibitemlabel{}Appel, A.~W.~(2012). Verified Software Toolchain. In \emph{Proceedings of the 4th International Conference on NASA Formal Methods} (pp. 2–2). Berlin, Heidelberg: Springer-Verlag. https://doi.org/\href{https://dx.doi.org/10.1007/978-3-642-28891-3_2}{10.1007/978-3-642-28891-3\_2}\label{appel_verified_2012}%mdk%mdk

%mdk-data-line={FormalReview.bib:313}
\bibitem{appel_verification_2015}\mdbibitemlabel{}Appel, A.~W.~(2015). Verification of a Cryptographic Primitive: SHA-256. \emph{ACM Trans. Program. Lang. Syst.}, \emph{37}(2), 7:1–7:31. https://doi.org/\href{https://dx.doi.org/10.1145/2701415}{10.1145/2701415}\label{appel_verification_2015}%mdk%mdk

%mdk-data-line={FormalReview.bib:1381}
\bibitem{appel_deepspecdb_2019}\mdbibitemlabel{}Appel, A.~W.~(2019). \emph{DeepSpecDB - github}. PrincetonUniversity. Retrieved from \href{https://github.com/PrincetonUniversity/DeepSpecDB}{{\ttfamily https://\hspace{0pt}github.\hspace{0pt}com/\hspace{0pt}PrincetonUniversity/\hspace{0pt}DeepSpecDB}}\label{appel_deepspecdb_2019}%mdk%mdk

%mdk-data-line={FormalReview.bib:1354}
\bibitem{appel_certicoq:_nodate}\mdbibitemlabel{}Appel, A.~W.~(n.d.). CertiCoq: A verified compiler for Coq - POPL 2017. Retrieved February 1, 2019, from \href{https://popl17.sigplan.org/event/main-certicoq-a-verified-compiler-for-coq}{{\ttfamily https://\hspace{0pt}popl17.\hspace{0pt}sigplan.\hspace{0pt}org/\hspace{0pt}event/\hspace{0pt}main-\hspace{0pt}certicoq-\hspace{0pt}a-\hspace{0pt}verified-\hspace{0pt}compiler-\hspace{0pt}for-\hspace{0pt}coq}}\label{appel_certicoq:_nodate}%mdk%mdk

%mdk-data-line={FormalReview.bib:299}
\bibitem{appel_verifiabble_2014}\mdbibitemlabel{}Appel, A.~W., Dockins, R., Hobor, A., Beringer, L., Dodds, J., Stewart, G., … Leroy, X.~(2014). \emph{Verifiabble C, Version 2.2}. Cambridge: Cambridge University Press. https://doi.org/\href{https://dx.doi.org/10.1017/CBO9781107256552}{10.1017/CBO9781107256552}\label{appel_verifiabble_2014}%mdk%mdk

%mdk-data-line={FormalReview.bib:2543}
\bibitem{arias_jscoq:_2017}\mdbibitemlabel{}Arias, E.~J.~G., Pin, B., \& Jouvelot, P.~(2017). jsCoq: Towards Hybrid Theorem Proving Interfaces. \emph{Electronic Proceedings in Theoretical Computer Science}, \emph{239}, 15–27. https://doi.org/\href{https://dx.doi.org/10.4204/EPTCS.239.2}{10.4204/EPTCS.239.2}\label{arias_jscoq:_2017}%mdk%mdk

%mdk-data-line={FormalReview.bib:3033}
\bibitem{azevedo_de_amorim_verified_2014}\mdbibitemlabel{}Azevedo de Amorim, A., Collins, N., DeHon, A., Demange, D., Hriţcu, C., Pichardie, D., … Tolmach, A.~(2014). A Verified Information-flow Architecture. In \emph{Proceedings of the 41st ACM SIGPLAN-SIGACT Symposium on Principles of Programming Languages} (pp. 165–178). New York, NY, USA: ACM.~https://doi.org/\href{https://dx.doi.org/10.1145/2535838.2535839}{10.1145/2535838.2535839}\label{azevedo_de_amorim_verified_2014}%mdk%mdk

%mdk-data-line={FormalReview.bib:1372}
\bibitem{barriere_vst_nodate}\mdbibitemlabel{}Barriere, A., \& Appel, A.~(n.d.). VST Verification of B+Trees with Cursors, 19.\label{barriere_vst_nodate}%mdk%mdk

%mdk-data-line={FormalReview.bib:1751}
\bibitem{bate_fundamentals_1971}\mdbibitemlabel{}Bate, R.~R., Mueller, D.~D., \& White, J.~E.~(1971). \emph{Fundamentals of astrodynamics}. New York: Dover Publications.\label{bate_fundamentals_1971}%mdk%mdk

%mdk-data-line={FormalReview.bib:1597}
\bibitem{batty_mark_compositional_2017}\mdbibitemlabel{}Batty Mark. (2017). Compositional relaxed concurrency. \emph{Philosophical Transactions of the Royal Society A: Mathematical, Physical and Engineering Sciences}, \emph{375}(2104), 20150406. https://doi.org/\href{https://dx.doi.org/10.1098/rsta.2015.0406}{10.1098/rsta.2015.0406}\label{batty_mark_compositional_2017}%mdk%mdk

%mdk-data-line={FormalReview.bib:1011}
\bibitem{beckert_verification_2006}\mdbibitemlabel{}Beckert, B., Hähnle, R., \& Schmitt, P.~H.~(Eds.). (2006). \emph{Verification of Object-Oriented Software. The KeY Approach} (Vol. 4334). Berlin, Heidelberg: Springer Berlin Heidelberg. https://doi.org/\href{https://dx.doi.org/10.1007/978-3-540-69061-0}{10.1007/978-3-540-69061-0}\label{beckert_verification_2006}%mdk%mdk

%mdk-data-line={FormalReview.bib:446}
\bibitem{bedford_coqatoo:_2017}\mdbibitemlabel{}Bedford, A.~(2017). Coqatoo: Generating Natural Language Versions of Coq Proofs. \emph{arXiv:1712.03894 {}[cs]}. Retrieved from arXiv:\href{http://arxiv.org/abs/1712.03894}{1712.03894}\label{bedford_coqatoo:_2017}%mdk%mdk

%mdk-data-line={FormalReview.bib:438}
\bibitem{bedford_coqatoo:_nodate}\mdbibitemlabel{}Bedford, A.~(n.d.). Coqatoo: Generating Natural Language Versions of Coq Proofs - Slides, 16.\label{bedford_coqatoo:_nodate}%mdk%mdk

%mdk-data-line={FormalReview.bib:3067}
\bibitem{berdine_smallfoot:_2006}\mdbibitemlabel{}Berdine, J., Calcagno, C., \& O’Hearn, P.~W.~(2006). Smallfoot: Modular Automatic Assertion Checking with Separation Logic. In F.~S.~de Boer, M.~M.~Bonsangue, S.~Graf, \& W.-P.~de Roever (Eds.), \emph{Formal Methods for Components and Objects} (pp. 115–137). Springer Berlin Heidelberg.\label{berdine_smallfoot:_2006}%mdk%mdk

%mdk-data-line={FormalReview.bib:279}
\bibitem{hutchison_verified_2014}\mdbibitemlabel{}Beringer, L., Stewart, G., Dockins, R., \& Appel, A.~W.~(2014). Verified Compilation for Shared-Memory C.~In Z.~Shao (Ed.), \emph{Programming Languages and Systems} (Vol. 8410, pp. 107–127). Berlin, Heidelberg: Springer Berlin Heidelberg. https://doi.org/\href{https://dx.doi.org/10.1007/978-3-642-54833-8_7,}{10.1007/978-3-642-54833-8\_7,}\label{hutchison_verified_2014}%mdk%mdk

%mdk-data-line={FormalReview.bib:827}
\bibitem{bertot_yves_nodate}\mdbibitemlabel{}Bertot, Y.~(n.d.). Yves Bertot. Retrieved January 31, 2019, from \href{http://www-sop.inria.fr/members/Yves.Bertot/index.html}{{\ttfamily http://\hspace{0pt}www-\hspace{0pt}sop.\hspace{0pt}inria.\hspace{0pt}fr/\hspace{0pt}members/\hspace{0pt}Yves.\hspace{0pt}Bertot/\hspace{0pt}index.\hspace{0pt}html}}\label{bertot_yves_nodate}%mdk%mdk

%mdk-data-line={FormalReview.bib:804}
\bibitem{bertot_interactive_2004}\mdbibitemlabel{}Bertot, Y., \& Castéran, P.~(2004). \emph{Interactive theorem proving and program development: Coq’Art: the calculus of inductive constructions}. Berlin ; New York: Springer. Retrieved from \href{http://www.labri.fr/perso/casteran/CoqArt/index.html}{{\ttfamily http://\hspace{0pt}www.\hspace{0pt}labri.\hspace{0pt}fr/\hspace{0pt}perso/\hspace{0pt}casteran/\hspace{0pt}CoqArt/\hspace{0pt}index.\hspace{0pt}html}}\label{bertot_interactive_2004}%mdk%mdk

%mdk-data-line={FormalReview.bib:2425}
\bibitem{birkedal_iris_nodate}\mdbibitemlabel{}Birkedal, L., \& Bizjak, A.~(n.d.). Iris Tutorial. Retrieved February 1, 2019, from \href{https://iris-project.org/tutorial-material.html}{{\ttfamily https://\hspace{0pt}iris-\hspace{0pt}project.\hspace{0pt}org/\hspace{0pt}tutorial-\hspace{0pt}material.\hspace{0pt}html}}\label{birkedal_iris_nodate}%mdk%mdk

%mdk-data-line={FormalReview.bib:1849}
\bibitem{blanchard_concurrent_2017}\mdbibitemlabel{}Blanchard, A., Loulergue, F., \& Kosmatov, N.~(2017). From Concurrent Programs to Simulating Sequential Programs: Correctness of a Transformation. \emph{Electronic Proceedings in Theoretical Computer Science}, \emph{253}, 109–123. https://doi.org/\href{https://dx.doi.org/10.4204/EPTCS.253.9}{10.4204/EPTCS.253.9}\label{blanchard_concurrent_2017}%mdk%mdk

%mdk-data-line={FormalReview.bib:1901}
\bibitem{blatter_static_2018}\mdbibitemlabel{}Blatter, L., Kosmatov, N., Le Gall, P., Prevosto, V., \& Petiot, G.~(2018). Static and Dynamic Verification of Relational Properties on Self-composed C Code. In C.~Dubois \& B.~Wolff (Eds.), \emph{Tests and Proofs} (pp. 44–62). Springer International Publishing.\label{blatter_static_2018}%mdk%mdk

%mdk-data-line={FormalReview.bib:1027}
\bibitem{bohrer_veriphy:_2018}\mdbibitemlabel{}Bohrer, B., Tan, Y.~K., Mitsch, S., Myreen, M.~O., \& Platzer, A.~(2018). VeriPhy: verified controller executables from verified cyber-physical system models. In \emph{Proceedings of the 39th ACM SIGPLAN Conference on Programming Language Design and Implementation~- PLDI 2018} (pp. 617–630). Philadelphia, PA, USA: ACM Press. https://doi.org/\href{https://dx.doi.org/10.1145/3192366.3192406}{10.1145/3192366.3192406}\label{bohrer_veriphy:_2018}%mdk%mdk

%mdk-data-line={FormalReview.bib:2910}
\bibitem{boldo_round-off_2017}\mdbibitemlabel{}Boldo, S., Faissole, F., \& Chapoutot, A.~(2017). Round-off Error Analysis of Explicit One-Step Numerical Integration Methods. In \emph{24th IEEE Symposium on Computer Arithmetic}. London, United Kingdom. https://doi.org/\href{https://dx.doi.org/10.1109/ARITH.2017.22}{10.1109/ARITH.2017.22}\label{boldo_round-off_2017}%mdk%mdk

%mdk-data-line={FormalReview.bib:2924}
\bibitem{boldo_round-off_2018}\mdbibitemlabel{}Boldo, S., Faissole, F., \& Chapoutot, A.~(2018). \emph{Round-off error and exceptional behavior analysis of explicit Runge-Kutta methods}. Retrieved from \href{https://hal.archives-ouvertes.fr/hal-01883843}{{\ttfamily https://\hspace{0pt}hal.\hspace{0pt}archives-\hspace{0pt}ouvertes.\hspace{0pt}fr/\hspace{0pt}hal-\hspace{0pt}01883843}}\label{boldo_round-off_2018}%mdk%mdk

%mdk-data-line={FormalReview.bib:2870}
\bibitem{hutchison_improving_2012}\mdbibitemlabel{}Boldo, S., Lelay, C., \& Melquiond, G.~(2012). Improving Real Analysis in Coq: A User-Friendly Approach to Integrals and Derivatives. In C.~Hawblitzel \& D.~Miller (Eds.), \emph{Certified Programs and Proofs} (Vol. 7679, pp. 289–304). Berlin, Heidelberg: Springer Berlin Heidelberg. https://doi.org/\href{https://dx.doi.org/10.1007/978-3-642-35308-6_22}{10.1007/978-3-642-35308-6\_22}\label{hutchison_improving_2012}%mdk%mdk

%mdk-data-line={FormalReview.bib:2953}
\bibitem{boldo_coquelicot:_2013}\mdbibitemlabel{}Boldo, S., Lelay, C., \& Melquiond, G.~(2013). Coquelicot: A User-Friendly Library of Real Analysis for Coq. Retrieved from \href{https://hal.inria.fr/hal-00860648/document}{{\ttfamily https://\hspace{0pt}hal.\hspace{0pt}inria.\hspace{0pt}fr/\hspace{0pt}hal-\hspace{0pt}00860648/\hspace{0pt}document}}\label{boldo_coquelicot:_2013}%mdk%mdk

%mdk-data-line={FormalReview.bib:2965}
\bibitem{boldo_formalization_2016}\mdbibitemlabel{}Boldo, S., Lelay, C., \& Melquiond, G.~(2016). Formalization of Real Analysis: A Survey of Proof Assistants and Libraries. \emph{Mathematical Structures in Computer Science}, \emph{26}(7), 1196–1233. https://doi.org/\href{https://dx.doi.org/10.1017/S0960129514000437}{10.1017/S0960129514000437}\label{boldo_formalization_2016}%mdk%mdk

%mdk-data-line={FormalReview.bib:476}
\bibitem{boulier_next_2017}\mdbibitemlabel{}Boulier, S., Pédrot, P.-M., \& Tabareau, N.~(2017). The next 700 syntactical models of type theory (pp. 182–194). https://doi.org/\href{https://dx.doi.org/10.1145/3018610.3018620}{10.1145/3018610.3018620}\label{boulier_next_2017}%mdk%mdk

%mdk-data-line={FormalReview.bib:490}
\bibitem{bowman_j1:_nodate}\mdbibitemlabel{}Bowman, J.~(n.d.). \emph{J1: a small Forth CPU Core for FPGAs}.\label{bowman_j1:_nodate}%mdk%mdk

%mdk-data-line={FormalReview.bib:1818}
\bibitem{brahmi_formalise_2018}\mdbibitemlabel{}Brahmi, A., Delmas, D., Essoussi, M.~H., Randimbivololona, F., Atki, A., \& Marie, T.~(2018). Formalise to automate: deployment of a safe and cost-efficient process for avionics software. In \emph{9th European Congress on Embedded Real Time Software and Systems (ERTS 2018)}. Toulouse, France. Retrieved from \href{https://hal.archives-ouvertes.fr/hal-01708332}{{\ttfamily https://\hspace{0pt}hal.\hspace{0pt}archives-\hspace{0pt}ouvertes.\hspace{0pt}fr/\hspace{0pt}hal-\hspace{0pt}01708332}}\label{brahmi_formalise_2018}%mdk%mdk

%mdk-data-line={FormalReview.bib:1832}
\bibitem{brahmi_formalise_nodate}\mdbibitemlabel{}Brahmi, A., Delmas, D., Essoussi, M.~H., Randimbivololona, F., Informatics, C., Nauzere, L., … Marie, T.~(n.d.). Formalise to automate: deployment of a safe and cost-efficient process for avionics software -Extended, 17.\label{brahmi_formalise_nodate}%mdk%mdk

%mdk-data-line={FormalReview.bib:2527}
\bibitem{brockschmidt_t2:_2016}\mdbibitemlabel{}Brockschmidt, M., Cook, B., Ishtiaq, S., Khlaaf, H., \& Piterman, N.~(2016). T2: Temporal Property Verification. In M.~Chechik \& J.-F.~Raskin (Eds.), \emph{Tools and Algorithms for the Construction and Analysis of Systems} (pp. 387–393). Springer Berlin Heidelberg.\label{brockschmidt_t2:_2016}%mdk%mdk

%mdk-data-line={FormalReview.bib:119}
\bibitem{brookes_semantics_2007}\mdbibitemlabel{}Brookes, S.~(2007). A semantics for concurrent separation logic. \emph{Theoretical Computer Science}, \emph{375}(1), 227–270. https://doi.org/\href{https://dx.doi.org/10.1016/j.tcs.2006.12.034}{10.1016/j.tcs.2006.12.034}\label{brookes_semantics_2007}%mdk%mdk

%mdk-data-line={FormalReview.bib:106}
\bibitem{brookes_concurrent_2016}\mdbibitemlabel{}Brookes, S., \& O’Hearn, P.~W.~(2016). Concurrent Separation Logic. \emph{ACM SIGLOG News}, \emph{3}(3), 47–65. https://doi.org/\href{https://dx.doi.org/10.1145/2984450.2984457}{10.1145/2984450.2984457}\label{brookes_concurrent_2016}%mdk%mdk

%mdk-data-line={FormalReview.bib:3119}
\bibitem{calcagno_moving_nodate}\mdbibitemlabel{}Calcagno, C., Distefano, D., Dubreil, J., \& O’Hearn, P.~(n.d.). Moving Fast with Software Verification.Facebook Research. Retrieved February 1, 2019, from \href{https://research.fb.com/publications/moving-fast-with-software-verification}{{\ttfamily https://\hspace{0pt}research.\hspace{0pt}fb.\hspace{0pt}com/\hspace{0pt}publications/\hspace{0pt}moving-\hspace{0pt}fast-\hspace{0pt}with-\hspace{0pt}software-\hspace{0pt}verification}}\label{calcagno_moving_nodate}%mdk%mdk

%mdk-data-line={FormalReview.bib:2}
\bibitem{calcagno_compositional_2011}\mdbibitemlabel{}Calcagno, C., Distefano, D., O’Hearn, P.~W., \& Yang, H.~(2011). Compositional Shape Analysis by Means of Bi-Abduction. \emph{Journal of the ACM}, \emph{58}(6), 1–66. https://doi.org/\href{https://dx.doi.org/10.1145/2049697.2049700}{10.1145/2049697.2049700}\label{calcagno_compositional_2011}%mdk%mdk

%mdk-data-line={FormalReview.bib:261}
\bibitem{cao_vst-floyd:_2018}\mdbibitemlabel{}Cao, Q., Beringer, L., Gruetter, S., Dodds, J., \& Appel, A.~W.~(2018). VST-Floyd: A Separation Logic Tool to Verify Correctness of C Programs. \emph{J.~Autom. Reason.}, \emph{61}(1), 367–422. https://doi.org/\href{https://dx.doi.org/10.1007/s10817-018-9457-5}{10.1007/s10817-018-9457-5}\label{cao_vst-floyd:_2018}%mdk%mdk

%mdk-data-line={FormalReview.bib:819}
\bibitem{casteran_pierre_nodate}\mdbibitemlabel{}Castéran, P.~(n.d.). Pierre Castéran’s Home page. Retrieved January 31, 2019, from \href{http://www.labri.fr/perso/casteran/index.html}{{\ttfamily http://\hspace{0pt}www.\hspace{0pt}labri.\hspace{0pt}fr/\hspace{0pt}perso/\hspace{0pt}casteran/\hspace{0pt}index.\hspace{0pt}html}}\label{casteran_pierre_nodate}%mdk%mdk

%mdk-data-line={FormalReview.bib:649}
\bibitem{chargueraud_characteristic_2010}\mdbibitemlabel{}Charguéraud, A.~(2010a). \emph{Characteristic Formulae for Mechanized Program Verification} (phdthesis). UNIVERSITÉ PARIS.DIDEROT, Paris, France.\label{chargueraud_characteristic_2010}%mdk%mdk

%mdk-data-line={FormalReview.bib:613}
\bibitem{chargueraud_program_2010}\mdbibitemlabel{}Charguéraud, A.~(2010b). Program Verification Through Characteristic Formulae. In \emph{Proceedings of the 15th ACM SIGPLAN International Conference on Functional Programming} (pp. 321–332). New York, NY, USA: ACM.~https://doi.org/\href{https://dx.doi.org/10.1145/1863543.1863590}{10.1145/1863543.1863590}\label{chargueraud_program_2010}%mdk%mdk

%mdk-data-line={FormalReview.bib:631}
\bibitem{chargueraud_characteristic_2011}\mdbibitemlabel{}Charguéraud, A.~(2011). Characteristic Formulae for the Verification of Imperative Programs. In \emph{Proceedings of the 16th ACM SIGPLAN International Conference on Functional Programming} (pp. 418–430). New York, NY, USA: ACM.~https://doi.org/\href{https://dx.doi.org/10.1145/2034773.2034828}{10.1145/2034773.2034828}\label{chargueraud_characteristic_2011}%mdk%mdk

%mdk-data-line={FormalReview.bib:1392}
\bibitem{chen_project_nodate}\mdbibitemlabel{}Chen, Y.~(n.d.). Project Report on DeepSpecDB, 35.\label{chen_project_nodate}%mdk%mdk

%mdk-data-line={FormalReview.bib:698}
\bibitem{chlipala_certified_2013}\mdbibitemlabel{}Chlipala, A.~(2013). \emph{Certified programming with dependent types: a pragmatic introduction to the Coq proof assistant}. Cambridge, MA: The MIT Press. Retrieved from \href{http://adam.chlipala.net/cpdt/}{{\ttfamily http://\hspace{0pt}adam.\hspace{0pt}chlipala.\hspace{0pt}net/\hspace{0pt}cpdt/\hspace{0pt}}}\label{chlipala_certified_2013}%mdk%mdk

%mdk-data-line={FormalReview.bib:719}
\bibitem{chlipala_formal_2019}\mdbibitemlabel{}Chlipala, A.~(2019). \emph{Formal Reasoning About Programs - Github}. Retrieved from \href{https://github.com/achlipala/frap}{{\ttfamily https://\hspace{0pt}github.\hspace{0pt}com/\hspace{0pt}achlipala/\hspace{0pt}frap}}\label{chlipala_formal_2019}%mdk%mdk

%mdk-data-line={FormalReview.bib:683}
\bibitem{chlipala_introduction_nodate}\mdbibitemlabel{}Chlipala, A.~(n.d.-a). An Introduction to Programming and Proving with Dependent Types in Coq. \emph{Journal of Formalized Reasoning}, \emph{3}(2), 93.\label{chlipala_introduction_nodate}%mdk%mdk

%mdk-data-line={FormalReview.bib:711}
\bibitem{chlipala_certied_nodate}\mdbibitemlabel{}Chlipala, A.~(n.d.-b). Certified Programming with Dependent Types, 369.\label{chlipala_certied_nodate}%mdk%mdk

%mdk-data-line={FormalReview.bib:1437}
\bibitem{chlipala_end_nodate}\mdbibitemlabel{}Chlipala, A., Delaware, B., Duchovni, S., Gross, J., Pit-Claudel, C., Suriyakarn, S., … ye, K.~(n.d.). THE END OF HISTORY? USING A PROOF ASSISTANT TO REPLACE LANGUAGE DESIGN WITH LIBRARY DESIGN.~Retrieved February 1, 2019, from \href{https://snapl.org/2017/abstracts/Chlipala.html}{{\ttfamily https://\hspace{0pt}snapl.\hspace{0pt}org/\hspace{0pt}2017/\hspace{0pt}abstracts/\hspace{0pt}Chlipala.\hspace{0pt}html}}\label{chlipala_end_nodate}%mdk%mdk

%mdk-data-line={FormalReview.bib:1528}
\bibitem{choi_kami:_2017}\mdbibitemlabel{}Choi, J., Vijayaraghavan, M., Sherman, B., Chlipala, A., \& Arvind. (2017). Kami: A Platform for High-level Parametric Hardware Specification and Its Modular Verification. \emph{Proc. ACM Program. Lang.}, \emph{1}, 24:1–24:30. https://doi.org/\href{https://dx.doi.org/10.1145/3110268}{10.1145/3110268}\label{choi_kami:_2017}%mdk%mdk

%mdk-data-line={FormalReview.bib:1162}
\bibitem{christakis_collaborative_2012}\mdbibitemlabel{}Christakis, M., Müller, P., \& Wüstholz, V.~(2012). Collaborative Verification and Testing with Explicit Assumptions. In D.~Giannakopoulou \& D.~Méry (Eds.), \emph{FM 2012: Formal Methods} (pp. 132–146). Springer Berlin Heidelberg.\label{christakis_collaborative_2012}%mdk%mdk

%mdk-data-line={FormalReview.bib:177}
\bibitem{conchon_alt-ergo_2018}\mdbibitemlabel{}Conchon, S., Coquereau, A., Iguernlala, M., \& Mebsout, A.~(2018). Alt-Ergo 2.2. In \emph{SMT Workshop: International Workshop on Satisfiability Modulo Theories}. Oxford, United Kingdom. Retrieved from \href{https://hal.inria.fr/hal-01960203}{{\ttfamily https://\hspace{0pt}hal.\hspace{0pt}inria.\hspace{0pt}fr/\hspace{0pt}hal-\hspace{0pt}01960203}}\label{conchon_alt-ergo_2018}%mdk%mdk

%mdk-data-line={FormalReview.bib:190}
\bibitem{conchon_increasing_2016}\mdbibitemlabel{}Conchon, S., \& Iguernlala, M.~(2016). Increasing Proofs Automation Rate of Atelier-B Thanks to Alt-Ergo. In T.~Lecomte, R.~Pinger, \& A.~Romanovsky (Eds.), \emph{Reliability, Safety, and Security of Railway Systems. Modelling, Analysis, Verification, and Certification} (pp. 243–253). Springer International Publishing.\label{conchon_increasing_2016}%mdk%mdk

%mdk-data-line={FormalReview.bib:2718}
\bibitem{costan_sanctum:_2016}\mdbibitemlabel{}Costan, V., Lebedev, I., \& Devadas, S.~(2016). Sanctum: Minimal Hardware Extensions for Strong Software Isolation (pp. 857–874). Retrieved from \href{https://www.usenix.org/conference/usenixsecurity16/technical-sessions/presentation/costan}{{\ttfamily https://\hspace{0pt}www.\hspace{0pt}usenix.\hspace{0pt}org/\hspace{0pt}conference/\hspace{0pt}usenixsecurity16/\hspace{0pt}technical-\hspace{0pt}sessions/\hspace{0pt}presentation/\hspace{0pt}costan}}\label{costan_sanctum:_2016}%mdk%mdk

%mdk-data-line={FormalReview.bib:2731}
\bibitem{costan_secure_2017}\mdbibitemlabel{}Costan, V., Lebedev, I., \& Devadas, S.~(2017a). Secure Processors Part I: Background, Taxonomy for Secure Enclaves and Intel SGX Architecture. \emph{Foundations and Trends® in Electronic Design Automation}, \emph{11}(1), 1–248. https://doi.org/\href{https://dx.doi.org/10.1561/1000000051}{10.1561/1000000051}\label{costan_secure_2017}%mdk%mdk

%mdk-data-line={FormalReview.bib:2748}
\bibitem{costan_secure_2017-1}\mdbibitemlabel{}Costan, V., Lebedev, I., \& Devadas, S.~(2017b). Secure Processors Part II: Intel SGX Security Analysis and MIT Sanctum Architecture. \emph{Foundations and Trends® in Electronic Design Automation}, \emph{11}(3), 249–361. https://doi.org/\href{https://dx.doi.org/10.1561/1000000052}{10.1561/1000000052}\label{costan_secure_2017-1}%mdk%mdk

%mdk-data-line={FormalReview.bib:577}
\bibitem{costanzo_end--end_2016}\mdbibitemlabel{}Costanzo, D., Shao, Z., \& Gu, R.~(2016). End-to-end Verification of Information-flow Security for C and Assembly Programs. In \emph{Proceedings of the 37th ACM SIGPLAN Conference on Programming Language Design and Implementation} (pp. 648–664). New York, NY, USA: ACM.~https://doi.org/\href{https://dx.doi.org/10.1145/2908080.2908100}{10.1145/2908080.2908100}\label{costanzo_end--end_2016}%mdk%mdk

%mdk-data-line={FormalReview.bib:566}
\bibitem{costanzo_end--end_nodate}\mdbibitemlabel{}Costanzo, D., Shao, Z., \& Gu, R.~(n.d.). End-to-End Verification of Information-Flow Security for C and Assembly Programs - Tech Report, 21. Retrieved from \href{http://flint.cs.yale.edu/certikos/publications/security-tr.pdf}{{\ttfamily http://\hspace{0pt}flint.\hspace{0pt}cs.\hspace{0pt}yale.\hspace{0pt}edu/\hspace{0pt}certikos/\hspace{0pt}publications/\hspace{0pt}security-\hspace{0pt}tr.\hspace{0pt}pdf}}\label{costanzo_end--end_nodate}%mdk%mdk

%mdk-data-line={FormalReview.bib:846}
\bibitem{crary_modules_2017}\mdbibitemlabel{}Crary, K.~(2017). Modules, Abstraction, and Parametric Polymorphism. In \emph{Proceedings of the 44th ACM SIGPLAN Symposium on Principles of Programming Languages} (pp. 100–113). New York, NY, USA: ACM.~https://doi.org/\href{https://dx.doi.org/10.1145/3009837.3009892}{10.1145/3009837.3009892}\label{crary_modules_2017}%mdk%mdk

%mdk-data-line={FormalReview.bib:2512}
\bibitem{crick_share_2014}\mdbibitemlabel{}Crick, T., Hall, B.~A., Ishtiaq, S., \& Takeda, K.~(2014). \textquotedblleft{}Share and Enjoy\textquotedblright{}: Publishing Useful and Usable Scientific Models. \emph{arXiv:1409.0367 {}[cs]}. Retrieved from arXiv:\href{http://arxiv.org/abs/1409.0367}{1409.0367}\label{crick_share_2014}%mdk%mdk

%mdk-data-line={FormalReview.bib:1054}
\bibitem{czajka_coqhammer:_nodate}\mdbibitemlabel{}Czajka, L., \& Kaliszyk, C.~(n.d.). CoqHammer: Strong Automation for Program Verification - CoqPL 2018. Retrieved January 31, 2019, from \href{https://popl18.sigplan.org/event/coqpl-2018-coqhammer-strong-automation-for-program-verification}{{\ttfamily https://\hspace{0pt}popl18.\hspace{0pt}sigplan.\hspace{0pt}org/\hspace{0pt}event/\hspace{0pt}coqpl-\hspace{0pt}2018-\hspace{0pt}coqhammer-\hspace{0pt}strong-\hspace{0pt}automation-\hspace{0pt}for-\hspace{0pt}program-\hspace{0pt}verification}}\label{czajka_coqhammer:_nodate}%mdk%mdk

%mdk-data-line={FormalReview.bib:1628}
\bibitem{david_cristina_program_2017}\mdbibitemlabel{}David Cristina, \& Kroening Daniel. (2017). Program synthesis: challenges and opportunities. \emph{Philosophical Transactions of the Royal Society A: Mathematical, Physical and Engineering Sciences}, \emph{375}(2104), 20150403. https://doi.org/\href{https://dx.doi.org/10.1098/rsta.2015.0403}{10.1098/rsta.2015.0403}\label{david_cristina_program_2017}%mdk%mdk

%mdk-data-line={FormalReview.bib:1733}
\bibitem{parigot_tactic_2000}\mdbibitemlabel{}Delahaye, D.~(2000). A Tactic Language for the System Coq. In M.~Parigot \& A.~Voronkov (Eds.), \emph{Logic for Programming and Automated Reasoning} (Vol. 1955, pp. 85–95). Berlin, Heidelberg: Springer Berlin Heidelberg. https://doi.org/\href{https://dx.doi.org/10.1007/3-540-44404-1_7}{10.1007/3-540-44404-1\_7}\label{parigot_tactic_2000}%mdk%mdk

%mdk-data-line={FormalReview.bib:1418}
\bibitem{delaware_fiat:_2015}\mdbibitemlabel{}Delaware, B., Pit-Claudel, C., Gross, J., \& Chlipala, A.~(2015). Fiat: Deductive Synthesis of Abstract Data Types in a Proof Assistant. In \emph{Proceedings of the 42Nd Annual ACM SIGPLAN-SIGACT Symposium on Principles of Programming Languages} (pp. 689–700). New York, NY, USA: ACM.~https://doi.org/\href{https://dx.doi.org/10.1145/2676726.2677006}{10.1145/2676726.2677006}\label{delaware_fiat:_2015}%mdk%mdk

%mdk-data-line={FormalReview.bib:1546}
\bibitem{delaware_narcissus:_nodate}\mdbibitemlabel{}Delaware, B., Suriyakarn, S., Pit-Claudel, C., Ye, Q., \& Chlipala, A.~(n.d.). Narcissus: Correct-By-Construction Derivation of Decoders and Encoders from Binary Formats, 14.\label{delaware_narcissus:_nodate}%mdk%mdk

%mdk-data-line={FormalReview.bib:1555}
\bibitem{delaware_narcissus:_2018}\mdbibitemlabel{}Delaware, B., Suriyakarn, S., Pit\textendash{}Claudel, C., Ye, Q., \& Chlipala, A.~(2018). Narcissus: Deriving Correct-By-Construction Decoders and Encoders from Binary Formats. Retrieved from \href{https://arxiv.org/abs/1803.04870v2}{{\ttfamily https://\hspace{0pt}arxiv.\hspace{0pt}org/\hspace{0pt}abs/\hspace{0pt}1803.\hspace{0pt}04870v2}}\label{delaware_narcissus:_2018}%mdk%mdk

%mdk-data-line={FormalReview.bib:1799}
\bibitem{delignat-lavaud_implementing_2017}\mdbibitemlabel{}Delignat-Lavaud, A., Fournet, C., Kohlweiss, M., Protzenko, J., Rastogi, A., Swamy, N., … Zinzindohoue, J.~K.~(2017). Implementing and Proving the TLS 1.3 Record Layer. Retrieved from \href{https://www.microsoft.com/en-us/research/publication/implementing-proving-tls-1-3-record-layer/}{{\ttfamily https://\hspace{0pt}www.\hspace{0pt}microsoft.\hspace{0pt}com/\hspace{0pt}en-\hspace{0pt}us/\hspace{0pt}research/\hspace{0pt}publication/\hspace{0pt}implementing-\hspace{0pt}proving-\hspace{0pt}tls-\hspace{0pt}1-\hspace{0pt}3-\hspace{0pt}record-\hspace{0pt}layer/\hspace{0pt}}}\label{delignat-lavaud_implementing_2017}%mdk%mdk

%mdk-data-line={FormalReview.bib:1925}
\bibitem{dijkstra_guarded_1975}\mdbibitemlabel{}Dijkstra, E.~W.~(1975). Guarded Commands, Nondeterminacy and Formal Derivation of Programs. \emph{Commun. ACM}, \emph{18}(8), 453–457. https://doi.org/\href{https://dx.doi.org/10.1145/360933.360975}{10.1145/360933.360975}\label{dijkstra_guarded_1975}%mdk%mdk

%mdk-data-line={FormalReview.bib:3100}
\bibitem{hermanns_local_2006}\mdbibitemlabel{}Distefano, D., O’Hearn, P.~W., \& Yang, H.~(2006). A Local Shape Analysis Based on Separation Logic. In H.~Hermanns \& J.~Palsberg (Eds.), \emph{Tools and Algorithms for the Construction and Analysis of Systems} (Vol. 3920, pp. 287–302). Berlin, Heidelberg: Springer Berlin Heidelberg. https://doi.org/\href{https://dx.doi.org/10.1007/11691372_19}{10.1007/11691372\_19}\label{hermanns_local_2006}%mdk%mdk

%mdk-data-line={FormalReview.bib:2211}
\bibitem{hutchison_fresh_2009}\mdbibitemlabel{}Dockins, R., Hobor, A., \& Appel, A.~W.~(2009). A Fresh Look at Separation Algebras and Share Accounting. In Z.~Hu (Ed.), \emph{Programming Languages and Systems} (Vol. 5904, pp. 161–177). Berlin, Heidelberg: Springer Berlin Heidelberg. https://doi.org/\href{https://dx.doi.org/10.1007/978-3-642-10672-9_13}{10.1007/978-3-642-10672-9\_13}\label{hutchison_fresh_2009}%mdk%mdk

%mdk-data-line={FormalReview.bib:2633}
\bibitem{ebner_metaprogramming_2017}\mdbibitemlabel{}Ebner, G., Ullrich, S., Roesch, J., Avigad, J., \& Moura, L.~de. (2017). A Metaprogramming Framework for Formal Verification. \emph{Proc. ACM Program. Lang.}, \emph{1}, 34:1–34:29. https://doi.org/\href{https://dx.doi.org/10.1145/3110278}{10.1145/3110278}\label{ebner_metaprogramming_2017}%mdk%mdk

%mdk-data-line={FormalReview.bib:1675}
\bibitem{ekici_smtcoq:_2017}\mdbibitemlabel{}Ekici, B., Mebsout, A., Tinelli, C., Keller, C., Katz, G., Reynolds, A., \& Barrett, C.~(2017). SMTCoq: A Plug-In for Integrating SMT Solvers into Coq. In R.~Majumdar \& V.~Kunčak (Eds.), \emph{Computer Aided Verification} (pp. 126–133). Springer International Publishing.\label{ekici_smtcoq:_2017}%mdk%mdk

%mdk-data-line={FormalReview.bib:2054}
\bibitem{filinski_representing_1994}\mdbibitemlabel{}Filinski, A.~(1994). Representing Monads. In \emph{Proceedings of the 21st ACM SIGPLAN-SIGACT Symposium on Principles of Programming Languages} (pp. 446–457). New York, NY, USA: ACM.~https://doi.org/\href{https://dx.doi.org/10.1145/174675.178047}{10.1145/174675.178047}\label{filinski_representing_1994}%mdk%mdk

%mdk-data-line={FormalReview.bib:2071}
\bibitem{filinski_representing_1999}\mdbibitemlabel{}Filinski, A.~(1999). Representing Layered Monads. In \emph{Proceedings of the 26th ACM SIGPLAN-SIGACT Symposium on Principles of Programming Languages} (pp. 175–188). New York, NY, USA: ACM.~https://doi.org/\href{https://dx.doi.org/10.1145/292540.292557}{10.1145/292540.292557}\label{filinski_representing_1999}%mdk%mdk

%mdk-data-line={FormalReview.bib:1496}
\bibitem{fisher_kathleen_hacms_2017}\mdbibitemlabel{}Fisher Kathleen, Launchbury John, \& Richards Raymond. (2017). The HACMS program: using formal methods to eliminate exploitable bugs. \emph{Philosophical Transactions of the Royal Society A: Mathematical, Physical and Engineering Sciences}, \emph{375}(2104), 20150401. https://doi.org/\href{https://dx.doi.org/10.1098/rsta.2015.0401}{10.1098/rsta.2015.0401}\label{fisher_kathleen_hacms_2017}%mdk%mdk

%mdk-data-line={FormalReview.bib:2243}
\bibitem{fournet_deploying_nodate}\mdbibitemlabel{}Fournet, C., Hawblitzel, C., Parno, B., \& Swamy, N.~(n.d.). Deploying a Verified Secure Implementation of the HTTPS Ecosystem, 10.\label{fournet_deploying_nodate}%mdk%mdk

%mdk-data-line={FormalReview.bib:1763}
\bibitem{fowler_deriving_nodate}\mdbibitemlabel{}Fowler, M.~(n.d.). Deriving Kepler’s Laws from the Inverse-Square Law. Retrieved February 1, 2019, from \href{http://galileo.phys.virginia.edu/classes/152.mf1i.spring02/KeplersLaws.htm}{{\ttfamily http://\hspace{0pt}galileo.\hspace{0pt}phys.\hspace{0pt}virginia.\hspace{0pt}edu/\hspace{0pt}classes/\hspace{0pt}152.\hspace{0pt}mf1i.\hspace{0pt}spring02/\hspace{0pt}KeplersLaws.\hspace{0pt}htm}}\label{fowler_deriving_nodate}%mdk%mdk

%mdk-data-line={FormalReview.bib:958}
\bibitem{felty_keymaera_2015}\mdbibitemlabel{}Fulton, N., Mitsch, S., Quesel, J.-D., Völp, M., \& Platzer, A.~(2015). KeYmaera X: An Axiomatic Tactical Theorem Prover for Hybrid Systems. In A.~P.~Felty \& A.~Middeldorp (Eds.), \emph{Automated Deduction - CADE-25} (Vol. 9195, pp. 527–538). Cham: Springer International Publishing. https://doi.org/\href{https://dx.doi.org/10.1007/978-3-319-21401-6_36}{10.1007/978-3-319-21401-6\_36}\label{felty_keymaera_2015}%mdk%mdk

%mdk-data-line={FormalReview.bib:2706}
\bibitem{gasser_building_1988}\mdbibitemlabel{}Gasser, M.~(1988). \emph{Building a secure computer system}. New York: Van Nostrand Reinhold Co.\label{gasser_building_1988}%mdk%mdk

%mdk-data-line={FormalReview.bib:1446}
\bibitem{gonthier_formal_2008}\mdbibitemlabel{}Gonthier, G.~(2008). Formal Proof—The Four- Color Theorem, \emph{55}(11), 12.\label{gonthier_formal_2008}%mdk%mdk

%mdk-data-line={FormalReview.bib:2088}
\bibitem{gonthier_introduction_2010}\mdbibitemlabel{}Gonthier, G., \& Mahboubi, A.~(2010). An introduction to small scale reflection in Coq. \emph{Journal of Formalized Reasoning}, \emph{3}(2), 95–152. https://doi.org/\href{https://dx.doi.org/10.6092/issn.1972-5787/1979}{10.6092/issn.1972-5787/1979}\label{gonthier_introduction_2010}%mdk%mdk

%mdk-data-line={FormalReview.bib:2106}
\bibitem{gonthier_how_2011}\mdbibitemlabel{}Gonthier, G., Ziliani, B., Nanevski, A., \& Dreyer, D.~(2011). How to Make Ad Hoc Proof Automation Less Ad Hoc. In \emph{Proceedings of the 16th ACM SIGPLAN International Conference on Functional Programming} (pp. 163–175). New York, NY, USA: ACM.~https://doi.org/\href{https://dx.doi.org/10.1145/2034773.2034798}{10.1145/2034773.2034798}\label{gonthier_how_2011}%mdk%mdk

%mdk-data-line={FormalReview.bib:72}
\bibitem{gorogiannis_true_2019}\mdbibitemlabel{}Gorogiannis, N., O’Hearn, P.~W., \& Sergey, I.~(2019). A true positives theorem for a static race detector. \emph{Proceedings of the ACM on Programming Languages}, \emph{3}, 1–29. https://doi.org/\href{https://dx.doi.org/10.1145/3290370}{10.1145/3290370}\label{gorogiannis_true_2019}%mdk%mdk

%mdk-data-line={FormalReview.bib:498}
\bibitem{gu_deep_2015}\mdbibitemlabel{}Gu, R., Koenig, J., Ramananandro, T., Shao, Z., Wu, X.~(Newman), Weng, S.-C., … Guo, Y.~(2015). Deep Specifications and Certified Abstraction Layers. In \emph{Proceedings of the 42Nd Annual ACM SIGPLAN-SIGACT Symposium on Principles of Programming Languages} (pp. 595–608). New York, NY, USA: ACM.~https://doi.org/\href{https://dx.doi.org/10.1145/2676726.2676975}{10.1145/2676726.2676975}\label{gu_deep_2015}%mdk%mdk

%mdk-data-line={FormalReview.bib:549}
\bibitem{gu_certikos:_2016}\mdbibitemlabel{}Gu, R., Shao, Z., Chen, H., Wu, X., Kim, J., Sjöberg, V., \& Costanzo, D.~(2016). CertiKOS: An Extensible Architecture for Building Certified Concurrent OS Kernels. In \emph{Proceedings of the 12th USENIX Conference on Operating Systems Design and Implementation} (pp. 653–669). Berkeley, CA, USA: USENIX Association. Retrieved from \href{http://dl.acm.org/citation.cfm?id=3026877.3026928}{{\ttfamily http://\hspace{0pt}dl.\hspace{0pt}acm.\hspace{0pt}org/\hspace{0pt}citation.\hspace{0pt}cfm?\hspace{0pt}id=\hspace{0pt}3026877.\hspace{0pt}3026928}}\label{gu_certikos:_2016}%mdk%mdk

%mdk-data-line={FormalReview.bib:595}
\bibitem{gu_certified_2018}\mdbibitemlabel{}Gu, R., Shao, Z., Kim, J., Wu, X.~(Newman), Koenig, J., Sjöberg, V., … Ramananandro, T.~(2018). Certified Concurrent Abstraction Layers. In \emph{Proceedings of the 39th ACM SIGPLAN Conference on Programming Language Design and Implementation} (pp. 646–661). New York, NY, USA: ACM.~https://doi.org/\href{https://dx.doi.org/10.1145/3192366.3192381}{10.1145/3192366.3192381}\label{gu_certified_2018}%mdk%mdk

%mdk-data-line={FormalReview.bib:2490}
\bibitem{hutchison_seloger:_2013}\mdbibitemlabel{}Haase, C., Ishtiaq, S., Ouaknine, J., \& Parkinson, M.~J.~(2013). SeLoger: A Tool for Graph-Based Reasoning in Separation Logic. In N.~Sharygina \& H.~Veith (Eds.), \emph{Computer Aided Verification} (Vol. 8044, pp. 790–795). Berlin, Heidelberg: Springer Berlin Heidelberg. https://doi.org/\href{https://dx.doi.org/10.1007/978-3-642-39799-8_55}{10.1007/978-3-642-39799-8\_55}\label{hutchison_seloger:_2013}%mdk%mdk

%mdk-data-line={FormalReview.bib:2160}
\bibitem{harper_framework_1993}\mdbibitemlabel{}Harper, R., Honsell, F., \& Plotkin, G.~(1993). A Framework for Defining Logics. \emph{J.~ACM}, \emph{40}(1), 143–184. https://doi.org/\href{https://dx.doi.org/10.1145/138027.138060}{10.1145/138027.138060}\label{harper_framework_1993}%mdk%mdk

%mdk-data-line={FormalReview.bib:1468}
\bibitem{harrison_formal_2008}\mdbibitemlabel{}Harrison, J.~(2008). Formal Proof—Theory and Practice, \emph{55}(11), 12.\label{harrison_formal_2008}%mdk%mdk

%mdk-data-line={FormalReview.bib:2177}
\bibitem{harrison_hol_2013}\mdbibitemlabel{}Harrison, J.~(2013). The HOL Light Theory of Euclidean Space. \emph{Journal of Automated Reasoning}, \emph{50}(2), 173–190. https://doi.org/\href{https://dx.doi.org/10.1007/s10817-012-9250-9}{10.1007/s10817-012-9250-9}\label{harrison_hol_2013}%mdk%mdk

%mdk-data-line={FormalReview.bib:1202}
\bibitem{hatcliff_behavioral_2012}\mdbibitemlabel{}Hatcliff, J., Leavens, G.~T., Leino, K.~R.~M., Müller, P., \& Parkinson, M.~(2012). Behavioral Interface Specification Languages. \emph{ACM Comput. Surv.}, \emph{44}(3), 16:1–16:58. https://doi.org/\href{https://dx.doi.org/10.1145/2187671.2187678}{10.1145/2187671.2187678}\label{hatcliff_behavioral_2012}%mdk%mdk

%mdk-data-line={FormalReview.bib:2570}
\bibitem{hathhorn_defining_2015}\mdbibitemlabel{}Hathhorn, C., Ellison, C., \& Roşu, G.~(2015). Defining the Undefinedness of C.~In \emph{Proceedings of the 36th ACM SIGPLAN Conference on Programming Language Design and Implementation} (pp. 336–345). New York, NY, USA: ACM.~https://doi.org/\href{https://dx.doi.org/10.1145/2737924.2737979}{10.1145/2737924.2737979}\label{hathhorn_defining_2015}%mdk%mdk

%mdk-data-line={FormalReview.bib:2260}
\bibitem{hawblitzel_ironfleet:_2015}\mdbibitemlabel{}Hawblitzel, C., Howell, J., Kapritsos, M., Lorch, J.~R., Parno, B., Roberts, M.~L., … Zill, B.~(2015). IronFleet: Proving Practical Distributed Systems Correct. In \emph{Proceedings of the 25th Symposium on Operating Systems Principles} (pp. 1–17). New York, NY, USA: ACM.~https://doi.org/\href{https://dx.doi.org/10.1145/2815400.2815428}{10.1145/2815400.2815428}\label{hawblitzel_ironfleet:_2015}%mdk%mdk

%mdk-data-line={FormalReview.bib:2251}
\bibitem{hawblitzel_ironclad_nodate}\mdbibitemlabel{}Hawblitzel, C., Howell, J., Lorch, J.~R., Narayan, A., Parno, B., Zhang, D., \& Zill, B.~(n.d.). Ironclad Apps: End-to-End Security via Automated Full-System Verification, 18.\label{hawblitzel_ironclad_nodate}%mdk%mdk

%mdk-data-line={FormalReview.bib:2278}
\bibitem{hawblitzel_automated_2015}\mdbibitemlabel{}Hawblitzel, C., Petrank, E., Qadeer, S., \& Tasiran, S.~(2015). Automated and Modular Refinement Reasoning for Concurrent Programs. In \emph{Computer Aided Verification} (pp. 449–465). Springer, Cham. https://doi.org/\href{https://dx.doi.org/10.1007/978-3-319-21668-3_26}{10.1007/978-3-319-21668-3\_26}\label{hawblitzel_automated_2015}%mdk%mdk

%mdk-data-line={FormalReview.bib:532}
\bibitem{herlihy_linearizability:_1990}\mdbibitemlabel{}Herlihy, M.~P., \& Wing, J.~M.~(1990). Linearizability: A Correctness Condition for Concurrent Objects. \emph{ACM Trans. Program. Lang. Syst.}, \emph{12}(3), 463–492. https://doi.org/\href{https://dx.doi.org/10.1145/78969.78972}{10.1145/78969.78972}\label{herlihy_linearizability:_1990}%mdk%mdk

%mdk-data-line={FormalReview.bib:243}
\bibitem{hobor_theory_2010}\mdbibitemlabel{}Hobor, A., Dockins, R., \& Appel, A.~W.~(2010). A Theory of Indirection via Approximation. In \emph{Proceedings of the 37th Annual ACM SIGPLAN-SIGACT Symposium on Principles of Programming Languages} (pp. 171–184). New York, NY, USA: ACM.~https://doi.org/\href{https://dx.doi.org/10.1145/1706299.1706322}{10.1145/1706299.1706322}\label{hobor_theory_2010}%mdk%mdk

%mdk-data-line={FormalReview.bib:2982}
\bibitem{holzl_type_2013}\mdbibitemlabel{}Hölzl, J., Immler, F., \& Huffman, B.~(2013). Type Classes and Filters for Mathematical Analysis in Isabelle/HOL.~In S.~Blazy, C.~Paulin-Mohring, \& D.~Pichardie (Eds.), \emph{Interactive Theorem Proving} (pp. 279–294). Springer Berlin Heidelberg.\label{holzl_type_2013}%mdk%mdk

%mdk-data-line={FormalReview.bib:225}
\bibitem{hritcu_micro-policies:_2015}\mdbibitemlabel{}Hriţcu, C.~(2015). Micro-Policies: Formally Verified, Tag-Based Security Monitors. In \emph{Proceedings of the 10th ACM Workshop on Programming Languages and Analysis for Security - PLAS’15} (pp. 1–1). Prague, Czech Republic: ACM Press. https://doi.org/\href{https://dx.doi.org/10.1145/2786558.2786560}{10.1145/2786558.2786560}\label{hritcu_micro-policies:_2015}%mdk%mdk

%mdk-data-line={FormalReview.bib:1989}
\bibitem{hritcu_quest_nodate}\mdbibitemlabel{}Hriţcu, C.~(n.d.). The Quest for Formally Secure Compartmentalizing Compilation, 96.\label{hritcu_quest_nodate}%mdk%mdk

%mdk-data-line={FormalReview.bib:1660}
\bibitem{hunt_warren_a._industrial_2017}\mdbibitemlabel{}Hunt Warren A., Kaufmann Matt, Moore J Strother, \& Slobodova Anna. (2017). Industrial hardware and software verification with ACL2. \emph{Philosophical Transactions of the Royal Society A: Mathematical, Physical and Engineering Sciences}, \emph{375}(2104), 20150399. https://doi.org/\href{https://dx.doi.org/10.1098/rsta.2015.0399}{10.1098/rsta.2015.0399}\label{hunt_warren_a._industrial_2017}%mdk%mdk

%mdk-data-line={FormalReview.bib:2935}
\bibitem{immler_verified_2018}\mdbibitemlabel{}Immler, F.~(2018). A Verified ODE Solver and the Lorenz Attractor. \emph{J Autom Reasoning}, \emph{61}(1), 73–111. https://doi.org/\href{https://dx.doi.org/10.1007/s10817-017-9448-y}{10.1007/s10817-017-9448-y}\label{immler_verified_2018}%mdk%mdk

%mdk-data-line={FormalReview.bib:835}
\bibitem{inria_inria_nodate}\mdbibitemlabel{}Inria. (n.d.). Inria - Inventors for the digital world.Inria. Retrieved January 31, 2019, from \href{https://www.inria.fr/en}{{\ttfamily https://\hspace{0pt}www.\hspace{0pt}inria.\hspace{0pt}fr/\hspace{0pt}en}}\label{inria_inria_nodate}%mdk%mdk

%mdk-data-line={FormalReview.bib:2474}
\bibitem{ishtiaq_bi_2011}\mdbibitemlabel{}Ishtiaq, S., \& O’Hearn, P.~W.~(2011). BI As an Assertion Language for Mutable Data Structures. \emph{SIGPLAN Not.}, \emph{46}(4), 84–96. https://doi.org/\href{https://dx.doi.org/10.1145/1988042.1988050}{10.1145/1988042.1988050}\label{ishtiaq_bi_2011}%mdk%mdk

%mdk-data-line={FormalReview.bib:2773}
\bibitem{jacobs_verifast/verifast:_2019}\mdbibitemlabel{}Jacobs, B.~(2019). \emph{verifast/verifast: Research prototype tool for modular formal verification of C and Java programs}. verifast. Retrieved from \href{https://github.com/verifast/verifast}{{\ttfamily https://\hspace{0pt}github.\hspace{0pt}com/\hspace{0pt}verifast/\hspace{0pt}verifast}}\label{jacobs_verifast/verifast:_2019}%mdk%mdk

%mdk-data-line={FormalReview.bib:2803}
\bibitem{jacobs_verifast_2008}\mdbibitemlabel{}Jacobs, B., \& Piessens, F.~(2008). \emph{The VeriFast program verifier}.\label{jacobs_verifast_2008}%mdk%mdk

%mdk-data-line={FormalReview.bib:2811}
\bibitem{jacobs_verifast_2017}\mdbibitemlabel{}Jacobs, B., Smans, J., \& Piessens, F.~(2017). The VeriFast Program Verifier: A Tutorial, 102.\label{jacobs_verifast_2017}%mdk%mdk

%mdk-data-line={FormalReview.bib:2785}
\bibitem{jacobs_featherweight_2015}\mdbibitemlabel{}Jacobs, B., Vogels, F., \& Piessens, F.~(2015). Featherweight VeriFast. \emph{Logical Methods in Computer Science}, \emph{11}(3). https://doi.org/\href{https://dx.doi.org/10.2168/LMCS-11\%25283:19\%25292015}{10.2168/LMCS-11(3:19)2015}\label{jacobs_featherweight_2015}%mdk%mdk

%mdk-data-line={FormalReview.bib:1841}
\bibitem{jeannet_apron_nodate}\mdbibitemlabel{}Jeannet, B., \& Miné, A.~(n.d.). APRON numerical abstract domain library. Retrieved February 1, 2019, from \href{http://apron.cri.ensmp.fr/library/}{{\ttfamily http://\hspace{0pt}apron.\hspace{0pt}cri.\hspace{0pt}ensmp.\hspace{0pt}fr/\hspace{0pt}library/\hspace{0pt}}}\label{jeannet_apron_nodate}%mdk%mdk

%mdk-data-line={FormalReview.bib:2417}
\bibitem{jung_iris_nodate}\mdbibitemlabel{}Jung, R.~(n.d.). Iris Project. Retrieved February 1, 2019, from \href{https://iris-project.org/}{{\ttfamily https://\hspace{0pt}iris-\hspace{0pt}project.\hspace{0pt}org/\hspace{0pt}}}\label{jung_iris_nodate}%mdk%mdk

%mdk-data-line={FormalReview.bib:34}
\bibitem{jung_rustbelt:_2017}\mdbibitemlabel{}Jung, R., Jourdan, J.-H., Krebbers, R., \& Dreyer, D.~(2017). RustBelt: securing the foundations of the rust programming language. \emph{Proceedings of the ACM on Programming Languages}, \emph{2}, 1–34. https://doi.org/\href{https://dx.doi.org/10.1145/3158154}{10.1145/3158154}\label{jung_rustbelt:_2017}%mdk%mdk

%mdk-data-line={FormalReview.bib:2433}
\bibitem{jung_iris_2018}\mdbibitemlabel{}Jung, R., Krebbers, R., Jourdan, J.-H., Bizjak, A., Birkedal, L., \& Dreyer, D.~(2018). Iris from the ground up: A modular foundation for higher-order concurrent separation logic. \emph{Journal of Functional Programming}, \emph{28}. https://doi.org/\href{https://dx.doi.org/10.1017/S0956796818000151}{10.1017/S0956796818000151}\label{jung_iris_2018}%mdk%mdk

%mdk-data-line={FormalReview.bib:2562}
\bibitem{kaiser_destruct_nodate}\mdbibitemlabel{}Kaiser, J.-O., \& Ziliani, B.~(n.d.). A \textquotedblleft{}destruct\textquotedblright{} Tactic for Mtac2 - POPL 2018. Retrieved February 1, 2019, from \href{https://popl18.sigplan.org/event/coqpl-2018-a-destruct-tactic-for-mtac2}{{\ttfamily https://\hspace{0pt}popl18.\hspace{0pt}sigplan.\hspace{0pt}org/\hspace{0pt}event/\hspace{0pt}coqpl-\hspace{0pt}2018-\hspace{0pt}a-\hspace{0pt}destruct-\hspace{0pt}tactic-\hspace{0pt}for-\hspace{0pt}mtac2}}\label{kaiser_destruct_nodate}%mdk%mdk

%mdk-data-line={FormalReview.bib:2662}
\bibitem{kang_crellvm:_2018}\mdbibitemlabel{}Kang, J., Kim, Y., Song, Y., Lee, J., Park, S., Shin, M.~D., … Yi, K.~(2018). Crellvm: Verified Credible Compilation for LLVM.~In \emph{Proceedings of the 39th ACM SIGPLAN Conference on Programming Language Design and Implementation} (pp. 631–645). New York, NY, USA: ACM.~https://doi.org/\href{https://dx.doi.org/10.1145/3192366.3192377}{10.1145/3192366.3192377}\label{kang_crellvm:_2018}%mdk%mdk

%mdk-data-line={FormalReview.bib:384}
\bibitem{kastner_program_2015}\mdbibitemlabel{}Kästner, D., \& Pohland, J.~(2015). Program Analysis on Evolving Software. In M.~Roy (Ed.), \emph{CARS 2015 - Critical Automotive applications: Robustness \& Safety}. Paris, France. Retrieved from \href{https://hal.archives-ouvertes.fr/hal-01192985}{{\ttfamily https://\hspace{0pt}hal.\hspace{0pt}archives-\hspace{0pt}ouvertes.\hspace{0pt}fr/\hspace{0pt}hal-\hspace{0pt}01192985}}\label{kastner_program_2015}%mdk%mdk

%mdk-data-line={FormalReview.bib:414}
\bibitem{kastner_astree:_nodate}\mdbibitemlabel{}Kästner, D., Wilhelm, S., Nenova, S., Miné, A., Rival, X., Mauborgne, L., … Cousot, R.~(n.d.). Astree: Proving the Absence of Runtime Errors, 9. Retrieved from \href{https://www.di.ens.fr/~rival/papers/erts10.pdf}{{\ttfamily https://\hspace{0pt}www.\hspace{0pt}di.\hspace{0pt}ens.\hspace{0pt}fr/\hspace{0pt}\textasciitilde{}rival/\hspace{0pt}papers/\hspace{0pt}erts10.\hspace{0pt}pdf}}\label{kastner_astree:_nodate}%mdk%mdk

%mdk-data-line={FormalReview.bib:1582}
\bibitem{klein_gerwin_provably_2017}\mdbibitemlabel{}Klein Gerwin, Andronick June, Keller Gabriele, Matichuk Daniel, Murray Toby, \& O’Connor Liam. (2017). Provably trustworthy systems. \emph{Philosophical Transactions of the Royal Society A: Mathematical, Physical and Engineering Sciences}, \emph{375}(2104), 20150404. https://doi.org/\href{https://dx.doi.org/10.1098/rsta.2015.0404}{10.1098/rsta.2015.0404}\label{klein_gerwin_provably_2017}%mdk%mdk

%mdk-data-line={FormalReview.bib:1253}
\bibitem{koenig_programming_2016}\mdbibitemlabel{}Koenig, J., \& Leino, R.~(2016). Programming Language Features for Refinement. Retrieved from \href{https://www.microsoft.com/en-us/research/publication/programming-language-features-refinement/}{{\ttfamily https://\hspace{0pt}www.\hspace{0pt}microsoft.\hspace{0pt}com/\hspace{0pt}en-\hspace{0pt}us/\hspace{0pt}research/\hspace{0pt}publication/\hspace{0pt}programming-\hspace{0pt}language-\hspace{0pt}features-\hspace{0pt}refinement/\hspace{0pt}}}\label{koenig_programming_2016}%mdk%mdk

%mdk-data-line={FormalReview.bib:2998}
\bibitem{krebbers_type_2011}\mdbibitemlabel{}Krebbers, R., \& Spitters, B.~(2011). Type classes for efficient exact real arithmetic in Coq. \emph{arXiv:1106.3448 {}[cs, Math]}. https://doi.org/\href{https://dx.doi.org/10.2168/LMCS-9\%25281:01\%25292013}{10.2168/LMCS-9(1:01)2013}\label{krebbers_type_2011}%mdk%mdk

%mdk-data-line={FormalReview.bib:18}
\bibitem{krishnan_modelling_2018}\mdbibitemlabel{}Krishnan, R., \& Lalithambika, V.~R.~(2018). Modelling and validating 1553B protocol using the SPIN model checker. In \emph{2018 10th International Conference on Communication Systems \& Networks (COMSNETS)} (pp. 472–475). Bengaluru: IEEE.~https://doi.org/\href{https://dx.doi.org/10.1109/COMSNETS.2018.8328247}{10.1109/COMSNETS.2018.8328247}\label{krishnan_modelling_2018}%mdk%mdk

%mdk-data-line={FormalReview.bib:2597}
\bibitem{kubota_foundations_2016}\mdbibitemlabel{}Kubota, K.~(2016). Foundations of Mathematics. https://doi.org/\href{https://dx.doi.org/10.4444/100.111}{10.4444/100.111}\label{kubota_foundations_2016}%mdk%mdk

%mdk-data-line={FormalReview.bib:2588}
\bibitem{kubota_foundations_nodate}\mdbibitemlabel{}Kubota, K.~(n.d.). Foundations of Mathematics – Owl of Minerva Press. Retrieved February 1, 2019, from \href{http://owlofminerva.net/foundations-of-mathematics/}{{\ttfamily http://\hspace{0pt}owlofminerva.\hspace{0pt}net/\hspace{0pt}foundations-\hspace{0pt}of-\hspace{0pt}mathematics/\hspace{0pt}}}\label{kubota_foundations_nodate}%mdk%mdk

%mdk-data-line={FormalReview.bib:2326}
\bibitem{lahiri_symdiff:_2012}\mdbibitemlabel{}Lahiri, S.~K., Hawblitzel, C., Kawaguchi, M., \& Rebêlo, H.~(2012). SYMDIFF: A Language-Agnostic Semantic Diff Tool for Imperative Programs. In P.~Madhusudan \& S.~A.~Seshia (Eds.), \emph{Computer Aided Verification} (pp. 712–717). Springer Berlin Heidelberg.\label{lahiri_symdiff:_2012}%mdk%mdk

%mdk-data-line={FormalReview.bib:2294}
\bibitem{lahiri_automatic_2015}\mdbibitemlabel{}Lahiri, S.~K., Sinha, R., \& Hawblitzel, C.~(2015). Automatic Rootcausing for Program Equivalence Failures in Binaries. In D.~Kroening \& C.~S.~Păsăreanu (Eds.), \emph{Computer Aided Verification} (pp. 362–379). Springer International Publishing.\label{lahiri_automatic_2015}%mdk%mdk

%mdk-data-line={FormalReview.bib:2608}
\bibitem{lamport_specifying_nodate}\mdbibitemlabel{}Lamport, L.~(n.d.). Specifying Systems. Retrieved February 1, 2019, from \href{https://lamport.azurewebsites.net/tla/book.html}{{\ttfamily https://\hspace{0pt}lamport.\hspace{0pt}azurewebsites.\hspace{0pt}net/\hspace{0pt}tla/\hspace{0pt}book.\hspace{0pt}html}}\label{lamport_specifying_nodate}%mdk%mdk

%mdk-data-line={FormalReview.bib:2836}
\bibitem{lampson_abcds_2001}\mdbibitemlabel{}Lampson, B.~(2001). The ABCD’s of Paxos. In \emph{Proceedings of the Twentieth Annual ACM Symposium on Principles of Distributed Computing} (p. 13 – ). New York, NY, USA: ACM.~https://doi.org/\href{https://dx.doi.org/10.1145/383962.383969}{10.1145/383962.383969}\label{lampson_abcds_2001}%mdk%mdk

%mdk-data-line={FormalReview.bib:1771}
\bibitem{lancaster_unified_1969}\mdbibitemlabel{}Lancaster, E.~R., \& Blanchard, R.~C.~(1969). A unified form of lambert’s theorem. \emph{NASA Technical Note}, \emph{\{TN\} D-5368}, 18.\label{lancaster_unified_1969}%mdk%mdk

%mdk-data-line={FormalReview.bib:1145}
\bibitem{leino_assertional_2015}\mdbibitemlabel{}Leino, K.~R.~M., \& Lucio, P.~(2015). An Assertional Proof of the Stability and Correctness of Natural Mergesort. \emph{ACM Trans. Comput. Logic}, \emph{17}(1), 6:1–6:22. https://doi.org/\href{https://dx.doi.org/10.1145/2814571}{10.1145/2814571}\label{leino_assertional_2015}%mdk%mdk

%mdk-data-line={FormalReview.bib:779}
\bibitem{chaudhuri_trigger_2016}\mdbibitemlabel{}Leino, K.~R.~M., \& Pit-Claudel, C.~(2016). Trigger Selection Strategies to Stabilize Program Verifiers. In S.~Chaudhuri \& A.~Farzan (Eds.), \emph{Computer Aided Verification} (Vol. 9779, pp. 361–381). Cham: Springer International Publishing. https://doi.org/\href{https://dx.doi.org/10.1007/978-3-319-41528-4_20}{10.1007/978-3-319-41528-4\_20}\label{chaudhuri_trigger_2016}%mdk%mdk

%mdk-data-line={FormalReview.bib:1264}
\bibitem{leino_compiling_2016}\mdbibitemlabel{}Leino, R.~(2016a). Compiling Hilbert’s epsilon Operator. \emph{LPAR-20. 20th International Conferences on Logic for Programming, Artificial Intelligence and Reasoning}, \emph{35}. Retrieved from \href{https://www.microsoft.com/en-us/research/publication/compiling-hilberts-\%25cf\%25b5-operator/}{{\ttfamily https://\hspace{0pt}www.\hspace{0pt}microsoft.\hspace{0pt}com/\hspace{0pt}en-\hspace{0pt}us/\hspace{0pt}research/\hspace{0pt}publication/\hspace{0pt}compiling-\hspace{0pt}hilberts-\hspace{0pt}\hspace{0pt}\%cf\hspace{0pt}\%b5-\hspace{0pt}operator/\hspace{0pt}}}\label{leino_compiling_2016}%mdk%mdk

%mdk-data-line={FormalReview.bib:1300}
\bibitem{leino_well-founded_2016}\mdbibitemlabel{}Leino, R.~(2016b). Well-Founded Functions and Extreme Predicates in Dafny: A Tutorial, \emph{40}. Retrieved from \href{https://www.microsoft.com/en-us/research/publication/well-founded-functions-extreme-predicates-dafny-tutorial/}{{\ttfamily https://\hspace{0pt}www.\hspace{0pt}microsoft.\hspace{0pt}com/\hspace{0pt}en-\hspace{0pt}us/\hspace{0pt}research/\hspace{0pt}publication/\hspace{0pt}well-\hspace{0pt}founded-\hspace{0pt}functions-\hspace{0pt}extreme-\hspace{0pt}predicates-\hspace{0pt}dafny-\hspace{0pt}tutorial/\hspace{0pt}}}\label{leino_well-founded_2016}%mdk%mdk

%mdk-data-line={FormalReview.bib:1178}
\bibitem{leino_co-induction_2013}\mdbibitemlabel{}Leino, R., \& Moskal, M.~(2013). Co-Induction Simply: Automatic Co-Inductive Proofs in a Program Verifier. Retrieved from \href{https://www.microsoft.com/en-us/research/publication/co-induction-simply-automatic-co-inductive-proofs-in-a-program-verifier/}{{\ttfamily https://\hspace{0pt}www.\hspace{0pt}microsoft.\hspace{0pt}com/\hspace{0pt}en-\hspace{0pt}us/\hspace{0pt}research/\hspace{0pt}publication/\hspace{0pt}co-\hspace{0pt}induction-\hspace{0pt}simply-\hspace{0pt}automatic-\hspace{0pt}co-\hspace{0pt}inductive-\hspace{0pt}proofs-\hspace{0pt}in-\hspace{0pt}a-\hspace{0pt}program-\hspace{0pt}verifier/\hspace{0pt}}}\label{leino_co-induction_2013}%mdk%mdk

%mdk-data-line={FormalReview.bib:1219}
\bibitem{leino_verification_2016}\mdbibitemlabel{}Leino, R., Müller, P., \& Smans, J.~(2016). Verification of Concurrent Programs with Chalice. Retrieved from \href{https://www.microsoft.com/en-us/research/publication/verification-concurrent-programs-chalice/}{{\ttfamily https://\hspace{0pt}www.\hspace{0pt}microsoft.\hspace{0pt}com/\hspace{0pt}en-\hspace{0pt}us/\hspace{0pt}research/\hspace{0pt}publication/\hspace{0pt}verification-\hspace{0pt}concurrent-\hspace{0pt}programs-\hspace{0pt}chalice/\hspace{0pt}}}\label{leino_verification_2016}%mdk%mdk

%mdk-data-line={FormalReview.bib:1289}
\bibitem{leino_verified_2016}\mdbibitemlabel{}Leino, R., \& Polikarpova, N.~(2016). Verified Calculations. Retrieved from \href{https://www.microsoft.com/en-us/research/publication/verified-calculations/}{{\ttfamily https://\hspace{0pt}www.\hspace{0pt}microsoft.\hspace{0pt}com/\hspace{0pt}en-\hspace{0pt}us/\hspace{0pt}research/\hspace{0pt}publication/\hspace{0pt}verified-\hspace{0pt}calculations/\hspace{0pt}}}\label{leino_verified_2016}%mdk%mdk

%mdk-data-line={FormalReview.bib:1241}
\bibitem{leino_fine-grained_2016}\mdbibitemlabel{}Leino, R., \& Wüstholz, V.~(2016). Fine-grained Caching of Verification Results, \emph{9206}. Retrieved from \href{https://www.microsoft.com/en-us/research/publication/fine-grained-caching-verification-results/}{{\ttfamily https://\hspace{0pt}www.\hspace{0pt}microsoft.\hspace{0pt}com/\hspace{0pt}en-\hspace{0pt}us/\hspace{0pt}research/\hspace{0pt}publication/\hspace{0pt}fine-\hspace{0pt}grained-\hspace{0pt}caching-\hspace{0pt}verification-\hspace{0pt}results/\hspace{0pt}}}\label{leino_fine-grained_2016}%mdk%mdk

%mdk-data-line={FormalReview.bib:1230}
\bibitem{leino_stepwise_2016}\mdbibitemlabel{}Leino, R., \& Yessenov, K.~(2016). Stepwise Refinement of Heap-Manipulating Code in Chalice. Retrieved from \href{https://www.microsoft.com/en-us/research/publication/stepwise-refinement-heap-manipulating-code-chalice/}{{\ttfamily https://\hspace{0pt}www.\hspace{0pt}microsoft.\hspace{0pt}com/\hspace{0pt}en-\hspace{0pt}us/\hspace{0pt}research/\hspace{0pt}publication/\hspace{0pt}stepwise-\hspace{0pt}refinement-\hspace{0pt}heap-\hspace{0pt}manipulating-\hspace{0pt}code-\hspace{0pt}chalice/\hspace{0pt}}}\label{leino_stepwise_2016}%mdk%mdk

%mdk-data-line={FormalReview.bib:2820}
\bibitem{leroy_ocaml_nodate}\mdbibitemlabel{}Leroy, X.~(n.d.). OCaml Home Page. Retrieved February 1, 2019, from \href{https://ocaml.org/}{{\ttfamily https://\hspace{0pt}ocaml.\hspace{0pt}org/\hspace{0pt}}}\label{leroy_ocaml_nodate}%mdk%mdk

%mdk-data-line={FormalReview.bib:2650}
\bibitem{letouzey_certified_nodate}\mdbibitemlabel{}Letouzey, P.~(n.d.). Certified functional programming : Program extraction within Coq proof assistant.ResearchGate. Retrieved February 1, 2019, from \href{https://www.researchgate.net/publication/280790704_Certified_functional_programming_Program_extraction_within_Coq_proof_assistant}{{\ttfamily https://\hspace{0pt}www.\hspace{0pt}researchgate.\hspace{0pt}net/\hspace{0pt}publication/\hspace{0pt}280790704\_\hspace{0pt}Certified\_\hspace{0pt}functional\_\hspace{0pt}programming\_\hspace{0pt}Program\_\hspace{0pt}extraction\_\hspace{0pt}within\_\hspace{0pt}Coq\_\hspace{0pt}proof\_\hspace{0pt}assistant}}\label{letouzey_certified_nodate}%mdk%mdk

%mdk-data-line={FormalReview.bib:2681}
\bibitem{luo_extended_nodate}\mdbibitemlabel{}Luo, Z.~(n.d.). An Extended Calculus of Constructions. Retrieved February 1, 2019, from \href{http://www.lfcs.inf.ed.ac.uk/reports/90/ECS-LFCS-90-118/}{{\ttfamily http://\hspace{0pt}www.\hspace{0pt}lfcs.\hspace{0pt}inf.\hspace{0pt}ed.\hspace{0pt}ac.\hspace{0pt}uk/\hspace{0pt}reports/\hspace{0pt}90/\hspace{0pt}ECS-\hspace{0pt}LFCS-\hspace{0pt}90-\hspace{0pt}118/\hspace{0pt}}}\label{luo_extended_nodate}%mdk%mdk

%mdk-data-line={FormalReview.bib:1700}
\bibitem{malecha_reflection_2018}\mdbibitemlabel{}Malecha, G.~(2018). \emph{Reflection library for Coq. Contribute to gmalecha/template-coq development by creating an account on GitHub}. Retrieved from \href{https://github.com/gmalecha/template-coq}{{\ttfamily https://\hspace{0pt}github.\hspace{0pt}com/\hspace{0pt}gmalecha/\hspace{0pt}template-\hspace{0pt}coq}}\label{malecha_reflection_2018}%mdk%mdk

%mdk-data-line={FormalReview.bib:2892}
\bibitem{martin-dorel_proving_2016}\mdbibitemlabel{}Martin-Dorel, É., \& Melquiond, G.~(2016). Proving Tight Bounds on Univariate Expressions with Elementary Functions in Coq. \emph{J Autom Reasoning}, \emph{57}(3), 187–217. https://doi.org/\href{https://dx.doi.org/10.1007/s10817-015-9350-4}{10.1007/s10817-015-9350-4}\label{martin-dorel_proving_2016}%mdk%mdk

%mdk-data-line={FormalReview.bib:757}
\bibitem{martin_mastering_2013}\mdbibitemlabel{}Martin, K., Hoffman, B., \& Cedilnik, A.~(2013). \emph{Mastering CMake: a cross-platform build system ; covers installing and running CMake ; details converting existing build processes to CMake ; create powerful cross-platform build scripts} (6. ed). Clifton Park, NY: Kitware.\label{martin_mastering_2013}%mdk%mdk

%mdk-data-line={FormalReview.bib:1917}
\bibitem{melquiond_why3_nodate}\mdbibitemlabel{}Melquiond, G.~(n.d.). Why3. Retrieved February 1, 2019, from \href{http://why3.lri.fr/}{{\ttfamily http://\hspace{0pt}why3.\hspace{0pt}lri.\hspace{0pt}fr/\hspace{0pt}}}\label{melquiond_why3_nodate}%mdk%mdk

%mdk-data-line={FormalReview.bib:425}
\bibitem{mine_taking_2016}\mdbibitemlabel{}Miné, A., Mauborgne, L., Rival, X., Feret, J., Cousot, P., Kästner, D., … Ferdinand, C.~(2016). Taking Static Analysis to the Next Level: Proving the Absence of Run-Time Errors and Data Races with Astrée. In \emph{8th European Congress on Embedded Real Time Software and Systems (ERTS 2016)}. Toulouse, France. Retrieved from \href{https://hal.archives-ouvertes.fr/hal-01271552}{{\ttfamily https://\hspace{0pt}hal.\hspace{0pt}archives-\hspace{0pt}ouvertes.\hspace{0pt}fr/\hspace{0pt}hal-\hspace{0pt}01271552}}\label{mine_taking_2016}%mdk%mdk

%mdk-data-line={FormalReview.bib:2828}
\bibitem{minsky_real_nodate}\mdbibitemlabel{}Minsky, Y., Madhavapeddy, A., \& Hickey, J.~(n.d.). Real World OCaml. Retrieved February 1, 2019, from \href{http://dev.realworldocaml.org/}{{\ttfamily http://\hspace{0pt}dev.\hspace{0pt}realworldocaml.\hspace{0pt}org/\hspace{0pt}}}\label{minsky_real_nodate}%mdk%mdk

%mdk-data-line={FormalReview.bib:2124}
\bibitem{mokhov_algebraic_2017}\mdbibitemlabel{}Mokhov, A.~(2017). Algebraic Graphs with Class (Functional Pearl). In \emph{Proceedings of the 10th ACM SIGPLAN International Symposium on Haskell} (pp. 2–13). New York, NY, USA: ACM.~https://doi.org/\href{https://dx.doi.org/10.1145/3122955.3122956}{10.1145/3122955.3122956}\label{mokhov_algebraic_2017}%mdk%mdk

%mdk-data-line={FormalReview.bib:397}
\bibitem{monniaux_parallel_2005}\mdbibitemlabel{}Monniaux, D.~(2005). The parallel implementation of the Astr\textbackslash{}’\{e\}e static analyzer. \emph{arXiv:cs/0701191}, \emph{3780}, 86–96. https://doi.org/\href{https://dx.doi.org/10.1007/11575467_7}{10.1007/11575467\_7}\label{monniaux_parallel_2005}%mdk%mdk

%mdk-data-line={FormalReview.bib:516}
\bibitem{murawski_invitation_2016}\mdbibitemlabel{}Murawski, A.~S., \& Tzevelekos, N.~(2016). An Invitation to Game Semantics. \emph{ACM SIGLOG News}, \emph{3}(2), 56–67. https://doi.org/\href{https://dx.doi.org/10.1145/2948896.2948902}{10.1145/2948896.2948902}\label{murawski_invitation_2016}%mdk%mdk

%mdk-data-line={FormalReview.bib:2381}
\bibitem{nelson_hyperkernel:_2017}\mdbibitemlabel{}Nelson, L., Sigurbjarnarson, H., Zhang, K., Johnson, D., Bornholt, J., Torlak, E., \& Wang, X.~(2017a). Hyperkernel: Push-Button Verification of an OS Kernel. In \emph{Proceedings of the 26th Symposium on Operating Systems Principles} (pp. 252–269). New York, NY, USA: ACM.~https://doi.org/\href{https://dx.doi.org/10.1145/3132747.3132748}{10.1145/3132747.3132748}\label{nelson_hyperkernel:_2017}%mdk%mdk

%mdk-data-line={FormalReview.bib:2399}
\bibitem{nelson_hyperkernel:_2017-1}\mdbibitemlabel{}Nelson, L., Sigurbjarnarson, H., Zhang, K., Johnson, D., Bornholt, J., Torlak, E., \& Wang, X.~(2017b). Hyperkernel: Push-Button Verification of an OS Kernel - Slides. In \emph{Proceedings of the 26th Symposium on Operating Systems Principles~- SOSP ’17} (pp. 252–269). Shanghai, China: ACM Press. https://doi.org/\href{https://dx.doi.org/10.1145/3132747.3132748}{10.1145/3132747.3132748}\label{nelson_hyperkernel:_2017-1}%mdk%mdk

%mdk-data-line={FormalReview.bib:135}
\bibitem{ohearn_categorical_2015}\mdbibitemlabel{}O’Hearn, P.~(2015). From Categorical Logic to Facebook Engineering. In \emph{Proceedings of the 2015 30th Annual ACM/IEEE Symposium on Logic in Computer Science (LICS)} (pp. 17–20). Washington, DC, USA: IEEE Computer Society. https://doi.org/\href{https://dx.doi.org/10.1109/LICS.2015.11}{10.1109/LICS.2015.11}\label{ohearn_categorical_2015}%mdk%mdk

%mdk-data-line={FormalReview.bib:51}
\bibitem{ohearn_separation_2019}\mdbibitemlabel{}O’Hearn, P.~(2019). Separation logic. \emph{Communications of the ACM}, \emph{62}(2), 86–95. https://doi.org/\href{https://dx.doi.org/10.1145/3211968}{10.1145/3211968}\label{ohearn_separation_2019}%mdk%mdk

%mdk-data-line={FormalReview.bib:3084}
\bibitem{ohearn_local_2001}\mdbibitemlabel{}O’Hearn, P., Reynolds, J., \& Yang, H.~(2001). Local Reasoning about Programs that Alter Data Structures. In L.~Fribourg (Ed.), \emph{Computer Science Logic} (pp. 1–19). Springer Berlin Heidelberg.\label{ohearn_local_2001}%mdk%mdk

%mdk-data-line={FormalReview.bib:88}
\bibitem{ohearn_continuous_2018}\mdbibitemlabel{}O’Hearn, P.~W.~(2018). Continuous Reasoning: Scaling the impact of formal methods. In \emph{Proceedings of the 33rd Annual ACM/IEEE Symposium on Logic in Computer Science~- LICS ’18} (pp. 13–25). Oxford, United Kingdom: ACM Press. https://doi.org/\href{https://dx.doi.org/10.1145/3209108.3209109}{10.1145/3209108.3209109}\label{ohearn_continuous_2018}%mdk%mdk

%mdk-data-line={FormalReview.bib:66}
\bibitem{ohearn_peter_nodate}\mdbibitemlabel{}O’Hearn, P.~W.~(n.d.). Peter W O’hearn - acm profile. Retrieved from \href{https://dl.acm.org/author_page.cfm?id=81332519314\%26coll=DL\%26dl=ACM\%26trk=0}{{\ttfamily https://\hspace{0pt}dl.\hspace{0pt}acm.\hspace{0pt}org/\hspace{0pt}author\_\hspace{0pt}page.\hspace{0pt}cfm?\hspace{0pt}id=\hspace{0pt}81332519314\&\hspace{0pt}coll=\hspace{0pt}DL\&\hspace{0pt}dl=\hspace{0pt}ACM\&\hspace{0pt}trk=\hspace{0pt}0}}\label{ohearn_peter_nodate}%mdk%mdk

%mdk-data-line={FormalReview.bib:2624}
\bibitem{pakin_comprehensive_nodate}\mdbibitemlabel{}Pakin, S.~(n.d.). The Comprehensive LaTeX Symbol List, 358.\label{pakin_comprehensive_nodate}%mdk%mdk

%mdk-data-line={FormalReview.bib:1721}
\bibitem{parigot_logic_2000}\mdbibitemlabel{}Parigot, M., \& Voronkov, A.~(2000). \emph{Logic for Programming and Automated Reasoning: 7th International Conference, LPAR 2000 Reunion Island, France, November 6-10, 2000 Proceedings}. Berlin, Heidelberg: Springer Berlin Heidelberg.\label{parigot_logic_2000}%mdk%mdk

%mdk-data-line={FormalReview.bib:1277}
\bibitem{parkinson_relationship_nodate}\mdbibitemlabel{}Parkinson, M.~J., \& Summers, A.~J.~(n.d.). The Relationship between Separation Logic and Implicit Dynamic Frames. \emph{LNCS}, \emph{6602}, 439–458.\label{parkinson_relationship_nodate}%mdk%mdk

%mdk-data-line={FormalReview.bib:2697}
\bibitem{patterson_compositional_nodate-1}\mdbibitemlabel{}Patterson, D., \& Ahmed, A.~(n.d.-a). On Compositional Compiler Correctness and Fully Abstract Compilation, 3. Retrieved from \href{https://popl18.sigplan.org/event/prisc-2018-on-compositional-compiler-correctness-and-fully-abstract-compilation}{{\ttfamily https://\hspace{0pt}popl18.\hspace{0pt}sigplan.\hspace{0pt}org/\hspace{0pt}event/\hspace{0pt}prisc-\hspace{0pt}2018-\hspace{0pt}on-\hspace{0pt}compositional-\hspace{0pt}compiler-\hspace{0pt}correctness-\hspace{0pt}and-\hspace{0pt}fully-\hspace{0pt}abstract-\hspace{0pt}compilation}}\label{patterson_compositional_nodate-1}%mdk%mdk

%mdk-data-line={FormalReview.bib:2689}
\bibitem{patterson_compositional_nodate}\mdbibitemlabel{}Patterson, D., \& Ahmed, A.~(n.d.-b). On Compositional Compiler Correctness and Fully Abstract Compilation - POPL 2018. Retrieved February 1, 2019, from \href{https://popl18.sigplan.org/event/prisc-2018-on-compositional-compiler-correctness-and-fully-abstract-compilation}{{\ttfamily https://\hspace{0pt}popl18.\hspace{0pt}sigplan.\hspace{0pt}org/\hspace{0pt}event/\hspace{0pt}prisc-\hspace{0pt}2018-\hspace{0pt}on-\hspace{0pt}compositional-\hspace{0pt}compiler-\hspace{0pt}correctness-\hspace{0pt}and-\hspace{0pt}fully-\hspace{0pt}abstract-\hspace{0pt}compilation}}\label{patterson_compositional_nodate}%mdk%mdk

%mdk-data-line={FormalReview.bib:2449}
\bibitem{paulson_foundation_2000}\mdbibitemlabel{}Paulson, L.~C.~(2000). The Foundation of a Generic Theorem Prover. \emph{arXiv:cs/9301105}. Retrieved from arXiv:\href{http://arxiv.org/abs/cs/9301105}{cs/9301105}\label{paulson_foundation_2000}%mdk%mdk

%mdk-data-line={FormalReview.bib:1479}
\bibitem{petcher_foundational_2015}\mdbibitemlabel{}Petcher, A., \& Morrisett, G.~(2015). The Foundational Cryptography Framework. In R.~Focardi \& A.~Myers (Eds.), \emph{Principles of Security and Trust} (pp. 53–72). Springer Berlin Heidelberg. Retrieved from \href{http://www.cs.cornell.edu/~jgm/papers/FCF.pdf}{{\ttfamily http://\hspace{0pt}www.\hspace{0pt}cs.\hspace{0pt}cornell.\hspace{0pt}edu/\hspace{0pt}\textasciitilde{}jgm/\hspace{0pt}papers/\hspace{0pt}FCF.\hspace{0pt}pdf}}\label{petcher_foundational_2015}%mdk%mdk

%mdk-data-line={FormalReview.bib:1886}
\bibitem{petiot_your_2015}\mdbibitemlabel{}Petiot, G., Kosmatov, N., Botella, B., Giorgetti, A., \& Julliand, J.~(2015). Your Proof Fails? Testing Helps to Find the Reason. \emph{arXiv:1508.01691 {}[cs]}. Retrieved from arXiv:\href{http://arxiv.org/abs/1508.01691}{1508.01691}\label{petiot_your_2015}%mdk%mdk

%mdk-data-line={FormalReview.bib:1868}
\bibitem{petiot_how_2018}\mdbibitemlabel{}Petiot, G., Kosmatov, N., Botella, B., Giorgetti, A., \& Julliand, J.~(2018). How testing helps to diagnose proof failures. \emph{Form Asp Comp}, \emph{30}(6), 629–657. https://doi.org/\href{https://dx.doi.org/10.1007/s00165-018-0456-4}{10.1007/s00165-018-0456-4}\label{petiot_how_2018}%mdk%mdk

%mdk-data-line={FormalReview.bib:796}
\bibitem{pit-claudel_clement_nodate}\mdbibitemlabel{}Pit-Claudel, C.~(n.d.). Clément Pit-Claudel. Retrieved January 31, 2019, from \href{http://pit-claudel.fr/clement/}{{\ttfamily http://\hspace{0pt}pit-\hspace{0pt}claudel.\hspace{0pt}fr/\hspace{0pt}clement/\hspace{0pt}}}\label{pit-claudel_clement_nodate}%mdk%mdk

%mdk-data-line={FormalReview.bib:729}
\bibitem{pit-claudel_extensible_nodate}\mdbibitemlabel{}Pit-Claudel, C., Wang, P., Delaware, B., Gross, J., \& Chlipala, A.~(n.d.). Extensible Extraction of Efficient Imperative Programs with Foreign Functions, Manually Managed Memory, and Proofs, 14. Retrieved from \href{http://pit-claudel.fr/clement/papers/fiat-to-facade.pdf}{{\ttfamily http://\hspace{0pt}pit-\hspace{0pt}claudel.\hspace{0pt}fr/\hspace{0pt}clement/\hspace{0pt}papers/\hspace{0pt}fiat-\hspace{0pt}to-\hspace{0pt}facade.\hspace{0pt}pdf}}\label{pit-claudel_extensible_nodate}%mdk%mdk

%mdk-data-line={FormalReview.bib:899}
\bibitem{platzer_differential_2008}\mdbibitemlabel{}Platzer, A.~(2008). Differential Dynamic Logic for Hybrid Systems. \emph{Journal of Automated Reasoning}, \emph{41}(2), 143–189. https://doi.org/\href{https://dx.doi.org/10.1007/s10817-008-9103-8}{10.1007/s10817-008-9103-8}\label{platzer_differential_2008}%mdk%mdk

%mdk-data-line={FormalReview.bib:882}
\bibitem{platzer_differential_2015}\mdbibitemlabel{}Platzer, A.~(2015). Differential Game Logic. \emph{ACM Trans. Comput. Logic}, \emph{17}(1), 1:1–1:51. https://doi.org/\href{https://dx.doi.org/10.1145/2817824}{10.1145/2817824}\label{platzer_differential_2015}%mdk%mdk

%mdk-data-line={FormalReview.bib:992}
\bibitem{platzer_complete_2017}\mdbibitemlabel{}Platzer, A.~(2017). A Complete Uniform Substitution Calculus for Differential Dynamic Logic. \emph{Journal of Automated Reasoning}, \emph{59}(2), 219–265. https://doi.org/\href{https://dx.doi.org/10.1007/s10817-016-9385-1}{10.1007/s10817-016-9385-1}\label{platzer_complete_2017}%mdk%mdk

%mdk-data-line={FormalReview.bib:865}
\bibitem{platzer_differential_2018}\mdbibitemlabel{}Platzer, A.~(2018a). Differential Equations \& Differential Invariants. In A.~Platzer, \emph{Logical Foundations of Cyber-Physical Systems} (pp. 287–322). Cham: Springer International Publishing. https://doi.org/\href{https://dx.doi.org/10.1007/978-3-319-63588-0_10}{10.1007/978-3-319-63588-0\_10}\label{platzer_differential_2018}%mdk%mdk

%mdk-data-line={FormalReview.bib:937}
\bibitem{platzer_logical_2018}\mdbibitemlabel{}Platzer, A.~(2018). \emph{Logical Foundations of Cyber-Physical Systems}. Springer International Publishing. Retrieved from \href{https://www.springer.com/gp/book/9783319635873}{{\ttfamily https://\hspace{0pt}www.\hspace{0pt}springer.\hspace{0pt}com/\hspace{0pt}gp/\hspace{0pt}book/\hspace{0pt}9783319635873}}\label{platzer_logical_2018}%mdk%mdk

%mdk-data-line={FormalReview.bib:978}
\bibitem{platzer_logical_2018-1}\mdbibitemlabel{}Platzer, A.~(2018b). \emph{Logical Foundations of Cyber-Physical Systems - Slides}. Cham: Springer International Publishing. https://doi.org/\href{https://dx.doi.org/10.1007/978-3-319-63588-0}{10.1007/978-3-319-63588-0}\label{platzer_logical_2018-1}%mdk%mdk

%mdk-data-line={FormalReview.bib:950}
\bibitem{platzer_keymaera_nodate}\mdbibitemlabel{}Platzer, A.~(n.d.). KeYmaera X: Documentation. Retrieved January 31, 2019, from \href{http://www.ls.cs.cmu.edu/KeYmaeraX/documentation.html}{{\ttfamily http://\hspace{0pt}www.\hspace{0pt}ls.\hspace{0pt}cs.\hspace{0pt}cmu.\hspace{0pt}edu/\hspace{0pt}KeYmaeraX/\hspace{0pt}documentation.\hspace{0pt}html}}\label{platzer_keymaera_nodate}%mdk%mdk

%mdk-data-line={FormalReview.bib:1128}
\bibitem{polikarpova_structuring_2019}\mdbibitemlabel{}Polikarpova, N., \& Sergey, I.~(2019). Structuring the Synthesis of Heap-manipulating Programs. \emph{Proc. ACM Program. Lang.}, \emph{3}, 72:1–72:30. https://doi.org/\href{https://dx.doi.org/10.1145/3290385}{10.1145/3290385}\label{polikarpova_structuring_2019}%mdk%mdk

%mdk-data-line={FormalReview.bib:2765}
\bibitem{pottier_menhir_nodate}\mdbibitemlabel{}Pottier, F., \& REgis-Gianas, Y.~(n.d.). Menhir Reference Manual (version 20181113). Retrieved February 1, 2019, from \href{http://gallium.inria.fr/~fpottier/menhir/manual.html}{{\ttfamily http://\hspace{0pt}gallium.\hspace{0pt}inria.\hspace{0pt}fr/\hspace{0pt}\textasciitilde{}fpottier/\hspace{0pt}menhir/\hspace{0pt}manual.\hspace{0pt}html}}\label{pottier_menhir_nodate}%mdk%mdk

%mdk-data-line={FormalReview.bib:1782}
\bibitem{protzenko_verified_2017}\mdbibitemlabel{}Protzenko, J., Zinzindohoué, J.-K., Rastogi, A., Ramananandro, T., Wang, P., Zanella-Béguelin, S., … Swamy, N.~(2017). Verified Low-level Programming Embedded in F*. \emph{Proc. ACM Program. Lang.}, \emph{1}, 17:1–17:29. https://doi.org/\href{https://dx.doi.org/10.1145/3110261}{10.1145/3110261}\label{protzenko_verified_2017}%mdk%mdk

%mdk-data-line={FormalReview.bib:166}
\bibitem{qureshi_formal_nodate}\mdbibitemlabel{}Qureshi, Z.~H.~(n.d.). Formal Modelling and Analysis of Mission-Critical Software in Military Avionics Systems. \emph{11th Australian Workshop on Safety Related Programmable Systems (SCS’06)}, 11. Retrieved from \href{http://crpit.com/confpapers/CRPITV69Qureshi.pdf}{{\ttfamily http://\hspace{0pt}crpit.\hspace{0pt}com/\hspace{0pt}confpapers/\hspace{0pt}CRPITV69Qureshi.\hspace{0pt}pdf}}\label{qureshi_formal_nodate}%mdk%mdk

%mdk-data-line={FormalReview.bib:2142}
\bibitem{ramsey_applicative_2006}\mdbibitemlabel{}Ramsey, N., \& Dias, J.~(2006). An Applicative Control-Flow Graph Based on Huet’s Zipper. \emph{Electronic Notes in Theoretical Computer Science}, \emph{148}(2), 105–126. https://doi.org/\href{https://dx.doi.org/10.1016/j.entcs.2005.11.042}{10.1016/j.entcs.2005.11.042}\label{ramsey_applicative_2006}%mdk%mdk

%mdk-data-line={FormalReview.bib:461}
\bibitem{sherman_making_2017}\mdbibitemlabel{}Sherman, B.~(2017). \emph{Making Discrete Decisions Based on Continuous Values} (Master of Science). MIT, Cambridge, MA.~Retrieved from \href{http://adam.chlipala.net/theses/sherman_sm.pdf}{{\ttfamily http://\hspace{0pt}adam.\hspace{0pt}chlipala.\hspace{0pt}net/\hspace{0pt}theses/\hspace{0pt}sherman\_\hspace{0pt}sm.\hspace{0pt}pdf}}\label{sherman_making_2017}%mdk%mdk

%mdk-data-line={FormalReview.bib:3013}
\bibitem{shrobe_trust-management_2009}\mdbibitemlabel{}Shrobe, H., DeHon, A., \& Knight, T.~(2009). Trust-Management, Intrusion-Tolerance, Accountability, and Reconstitution Architecture (TIARA), 133. Retrieved from \href{https://apps.dtic.mil/dtic/tr/fulltext/u2/a511350.pdf}{{\ttfamily https://\hspace{0pt}apps.\hspace{0pt}dtic.\hspace{0pt}mil/\hspace{0pt}dtic/\hspace{0pt}tr/\hspace{0pt}fulltext/\hspace{0pt}u2/\hspace{0pt}a511350.\hspace{0pt}pdf}}\label{shrobe_trust-management_2009}%mdk%mdk

%mdk-data-line={FormalReview.bib:2361}
\bibitem{shulman_hott_2013}\mdbibitemlabel{}Shulman, M.~(2013, March 12). The HoTT Book.Homotopy Type Theory. Retrieved February 1, 2019, from \href{https://homotopytypetheory.org/book/}{{\ttfamily https://\hspace{0pt}homotopytypetheory.\hspace{0pt}org/\hspace{0pt}book/\hspace{0pt}}}\label{shulman_hott_2013}%mdk%mdk

%mdk-data-line={FormalReview.bib:1401}
\bibitem{sozeau_equations:_2010}\mdbibitemlabel{}Sozeau, M.~(2010). Equations: A Dependent Pattern-Matching Compiler. In M.~Kaufmann \& L.~C.~Paulson (Eds.), \emph{Interactive Theorem Proving} (pp. 419–434). Springer Berlin Heidelberg.\label{sozeau_equations:_2010}%mdk%mdk

%mdk-data-line={FormalReview.bib:1710}
\bibitem{sozeau_metacoq_2019}\mdbibitemlabel{}Sozeau, M.~(2019). \emph{MetaCoq - Metaprogramming in Coq (Was template-coq)}. MetaCoq. Retrieved from \href{https://github.com/MetaCoq/metacoq}{{\ttfamily https://\hspace{0pt}github.\hspace{0pt}com/\hspace{0pt}MetaCoq/\hspace{0pt}metacoq}}\label{sozeau_metacoq_2019}%mdk%mdk

%mdk-data-line={FormalReview.bib:1329}
\bibitem{sozeau_typed_nodate}\mdbibitemlabel{}Sozeau, M.~(n.d.-a). Typed Template Coq - POPL 2018. Retrieved February 1, 2019, from \href{https://popl18.sigplan.org/event/coqpl-2018-typed-template-coq}{{\ttfamily https://\hspace{0pt}popl18.\hspace{0pt}sigplan.\hspace{0pt}org/\hspace{0pt}event/\hspace{0pt}coqpl-\hspace{0pt}2018-\hspace{0pt}typed-\hspace{0pt}template-\hspace{0pt}coq}}\label{sozeau_typed_nodate}%mdk%mdk

%mdk-data-line={FormalReview.bib:1346}
\bibitem{sozeau_typed_nodate-1}\mdbibitemlabel{}Sozeau, M.~(n.d.-b). Typed Template Coq - Slides, 11.\label{sozeau_typed_nodate-1}%mdk%mdk

%mdk-data-line={FormalReview.bib:2193}
\bibitem{spector-zabusky_total_2018}\mdbibitemlabel{}Spector-Zabusky, A., Breitner, J., Rizkallah, C., \& Weirich, S.~(2018). Total Haskell is Reasonable Coq. In \emph{Proceedings of the 7th ACM SIGPLAN International Conference on Certified Programs and Proofs} (pp. 14–27). New York, NY, USA: ACM.~https://doi.org/\href{https://dx.doi.org/10.1145/3167092}{10.1145/3167092}\label{spector-zabusky_total_2018}%mdk%mdk

%mdk-data-line={FormalReview.bib:349}
\bibitem{stewart_verified_2012}\mdbibitemlabel{}Stewart, G., Beringer, L., \& Appel, A.~W.~(2012). Verified Heap Theorem Prover by Paramodulation. In \emph{Proceedings of the 17th ACM SIGPLAN International Conference on Functional Programming} (pp. 3–14). New York, NY, USA: ACM.~https://doi.org/\href{https://dx.doi.org/10.1145/2364527.2364531}{10.1145/2364527.2364531}\label{stewart_verified_2012}%mdk%mdk

%mdk-data-line={FormalReview.bib:2232}
\bibitem{swamy_project_nodate}\mdbibitemlabel{}Swamy, N.~(n.d.). Project Everest - Verified Secure Implementations of the HTTPS Ecosystem.Microsoft Research. Retrieved February 1, 2019, from \href{https://www.microsoft.com/en-us/research/project/project-everest-verified-secure-implementations-https-ecosystem/}{{\ttfamily https://\hspace{0pt}www.\hspace{0pt}microsoft.\hspace{0pt}com/\hspace{0pt}en-\hspace{0pt}us/\hspace{0pt}research/\hspace{0pt}project/\hspace{0pt}project-\hspace{0pt}everest-\hspace{0pt}verified-\hspace{0pt}secure-\hspace{0pt}implementations-\hspace{0pt}https-\hspace{0pt}ecosystem/\hspace{0pt}}}\label{swamy_project_nodate}%mdk%mdk

%mdk-data-line={FormalReview.bib:1942}
\bibitem{swamy_verifying_2013}\mdbibitemlabel{}Swamy, N., Chen, J., \& Livshits, B.~(2013). Verifying Higher-order Programs with the Dijkstra Monad. Retrieved from \href{https://www.microsoft.com/en-us/research/publication/verifying-higher-order-programs-with-the-dijkstra-monad/}{{\ttfamily https://\hspace{0pt}www.\hspace{0pt}microsoft.\hspace{0pt}com/\hspace{0pt}en-\hspace{0pt}us/\hspace{0pt}research/\hspace{0pt}publication/\hspace{0pt}verifying-\hspace{0pt}higher-\hspace{0pt}order-\hspace{0pt}programs-\hspace{0pt}with-\hspace{0pt}the-\hspace{0pt}dijkstra-\hspace{0pt}monad/\hspace{0pt}}}\label{swamy_verifying_2013}%mdk%mdk

%mdk-data-line={FormalReview.bib:1953}
\bibitem{swamy_dependent_2016}\mdbibitemlabel{}Swamy, N., Hriţcu, C., Keller, C., Rastogi, A., Delignat-Lavaud, A., Forest, S., … Zanella-Béguelin, S.~(2016). Dependent Types and Multi-monadic Effects in F*. In \emph{Proceedings of the 43rd Annual ACM SIGPLAN-SIGACT Symposium on Principles of Programming Languages} (pp. 256–270). New York, NY, USA: ACM.~https://doi.org/\href{https://dx.doi.org/10.1145/2837614.2837655}{10.1145/2837614.2837655}\label{swamy_dependent_2016}%mdk%mdk

%mdk-data-line={FormalReview.bib:2016}
\bibitem{syme_fsharp_2019}\mdbibitemlabel{}Syme, D.~(2019a). \emph{Fsharp design: RFCs and docs related to the F\# language design process,}. F\# Software Foundation Repositories. Retrieved from \href{https://github.com/fsharp/fslang-design}{{\ttfamily https://\hspace{0pt}github.\hspace{0pt}com/\hspace{0pt}fsharp/\hspace{0pt}fslang-\hspace{0pt}design}}\label{syme_fsharp_2019}%mdk%mdk

%mdk-data-line={FormalReview.bib:2027}
\bibitem{syme_fsharp_2019-1}\mdbibitemlabel{}Syme, D.~(2019b). \emph{The Fsharp Compiler, Core Library \& Tools (F\# Software Foundation Repository): fsharp/fsharp}. F\# Software Foundation Repositories. Retrieved from \href{https://github.com/fsharp/fsharp}{{\ttfamily https://\hspace{0pt}github.\hspace{0pt}com/\hspace{0pt}fsharp/\hspace{0pt}fsharp}}\label{syme_fsharp_2019-1}%mdk%mdk

%mdk-data-line={FormalReview.bib:2853}
\bibitem{van_renesse_paxos_2015}\mdbibitemlabel{}Van Renesse, R., \& Altinbuken, D.~(2015). Paxos Made Moderately Complex. \emph{ACM Comput. Surv.}, \emph{47}(3), 42:1–42:36. https://doi.org/\href{https://dx.doi.org/10.1145/2673577}{10.1145/2673577}\label{van_renesse_paxos_2015}%mdk%mdk

%mdk-data-line={FormalReview.bib:2373}
\bibitem{voevodsky_homotopy_nodate}\mdbibitemlabel{}Voevodsky, V.~(n.d.). Homotopy Type Theory: Univalent Foundations of Mathematics, 490.\label{voevodsky_homotopy_nodate}%mdk%mdk

%mdk-data-line={FormalReview.bib:2463}
\bibitem{wenzel_isabelle/isar_2018}\mdbibitemlabel{}Wenzel, M.~(2018). The Isabelle/Isar Reference Manual. Retrieved from \href{https://core.ac.uk/display/22830292}{{\ttfamily https://\hspace{0pt}core.\hspace{0pt}ac.\hspace{0pt}uk/\hspace{0pt}display/\hspace{0pt}22830292}}\label{wenzel_isabelle/isar_2018}%mdk%mdk

%mdk-data-line={FormalReview.bib:1644}
\bibitem{white_neil_formal_2017}\mdbibitemlabel{}White Neil, Matthews Stuart, \& Chapman Roderick. (2017). Formal verification: will the seedling ever flower? \emph{Philosophical Transactions of the Royal Society A: Mathematical, Physical and Engineering Sciences}, \emph{375}(2104), 20150402. https://doi.org/\href{https://dx.doi.org/10.1098/rsta.2015.0402}{10.1098/rsta.2015.0402}\label{white_neil_formal_2017}%mdk%mdk

%mdk-data-line={FormalReview.bib:1457}
\bibitem{wiedijk_formal_2008}\mdbibitemlabel{}Wiedijk, F.~(2008). Formal Proof—Getting Started, \emph{55}(11), 7.\label{wiedijk_formal_2008}%mdk%mdk

%mdk-data-line={FormalReview.bib:151}
\bibitem{wikipedia_category:formal_2017}\mdbibitemlabel{}Wikipedia. (2017). Category:Formal methods people. In \emph{Wikipedia}. Retrieved from \href{https://en.wikipedia.org/w/index.php?title=Category:Formal_methods_people\%26oldid=812800009}{{\ttfamily https://\hspace{0pt}en.\hspace{0pt}wikipedia.\hspace{0pt}org/\hspace{0pt}w/\hspace{0pt}index.\hspace{0pt}php?\hspace{0pt}title=\hspace{0pt}Category:Formal\_\hspace{0pt}methods\_\hspace{0pt}people\&\hspace{0pt}oldid=\hspace{0pt}812800009}}\label{wikipedia_category:formal_2017}%mdk%mdk

%mdk-data-line={FormalReview.bib:2309}
\bibitem{yang_safe_2011}\mdbibitemlabel{}Yang, J., \& Hawblitzel, C.~(2011). Safe to the last instruction: automated verification of a type-safe operating system. \emph{Communications of the ACM}, \emph{54}(12), 123. https://doi.org/\href{https://dx.doi.org/10.1145/2043174.2043197}{10.1145/2043174.2043197}\label{yang_safe_2011}%mdk%mdk

%mdk-data-line={FormalReview.bib:740}
\bibitem{feng_correct-by-construction_2018}\mdbibitemlabel{}Zhang, T., Wiegley, J., Giannakopoulos, T., Eakman, G., Pit-Claudel, C., Lee, I., \& Sokolsky, O.~(2018). Correct-by-Construction Implementation of Runtime Monitors Using Stepwise Refinement. In X.~Feng, M.~Müller-Olm, \& Z.~Yang (Eds.), \emph{Dependable Software Engineering. Theories, Tools, and Applications} (Vol. 10998, pp. 31–49). Cham: Springer International Publishing. https://doi.org/\href{https://dx.doi.org/10.1007/978-3-319-99933-3_3}{10.1007/978-3-319-99933-3\_3}\label{feng_correct-by-construction_2018}%mdk%mdk

%mdk-data-line={FormalReview.bib:1313}
\bibitem{acm_acm_nodate}\mdbibitemlabel{}(n.d.-a). ACM Classification Codes. Retrieved February 1, 2019, from \href{https://cran.r-project.org/web/classifications/ACM.html}{{\ttfamily https://\hspace{0pt}cran.\hspace{0pt}r-\hspace{0pt}project.\hspace{0pt}org/\hspace{0pt}web/\hspace{0pt}classifications/\hspace{0pt}ACM.\hspace{0pt}html}}\label{acm_acm_nodate}%mdk%mdk

%mdk-data-line={FormalReview.bib:1094}
\bibitem{acm_coq_nodate}\mdbibitemlabel{}(n.d.-b). Coq for PL conference series - CoqPL 2019. Retrieved January 31, 2019, from \href{https://popl18.sigplan.org/series/CoqPL}{{\ttfamily https://\hspace{0pt}popl18.\hspace{0pt}sigplan.\hspace{0pt}org/\hspace{0pt}series/\hspace{0pt}CoqPL}}\label{acm_coq_nodate}%mdk%mdk

%mdk-data-line={FormalReview.bib:1086}
\bibitem{acm_coqpl_nodate-1}\mdbibitemlabel{}(n.d.-c). CoqPL 2018 The Fourth International Workshop on Coq for Programming Languages - POPL 2018. Retrieved January 31, 2019, from \href{https://popl18.sigplan.org/track/CoqPL-2018}{{\ttfamily https://\hspace{0pt}popl18.\hspace{0pt}sigplan.\hspace{0pt}org/\hspace{0pt}track/\hspace{0pt}CoqPL-\hspace{0pt}2018}}\label{acm_coqpl_nodate-1}%mdk%mdk

%mdk-data-line={FormalReview.bib:1078}
\bibitem{acm_coqpl_nodate}\mdbibitemlabel{}(n.d.-d). CoqPL 2019 The Fifth International Workshop on Coq for Programming Languages - POPL 2019. Retrieved January 31, 2019, from \href{https://popl19.sigplan.org/track/CoqPL-2019\%23program}{{\ttfamily https://\hspace{0pt}popl19.\hspace{0pt}sigplan.\hspace{0pt}org/\hspace{0pt}track/\hspace{0pt}CoqPL-\hspace{0pt}2019\#\hspace{0pt}program}}\label{acm_coqpl_nodate}%mdk%mdk

%mdk-data-line={FormalReview.bib:1810}
\bibitem{cea_frama-c_nodate}\mdbibitemlabel{}(n.d.-e). Frama-C.~Retrieved February 1, 2019, from \href{https://frama-c.com/}{{\ttfamily https://\hspace{0pt}frama-\hspace{0pt}c.\hspace{0pt}com/\hspace{0pt}}}\label{cea_frama-c_nodate}%mdk%mdk

%mdk-data-line={FormalReview.bib:2616}
\bibitem{wikibook_latex_nodate}\mdbibitemlabel{}(n.d.-f). LaTeX - Wikibooks, open books for an open world. Retrieved February 1, 2019, from \href{https://en.wikibooks.org/wiki/LaTeX}{{\ttfamily https://\hspace{0pt}en.\hspace{0pt}wikibooks.\hspace{0pt}org/\hspace{0pt}wiki/\hspace{0pt}LaTeX}}\label{wikibook_latex_nodate}%mdk%mdk

%mdk-data-line={FormalReview.bib:1321}
\bibitem{acm_msc2010_nodate}\mdbibitemlabel{}(n.d.-g). MSC2010 database. Retrieved February 1, 2019, from \href{https://mathscinet.ams.org/msc/msc2010.html}{{\ttfamily https://\hspace{0pt}mathscinet.\hspace{0pt}ams.\hspace{0pt}org/\hspace{0pt}msc/\hspace{0pt}msc2010.\hspace{0pt}html}}\label{acm_msc2010_nodate}%mdk%mdk

%mdk-data-line={FormalReview.bib:1512}
\bibitem{royalsociety_philosophical_nodate}\mdbibitemlabel{}(n.d.-h). Philosophical Transactions of the Royal Society A: Mathematical, Physical and Engineering Sciences. Retrieved February 1, 2019, from \href{https://royalsocietypublishing.org/journal/rsta}{{\ttfamily https://\hspace{0pt}royalsocietypublishing.\hspace{0pt}org/\hspace{0pt}journal/\hspace{0pt}rsta}}\label{royalsociety_philosophical_nodate}%mdk%mdk

%mdk-data-line={FormalReview.bib:1102}
\bibitem{acm_popl_nodate}\mdbibitemlabel{}(n.d.-i). POPL conference series - POPL 2020. Retrieved January 31, 2019, from \href{https://popl18.sigplan.org/series/POPL}{{\ttfamily https://\hspace{0pt}popl18.\hspace{0pt}sigplan.\hspace{0pt}org/\hspace{0pt}series/\hspace{0pt}POPL}}\label{acm_popl_nodate}%mdk%mdk

%mdk-data-line={FormalReview.bib:1520}
\bibitem{royalsociety_proceedings_nodate}\mdbibitemlabel{}(n.d.-j). Proceedings of the Royal Society A: Mathematical, Physical and Engineering Sciences. Retrieved February 1, 2019, from \href{https://royalsocietypublishing.org/journal/rspa}{{\ttfamily https://\hspace{0pt}royalsocietypublishing.\hspace{0pt}org/\hspace{0pt}journal/\hspace{0pt}rspa}}\label{royalsociety_proceedings_nodate}%mdk%mdk
\par%mdk
\end{thebibliography}}%mdk%mdk%mdk


\end{document}
